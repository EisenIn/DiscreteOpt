\chapter{Termination, Cycling, and Degeneracy}
\label{cha:cycling-degeneracy}


We now deal first  with the question, whether the simplex method
terminates. The quick answer is no, if it is implemented in a careless
way. Notice that we have left it open to choose the entering variable
(there could be several columns whose reduced cost are less than zero)
and we also did not specify how to choose the exiting variable (there
could be several basis elements $B_i$ with $u(B_i)>0$ and
$x^*(B_i)/u(B_i)$ minimal to choose from. \emph{Pivoting rules}
specify the alternative to choose from and there are pivoting rules,
which prevent the simplex algorithm from \emph{cycling}. 



The following is an example of cycling, see also
\cite[p.~104]{BertsimasTsitsiklis97}. 



\begin{displaymath}
  \begin{tabularx}{0.7\textwidth}{AAAARRR|X}   
    \rowcolor{white}  
    -3/4&   20& -1/2&    6&    0&    0&    0&    3\\ \hline
    1/4&   -8&   -1&    9&    1&    0&    0&    0\\
    1/2&  -12& -1/2&    3&    0&    1&    0&    0\\
    0&    0&    1&    0&    0&    0&    1&    1\\
  \end{tabularx}
  %\quad \quad \quad   B = \{5,6,7\} 
\end{displaymath}

         

\begin{displaymath}
  \begin{tabularx}{0.7\textwidth}{RAAAARR|X}   
    \rowcolor{white}  
    0&   -4& -7/2&   33&    3&    0&    0&    3\\ \hline
    1&  -32&   -4&   36&    4&    0&    0&    0\\
    0&    4&  3/2&  -15&   -2&    1&    0&    0\\
    0&    0&    1&    0&    0&    0&    1&    1\\
\end{tabularx}
%\quad \quad \quad   B = \{1,6,7\} 
\end{displaymath}


\begin{displaymath}
  \begin{tabularx}{0.7\textwidth}{RRAAAAR|X}   
    \rowcolor{white}  
     0&     0&    -2&    18&     1&     1&     0&     3\\ \hline
     1&     0&     8&   -84&   -12&     8&     0&     0\\
     0&     1&   3/8& -15/4&  -1/2&   1/4&     0&     0\\
     0&     0&     1&     0&     0&     0&     1&     1\\
\end{tabularx}
%\quad \quad \quad   B = \{1,2,7\} 
\end{displaymath}


\begin{displaymath}
  \begin{tabularx}{0.7\textwidth}{ARRAAAR|X}   
    \rowcolor{white}  
   1/4&     0&     0&    -3&    -2&     3&     0&     3\\ \hline
   -3/64&     1&     0&  3/16&  1/16&  -1/8&     0&     0\\
   1/8&     0&     1& -21/2&  -3/2&     1&     0&     0\\
  -1/8&     0&     0&  21/2&   3/2&    -1&     1&     1\\
\end{tabularx}
%\quad \quad \quad   B = \{2,3,7\} 
\end{displaymath}

\begin{displaymath}
  \begin{tabularx}{0.7\textwidth}{AARAAAR|X}   
    \rowcolor{white}  
 -1/2&   16&    0&    0&   -1&    1&    0&    3\\ \hline
 -5/2&   56&    1&    0&    2&   -6&    0&    0\\
 -1/4& 16/3&    0&    1&  1/3& -2/3&    0&    0\\
  5/2&  -56&    0&    0&   -2&    6&    1&    1\\
\end{tabularx}
%\quad \quad \quad   B = \{3,4,7\} 
\end{displaymath}


\begin{displaymath}
  \begin{tabularx}{0.7\textwidth}{AAARRAR|X}   
    \rowcolor{white}  
 -7/4&   44&  1/2&    0&    0&   -2&    0&    3\\ \hline
  1/6&   -4& -1/6&    1&    0&  1/3&    0&    0\\
  -5/4&   28&  1/2&    0&    1&   -3&    0&    0\\
    0&    0&    1&    0&    0&    0&    1&    1\\
\end{tabularx}
%\quad \quad \quad   B = \{4,5,7\} 
\end{displaymath}


\begin{displaymath}
  \begin{tabularx}{0.7\textwidth}{AAAARRR|X}   
    \rowcolor{white}  
 -3/4&   20& -1/2&    6&    0&    0&    0&    3\\ \hline
  1/4&   -8&   -1&    9&    1&    0&    0&    0\\
  1/2&  -12& -1/2&    3&    0&    1&    0&    0\\
    0&    0&    1&    0&    0&    0&    1&    1\\
  \end{tabularx}
%\quad \quad \quad   B = \{5,6,7\} 
\end{displaymath}

In fact we have not made any progress in the objective function value
during all these pivots. Inspecting the pivot steps, we see that the
problem lies in the fact that we have basic feasible solutions which
are zero at some of the basic components. If this is never the case,
then the simplex method terminates, as we show now. 


Recall that a basis $B$ with reduced costs $\overline{c}\geq0$ is
optimal, see Lemma~\ref{lem:14}. There is a weak converse to this
lemma. Before we state it, we define the notion of degeneracy. 

\begin{definition}[Degenerate basic solution]
  \label{def:1}
  A basic solution $x^*$ associated to the basis $B$ is \emph{degenerate}, if
  there exists an index $i \in B$ with $x^*(i) =0$. If $x^*(i) \neq0$ for
  all $i \in B$, then $x^*$ is called \emph{non-degenerate}. 
\end{definition}


\begin{lemma}
  \label{lem:1}
  Let $x^*$ be a non-degenerate basic feasible solution associated to
  the basis $B= \{B_1,\ldots,B_m\}$. If there exists an index $i \in \{1,\ldots,n\}$ with
  $\overline{c}(i)<0$, then $x^*$ is not optimal. 
\end{lemma}

\begin{proof}
  Suppose that the reduced cost of variable $j$ is negative, 
  $\overline{c}(j)<0$ and inspect the tableau of $B$ 
  \begin{displaymath}
    \begin{array}{c|c}
      c^T - c_B^TA_B^{-1}A & -c_B^Tx^*_B \\ \hline 
      A_B^{-1}A        & x^*_B.
    \end{array}
  \end{displaymath}
Let $u(1),\ldots,u(m)$ be the components of the $j$-th column of the
system matrix of the tableau. If $u(i)\leq0$ for all $i$, then the linear program is
unbounded and the assertion follows. Otherwise let $K = \{ i \colon u(i) >
0\}$ and let $i^*\in K$ be an index where the minimum 
$\min \{x^*(B_i)/u(i) \colon i \in K\}$ is attained. The new objective
function value after the pivot step is 
\begin{equation}
  \label{eq:3}
  c^Tx^* + \overline{c}(j) x^*(B_{i^*})/u(i^*)
\end{equation}
and since $\overline{c}(j)<0$, this is smaller than $c^Tx^*$. \qed
  
\end{proof}



This Lemma also shows that the simplex method terminates if the basic
feasible solutions are all non-degenerate. 

\begin{theorem}
  \label{thr:2}
  If all basic feasible solutions are non-degenerate, then the simplex
  method terminates, regardless of the choice among the alternatives
  among entering and leaving variables. 
\end{theorem}


\begin{proof}
  The proof of the previous lemma, in particular equation~\eqref{eq:3}
  shows that  the objective function value strictly improves each
  iteration of the simplex method. Each tableau is uniquely defined by
  a basis $B$. We never re-visit a tableau, since the objective
  function is strictly increasing. The number of bases is bounded by
  $\binom{n}{m}$. The assertion follows. \qed
\end{proof}




\subsubsection*{Perturbation}

The termination argument for the simplex algorithm for linear programs,
who do not have degenerate basic feasible solutions, is very simple and
nice and in fact one can twist it into a pivot-rule for the
simplex-algorithm which makes the simplex algorithm terminate.  

We start with the  linear program 
\begin{equation}
  \label{eq:4}
  \min\{c^Tx \colon x \in \setR^n, \, Ax=b, \, x\geq0\}
\end{equation}
and assume that $S\subseteq\{1,\ldots,n\}$ is a feasible basis with $A_S
=I_m$. This assumption can be made, since we can replace $Ax = b$ with
$A_S^{-1}Ax = A_S^{-1}b$.

The idea is to transform \eqref{eq:4} into 
into a linear program 
\begin{equation}
  \label{eq:5}
  \min\{c^Tx \colon x \in \setR^n, \, Ax=b', \, x\geq0\}
\end{equation}
such that the following conditions hold. 
\begin{enumerate}[i)]
\item The LP~\eqref{eq:5} does not have degenerate basic solutions. \label{item:11}
\item If $B$ is an infeasible basis of \eqref{eq:4}, then $B$ is also an
  infeasible basis of~\eqref{eq:5}. \label{item:12}
\item The LP~\eqref{eq:5} is feasible. \label{item:13}
\end{enumerate}



\begin{lemma}
  \label{lem:2}
  The simplex algorithm terminates on the linear
  program~\eqref{eq:5}. If it 
  asserts that the linear program~\eqref{eq:5} is unbounded, then the
  linear program~\eqref{eq:4} is also unbounded. If it 
  outputs an optimal basis of~\eqref{eq:5} then this basis is also
  optimal for~\eqref{eq:4}. 
\end{lemma}

\begin{proof}
  
  
  It follows from Theorem~\ref{thr:2} and the property~\ref{item:11})
  that the simplex method terminates on the linear
  program~\eqref{eq:5}. If it asserts that~\eqref{eq:5} is unmounded,
  then there exists a $d\geq0$ with $Ad=0$ and $c^Td<0$. This means that
  the linear program~\eqref{eq:4} is also unbounded. 
  
  Suppose therefore that the simplex method on~\eqref{eq:5} terminates
  by outputting a 
  basis $\wt{B}$. This basis 
  $\wt{B}$ is a feasible basis of~\eqref{eq:4} by
  property~\ref{item:12}). The reduced costs of this basis are the
  same for the linear programs~\eqref{eq:4}  and~\eqref{eq:5} which
  implies that $\wt{B}$ is also an optimal basis of \eqref{eq:5}.
  \qed 
\end{proof}


Now we describe the construction of \eqref{eq:5}, more precisely the
construction of $b'$, which is 
\begin{equation}
  \label{eq:6}
  b' = b+ \mat{\varepsilon\\\varepsilon^2 \\\vdots \\\varepsilon^m}
\end{equation}
for some small $\varepsilon>0$. Since $\varepsilon>0$ the basic solution $x^*$ associated to
$S$ with 
\begin{displaymath}
  x^*_S=b+\mat{\varepsilon \\ \vdots\\ \varepsilon^m}\text{ and }x^*_{\overline{S}}=0 
\end{displaymath}
  is  a basic
feasible solution of~\eqref{eq:5} and thus~\ref{item:13}) follows. We show that we can
choose an $\varepsilon>0$ such that \ref{item:12}) holds. Suppose that $B$ is an
infeasible basis. This means that there exists an entry $i$ of $B$
such that $(A_B^{-1} \, b)(i) < 0$. Clearly, we can choose $\varepsilon_B>0$
small enough such that 
\begin{equation}
  \label{eq:7}
  \left[A_B^{-1}\,\left(b+\mat{\varepsilon\\\vdots\\\varepsilon^m}\right)\right](i)<0
\end{equation}
and the basis $B$ is still infeasible for the linear
program~\eqref{eq:5}. The most interesting property to infer is
property~\ref{item:11}). Let $B$ be any basis and consider the vector
\begin{equation}
  \label{eq:8}
  A_B^{-1} \left(b + \mat{\varepsilon\\\vdots\\\varepsilon^m}\right) 
\end{equation}
which determines the components of $x^*_B$ of the basic solution. If
we treat $\varepsilon$ as a variable, then the $i$-th component of this
vector~\eqref{eq:8} is a polynomial in $\varepsilon$ which we denote by
$p_{iB}(\varepsilon)$. Each of the polynomials $p_{iB}(\varepsilon)$ is unequal to the
zero polynomial,
see exercise~\ref{item:14} which implies that there are only a finite
(at most $m$) real-roots of $p_{iB}$. Consequently, there exists an
$\varepsilon_{iB}>0$ for each $p_{iB}$ such that $p_{iB}(\varepsilon) \neq 0$ for each $\varepsilon
\in ]0, \varepsilon_{iB}[$. 
Thus if we choose $\varepsilon>0$ such that it is smaller than all $\varepsilon_B$ and
all $\varepsilon_{iB}$ from above, then the perturbed problem with 
\begin{displaymath}
  b' = b + \mat{\varepsilon\\\vdots\\\varepsilon^m}
\end{displaymath}
satisfies the properties~\ref{item:11}-\ref{item:13}) from above. 


\subsection{Symbolic computation with $\varepsilon$} 
\label{sec:symb-comp-with}

The good news is that we do not have to calculate with a concrete
$\varepsilon>0$ at all. Let us understand which variable would leave the basis
if we would calculate with a very small $\varepsilon$. 

Recall that we consider the index set $K = \{ i \colon u(i)>0\}$, where
$u$ is the pivot column of the tableau 
\begin{displaymath}
  \label{c:eq:37}
  \begin{array}{c|c}
    c^T - c_B^TA_B^{-1}A & -c_B^Tx^*_B \\ \hline 
    A_B^{-1}A        & x^*_B.
  \end{array}
\end{displaymath}
The leaving variable is  $B_i$, where $i \in K$ minimizes the expression
$x^*(B_i) / u(i), \, i \in K$. We have 
\begin{displaymath}
  x^*(B_i) / u(i) = [(A_B^{-1} b)(i) + (A_B^{-1} \mat{\varepsilon\\\vdots\\\varepsilon^m}) (i)] / u(i). 
\end{displaymath}

\begin{definition}[Lexicographic order]
  Let $u,v \in \setR^n$ be vectors. We have $u<_{lex}v$, if $u=v$ or if
  the first nonzero component of $u-v$ is strictly negative. 
\end{definition}

Since $\varepsilon>0$ is arbitrarily  small we see that we need to find $i$
such that the vector $ [ (A_B^{-1} b)(i) | (A_B^{-1})_i] / u(i)$ is lexicographically
minimal. This is  the {\bf  lexicographic
pivoting rule}: 


\begin{enumerate}
\item If $\overline{c}\geq0$, then {\bf output optimal basis $B$}
  \label{c:item:3}
\item Otherwise, let $j$ be an index with $\overline{c}(j)<0$ and let
  $u = A_B^{-1}A^j$ be the $j$-th column of the system matrix of the
  tableau. If $u\leq0$, then {\bf output $-\infty$}, the linear program is
  unbounded. \label{c:item:7}
\item Otherwise compute the 
  vectors $[ (A_B^{-1} b) (i) | (A_B^{-1})_i] / u(i)$ 
  for all $i =
  1,\ldots,m$ with $u(i)>0$. Let $i^*$ be an index for which this vector
  is lexicographically smallest. 
  The index $B_{i^*}$ leaves the basis and $j$ enters the  basis, i.e., the new basis is $B' = B \setminus\{B_{i^*}\} \cup \{j\}$. 
\item Update the tableau to 
  \begin{displaymath}    
  \begin{array}{c|c}
    c^T - c_{B'}^TA_{B'}^{-1}A  & -c_{B'}^Tx^*_{B'} \\ \hline 
    A_{B'}^{-1}A        & x^*_{B'}.
  \end{array}
\end{displaymath}
\end{enumerate}


\begin{theorem}
  \label{thr:3}
  The simplex method terminates, if the leaving variable is chosen
  according to the lexicographic pivot rule above. 
\end{theorem}


\subsection{Bland's rule}

The following is an alternative pivoting rule which avoids cycling,
see, e.g.~\cite{Chvatal83}. 

\begin{quote}
  During an iteration of the simplex method
  (Chapter~\ref{sec:one-iter-simpl}), find a smallest $j$ such  
  that $\overline{c}(j)$ is less than $0$. Let this $j$ enter the
  basis. 
  
  Out of all the indices $i$ that may leave the basis, choose the
  smallest one. 
\end{quote}

%

% Let us apply this rule for our example. We start with the tableau 
% \begin{displaymath}
%   \begin{tabularx}{0.7\textwidth}{AAAARRR|X}   
%     \rowcolor{white}  
%     -3/4&   20& -1/2&    6&    0&    0&    0&    3\\ \hline
%     1/4&   -8&   -1&    9&    1&    0&    0&    0\\
%     1/2&  -12& -1/2&    3&    0&    1&    0&    0\\
%     0&    0&    1&    0&    0&    0&    1&    1\\
%   \end{tabularx}
%   %\quad \quad \quad   B = \{5,6,7\} 
% \end{displaymath}
% and let $1$ enter the basis. The ratios $x(i)/u(i)$ for $u(i)>0$ are
% minimized at $i=1,2$. 





\subsection*{Exercises}

\begin{enumerate}
\item Provide an example of a degenerate basic feasible solution $x^*$
  and an associated basis $B$ which is optimal but whose reduced cost
  vector $\overline{c}^T =c^T - c_B^T A_B^{-1}A$ does not satisfy
  $\overline{c}\geq0$.\label{item:10} 
\item Show that  $p_{iB}(\varepsilon)\neq 0$ holds for each $i\in \{1,\ldots,m\}$ and
  each basis $B$. \label{item:14}
\item Provide an example of a valid perturbation, as described above,
  such that a basis $B$ which is feasible for the original
  LP~\eqref{eq:4} is infeasible for the problem~\eqref{eq:5}. 
  \label{item:15} 
\end{enumerate}




\bibliographystyle{abbrv}

\bibliography{mybib,papers,books,my_publications}


%%% Local Variables: 
%%% mode: latex
%%% TeX-master: "lecture"
%%% End: 
