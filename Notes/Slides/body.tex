\begin{frame}{Informations sur} 

  \begin{alertblock}{}

    \begin{columns}
      \begin{column}{.5\textwidth}
        \begin{itemize}
        \item Contenue du cours
         \item Exercices et examen 
         \item Contact et forum
         \end{itemize}
       \end{column}
       \begin{column}{.5\textwidth}
         %  \begin{itemize}
         % \item Contenue du cours
         % \item Exercices et examen 
         % \item Contact et forum
         % \end{itemize}
       \end{column}
       
    \end{columns}
      
     \end{alertblock}
          \begin{center}
%       \input{./Images/projection-on-plane.tex}       
     \end{center}    
    
  
   \end{frame}
   



   \begin{frame}

     \begin{columns}
       \begin{column}{0.4\textwidth }                 
     \frametitle{Assistants}
     Assistant principal:
     \begin{itemize}
     \item Claudio Pfammatter 
     \end{itemize}

     \bigskip 
     
     Assistants doctorants:
     \begin{itemize}

     \item Eleonore Bach \item Mariana Martinez \item  Niklas Schmitz \item  Emre Özavci \item Michele Barruca 
     \end{itemize}
   \end{column}
   \begin{column}{0.6\textwidth }         
     Assistants étudiants      
     \begin{itemize}\item 
       Aleksandra Bogdanova 
     \item Benoit Cuenot 
     \item Emna Boughizane
     \item Leonard Lebrun
     \item  Patrick-Cristian Dan
     \item Rafael Sierra 
     \item Sandro Pfammatter
     \item Solveig Girardin
     \end{itemize}
   \end{column}
   
\end{columns}
\end{frame}

   \begin{frame}{Contenue}
   
     \begin{enumerate}
     \item Polynômes: {\small Racines, division avec reste,
         reconstruction, algorithme d'Euclide, Factorisation en
         irréductibles }
     \item Algorithmes et leur analyse:  {\small Déterminant, équations linéaires, polynôme caractéristique, diagonalisation }
     \item  Formes bilinéaires: {\small théorème de Sylvester, produits scalaires, norme, orthogonalisation, moindre carrées }
     \item Théorème spectral: {\small  valeurs singulières, pseudo-inverses}
     \item Systèmes différentiels linéaires: {\small exponentiel d'une matrice}
     \item La forme normale de Jordan
     \item  Algèbre linéaire sur les entiers{ \small structure de groupes abélien  engendrés finis }
   \end{enumerate}

   \end{frame}

   \begin{frame}{Notes du cours}

     \bigskip 
     
     Se trouvent sur moodle

     \bigskip 
     {Régulièrement}  mis à jours.

     \bigskip 
     {Fin du semestre}: Reflètent le contenu  relevant  
   
     
   \end{frame}
   


   \begin{frame}
     \frametitle{Exercices}

     \begin{itemize}
     \item Une série chaque semaine sur moodle.
     \item Première à télécharger  aujourd'hui 
     \item Exercice $+$ et $*$: \myemph{Pas} de corrigé. Seront \myemph{discutés} lors des sessions du mardi.
     \item Série de la semaine $n$
       \begin{itemize}
       \item Ex. $+$: Exercice préparatoire discuté Mardi semaine $n$
       \item Ex. $*$: Exercice type question ouverte examen   discuté Mardi semaine $n+1$
       \item Aujourd'hui: On discute exercice $+$ semaine $1$
       \item Autres exercices: Travaille pendant séance vendredi, semaine $n$.
       \item Ce vendredi: On discute exercices semaine $1$
       \end{itemize}
     \item Séance vendredi:
       \begin{itemize}
       \item Répartition des salles affichée sur moodle (selon nom)
       \item 14 assistants  à disposition pour questions, aide, discussions. 
       \end{itemize}
       
     \end{itemize}
     
   \end{frame}

   \begin{frame}
     \frametitle{Examen}

     \begin{itemize}
     \item 3-4 questions \myemph{ouvertes} (type ex. $*$). Une des trois est exercice $*$ original des séries
     \item $∼$ 7 questions  \myemph{choix multiples} 
     \item  $∼$ 14 questions \myemph{vrai/faux}     
     \end{itemize}

     \bigskip

     Examens des années précédentes  \myemph{sur moodle}
   \end{frame}

    \begin{frame}
     \frametitle{Take home exam} 
     \begin{itemize}
     \item 3 examens \myemph{take home} pendent le semestre
     \item Chaque représente $∼$ 1/3 d'un examen (une question ouverte, quelque CM + V/F)
     \item V/F et CM sont corrigés automatiquement
     \item Question ouverte à rendre pour feedback (en latex)  \myemph{fortement conseillé} 
     \item Dates:
       \begin{itemize}
       \item 10.03 - 16.03
       \item 07.04 - 13.04
       \item 12.05 - 18.05
       \end{itemize}
     \item Ne \myemph{comptent pas} vers la note finale 
     \end{itemize}
   \end{frame}


   
   \begin{frame}{Contact et communication}

     \begin{itemize}
     \item Questions, commentaires: \myemph{Utilisez forum!}
     \item Assistants sont présent sur forum 
     \item \myemph{Formatez les maths en latex!} 
     \item Ne m'envoyez pas de  e-mail!
     \item Essayez de résoudre les exercices soigneusement. Donnez vous assez de temps. 
     \item Office hours: Mardi 16:00 - 17:00.

       
      R.d.v. impératif: e-mail  {\tt pauline.bataillard@epfl.ch  }
     \end{itemize}
     
   \end{frame}


   

%%% Local Variables:
%%% mode: latex
%%% TeX-master: "Slides"
%%% End:
