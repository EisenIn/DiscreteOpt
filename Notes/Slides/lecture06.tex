\pagestyle{empty}



\begin{frame}{Review: Duality}

 
\end{frame}





\begin{frame}{Dual of the Dual}

 \begin{corollary}
   \label{thr:5}
   If the dual linear program has an optimal solution, then so does the
   primal linear program and the objective values coincide.
 \end{corollary}

  \begin{columns}
    \begin{column}{.5\textwidth}
      
    \end{column}
    \begin{column}{.5\textwidth}
      
    \end{column}       
  \end{columns}
\end{frame}





\begin{frame}{Table of possibilities}

  \begin{columns}
    \begin{column}{.5\textwidth}
      
    \end{column}
    \begin{column}{.5\textwidth}
      
    \end{column}       
  \end{columns}
\end{frame}





\begin{frame}{Further example}



  
  \begin{columns}
    \begin{column}{.5\textwidth}
      \begin{equation*}
        \begin{array}{rcl}
          \max &c^Tx \\
          Bx &  =&  b\\
          Cx & \leq & d. 
        \end{array}
      \end{equation*}
      (Primal)
    \end{column}
    \begin{column}{.5\textwidth}
      \begin{equation*}
        \begin{array}{c}
          \min \, b^Ty_1 + d^T y_2 \\
          \begin{array}{rcl}
            B^Ty_1 + C^T  y_2 & = &  c \\  y_2& \geq & 0. 
          \end{array}
        \end{array}
      \end{equation*}
      (Dual)
    \end{column}       
  \end{columns}
\end{frame}





\begin{frame}{}

  \begin{columns}
    \begin{column}{.5\textwidth}
      
    \end{column}
    \begin{column}{.5\textwidth}
      
    \end{column}       
  \end{columns}
\end{frame}



\begin{frame}{Zero sum games}


\begin{equation*}
  \label{eq:34}
  A =
  \begin{pmatrix}
    5 & 1 & 3 \\
    3 & 2 & 4 \\
    -3 & 0 &1 
  \end{pmatrix} 
\end{equation*}

\medskip 

Row player: Chooses row $i$

\smallskip 
Column player: Chooses column $j$


  \begin{columns}
    \begin{column}{.5\textwidth}
      
    \end{column}
    \begin{column}{.5\textwidth}
      
    \end{column}       
  \end{columns}
\end{frame}



\begin{frame}{Rock --  paper -- scissors}

  
\end{frame}


\begin{frame}{Deterministic strategies}


  \begin{columns}
    \begin{column}{.5\textwidth}
      \begin{displaymath}
        \max_i \min_j
      \end{displaymath}
    \end{column}
    \begin{column}{.5\textwidth}
       \begin{displaymath}
        \max_j \min_i
      \end{displaymath}
    \end{column}       
  \end{columns}
  
\end{frame}


\begin{frame}{Mixed strategies}

\begin{definition}
  \label{def:m1}
  Let $A \in \setR^{m\times n}$ define a two-player matrix game. A mixed
  strategy for the row-player is a vector $x \in \setR_{\geq0}^m$ with
  $\sum_{i=1}^m x_i = 1$. A mixed strategy for the column player is a
  vector $y \in  \setR_{\geq0}^n$ with $\sum_{j=1}^n y_i = 1$. 
\end{definition}

\begin{equation}
  \label{eq:37}
  E[\text{Payoff}]=x^TAy. 
\end{equation}

  \begin{columns}
    \begin{column}{.5\textwidth}
      
    \end{column}
    \begin{column}{.5\textwidth}
      
    \end{column}       
  \end{columns}
\end{frame}



\begin{frame}{Example: Mixed strategy rock -- paper -- scissors}
  
\end{frame}

\begin{frame}{Weak duality}
\begin{lemma}
  \label{d:lem:10}
  Let $A \in \setR^{m\times n}$, then 
   \begin{displaymath}
   \max_{x \in X} \min_{y\in Y} x^TA y \leq \min_{y\in Y} \max_{x\in X}
   x^TAy,       
  \end{displaymath}
  where $X$ and $Y$ denote the set of mixed row and column-strategies
  respectively. 
\end{lemma}
  \begin{columns}
    \begin{column}{.5\textwidth}
      
    \end{column}
    \begin{column}{.5\textwidth}
      
    \end{column}       
  \end{columns}
\end{frame}



\begin{frame}{Minimax-Theorem}



  \begin{theorem}[von Neumann (1928)]
  \label{d:thr:7}
  \begin{displaymath}
   \max_{x \in X} \min_{y\in Y} x^TA y = \min_{y\in Y} \max_{x\in X}
   x^TAy,       
  \end{displaymath}
  where $X$ and $Y$ denote the set of mixed row and column-strategies
  respectively. 
\end{theorem}
   \begin{columns}
    \begin{column}{.5\textwidth}
      
    \end{column}
    \begin{column}{.5\textwidth}
      
    \end{column}       
  \end{columns}
\end{frame}


\begin{frame}{}
  
\end{frame}



\begin{frame}{Duality  via Farkas' lemma}

  
\begin{theorem}[Second variant of Farkas' lemma]
\label{dual:thr:2farkas}
    Let $A \in \setR^{m\times n}$ and $b \in \setR^m$. The system $Ax\leq b$ has a
    solution if and only if for all $\lambda\geq0$ with $\lambda^TA =0$ one has
    $\lambda^Tb\geq0$. 
\end{theorem}

  
\end{frame}
\begin{frame}{Duality  via Farkas' lemma}


\end{frame}


\begin{frame}
  \frametitle{Algorithms and running time analysis}

  Consider the following algorithm to compute the product of two $n × n$ matrices $A,B ∈ ℚ^{n ×n}$: 
    \begin{tabbing}            
    {\bf for} \= $i=1,\dots,n$ \\ 
              \> {\bf for} \= $j=1,\dots,n$ \\
              \> \> $c_{ij} := 0$ \\
              \> \> {\bf for} \= $k=1,\dots,n$ \\
              \> \> \> $c_{ij} := c_{ij} + a_{ik} ⋅ a_{kj}$
            \end{tabbing} 
%
\end{frame}

\begin{frame}
  \frametitle{\boldmath $O$-notation}
  
\begin{definition}

 Let $T,f: \setN \to \setR_{\geq0}$ be functions. We say 
  \begin{itemize}
  \item \emph{$T(n) = O(f(n))$}, if there exist positive  constants
    $n_o\in \setN$ and  $c\in\setR_{>0}$ 
    with $$T(n) \leq c \cdot f(n) \text{ for all }n\geq n_0.$$  
  \item \emph{$T(n) = \Omega(f(n))$}, if there exist constants   $n_o\in \setN$
    and  $c\in\setR_{>0}$ 
    with $$T(n) \geq c \cdot f(n)\text{ for all } n\geq n_0.$$
  \item \emph{$T(n) = \Theta(f(n))$}  if $$T(n)=O(f(n))\text{ and }
     T(n) = \Omega(f(n)).$$
  \end{itemize}
\end{definition}


\end{frame}


\begin{frame}{Example}

  

\begin{example}
  The function $T(n)=2n^2 + 3n +1$ is in $O(n^2)$, since for all
  $n\geq1$ one has $2n^2 + 3n + 1 \leq 6n^2$. Here $n_0 = 1$ and $c =
  6$.  Similarly $T(n) = \Omega(n^2)$, since for each $n\geq1$ one has $2n^2 + 3n
  +1\geq n^2$. Thus $T(n)$ is in $\Theta(n^2)$. 
\end{example}
  
\end{frame}



\begin{frame}{Efficient algorithm, first definition}

  An algorithm runs in
\emph{polynomial time}, if there exists a constant $k$ such that the
algorithm runs in time $O(n^k)$, where $n$ is the length of the input 
of the algorithm.
  
\end{frame}
%%% Local Variables:
%%% mode: LaTeX
%%% TeX-master: "Slides"
%%% End:


