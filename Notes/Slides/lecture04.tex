\pagestyle{empty}



\begin{frame}{The simplex algorithm}

  \begin{columns}
    \begin{column}{.5\textwidth}
      \begin{displaymath}
        \begin{array}{c}
          \max c^T x\\
          A x ≤ b
        \end{array}
      \end{displaymath}


      $A ∈ℝ^{m × n}$

      \medskip

      $\rank(A) = n$ 
    \end{column}
    \begin{column}{.5\textwidth}
      
    \end{column}       
  \end{columns}
\end{frame}



\begin{frame}{Notation}


  \begin{columns}
    \begin{column}{.5\textwidth}
      Let $B \subseteq \{1,\dots,m\}$

      \bigskip 
      $A_B \in \R^{|B| \times n}$ rows  indexed by   $B$
      

      \bigskip 
      $b_B \in \R^{|B|}$  components  indexed by $B$ 
    \end{column}
    \begin{column}{.5\textwidth}
Example: For       $    A = 
    \begin{pmatrix}
      3 & 2  \\
      7 & 1\\
      8 & 4\\
    \end{pmatrix}$,  $b = 
    \begin{pmatrix}
      3 \\ 2\\ 6
    \end{pmatrix}$

    \medskip 
    and  $B = \{2,3\}$, one has

    \bigskip 
    \begin{displaymath}
    A_B =  \begin{pmatrix}
          7 & 1\\
      8 & 4\\
    \end{pmatrix} \text{ and }  b_B = 
    \begin{pmatrix}
      2\\ 6
    \end{pmatrix}.          
    \end{displaymath}

    \end{column}       
  \end{columns}
\end{frame}

 



\begin{frame}{Feasible basis}
  \begin{definition}
     An index set $B \subseteq \{1,\dots,m\}$ is a \emph{basis} if
      $|B| = n$ and $A_B$ is non-singular. If in addition $x^* =
      A_B^{-1} b_B$ is feasible, then $B$ is called a \emph{feasible
        basis}.
  \end{definition}
  \begin{columns}
    \begin{column}{.5\textwidth}
      
    \end{column}
    \begin{column}{.5\textwidth}
      
    \end{column}       
  \end{columns}
\end{frame}




\begin{frame}{Feasible basis vs extreme point}

  \begin{columns}
    \begin{column}{.5\textwidth}
      
    \end{column}
    \begin{column}{.5\textwidth}
      
    \end{column}       
  \end{columns}
\end{frame}




\begin{frame}{Optimal basis}
\begin{definition}
      \label{def:s-2}
       A basis $B$ is called \emph{optimal} if it is feasible and the
       unique $\lambda \in \setR^m$ 
       with 
       \begin{equation}
         \label{eq:s-5}
         \lambda^T A =  c^T \, \, \text{ and } \lambda_i = 0, \, i \notin B 
       \end{equation}
       satisfies $\lambda\geq0$. 
     \end{definition}

  \begin{columns}
    \begin{column}{.5\textwidth}
      
    \end{column}
    \begin{column}{.5\textwidth}
      
    \end{column}       
  \end{columns}
\end{frame}




\begin{frame}{Optimal basis vs. optimal solution}

\begin{theorem}
  \label{thr:s-4}
  If $B$ is an optimal basis, then $x^* = A_B^{-1} b_B$ is an optimal solution of the linear program. 
\end{theorem}

  \begin{columns}
    \begin{column}{.5\textwidth}
      
    \end{column}
    \begin{column}{.5\textwidth}
    
    \end{column}       
  \end{columns}
\end{frame}




\begin{frame}{ }

  \begin{columns}
    \begin{column}{.5\textwidth}
      
    \end{column}
    \begin{column}{.5\textwidth}
      
    \end{column}       
  \end{columns}
\end{frame}



\begin{frame}{Moving to an improving vertex}
  


  \begin{columns}
    \begin{column}{.5\textwidth}
      $d ∈ ℝ^n$ unique solution to 
  \begin{displaymath}
  a_j^T d =
  \begin{cases}
    0 & \text{ for } j \in B \setminus \{i\} \\
    -1 & \text{ if } j =i.
  \end{cases}
\end{displaymath}
    \end{column}
    \begin{column}{.5\textwidth}
        $c^T d = $ 
    \end{column}       
  \end{columns}
\end{frame}


\begin{frame}{How far can we move?}

  \begin{columns}
    \begin{column}{.5\textwidth}
      
    \end{column}
    \begin{column}{.5\textwidth}
      
    \end{column}       
  \end{columns}
\end{frame}




\begin{frame}{The simplex algorithm}
 \begin{tabbing}
    Start with feasible basis $B$ \\[1ex]
    {\tt while} \= $B$ is not optimal \\ [.7ex]
    \> Let $i \in B$ be index with $\lambda_i<0$ \\
    \> Compute  $d \in \setR^n$ with $a_j^T d = 0, \, j \in B \setminus\{i\}$
    and $a_i^T d = -1$ \\ 
    \> Determine $K = \{ k \colon 1 \leq k \leq m, \, a_k^Td >0\}$\\[.7ex]  
    \> {\tt if} \= $K = \emptyset$ \\   
    \> \> \emph{assert LP unbounded} \\
    \> {\tt else} \\
    \> \> Let $k \in K$ index where 
    $
    \displaystyle \min_{k \in K} (b_k - a_k^Tx^*)/a_k^Td
    $
    is attained \\ %(with  $x^* = A_B^{-1}b_B$)
    
    \> \>\emph{update} $B := B \setminus\{i\} \cup \{k\}$             
  \end{tabbing}
  \begin{columns}
    \begin{column}{.5\textwidth}
      
    \end{column}
    \begin{column}{.5\textwidth}
      
    \end{column}       
  \end{columns}
\end{frame}


\begin{frame}{Example} \scriptsize 

  \begin{columns}[t]
    \begin{column}{.5\textwidth}

 
  \begin{displaymath}
    A  =\begin{pmatrix}1 & 2 & 2\\2 & 1 & 2\\2 & 2 & 1\\-1 & 0 & 0\\0 & -1 & 0\\0 & 0 & -1\end{pmatrix}, \, b = 
    \begin{pmatrix}10\\14\\11\\0\\0\\0\end{pmatrix} \text{ and } c =
    \begin{pmatrix}
      6\\14\\13
    \end{pmatrix}
  \end{displaymath}
   starting basis
  \begin{displaymath}
    B = \{1,2,3\}. 
  \end{displaymath}
       \begin{displaymath}
    A_B = \begin{pmatrix}1 & 2 & 2\\2 & 1 & 2\\2 & 2 & 1\end{pmatrix} b_B = \begin{pmatrix}10\\14\\11\end{pmatrix} \text{ and } x^*_B = \begin{pmatrix}4\\0\\3\end{pmatrix}. 
  \end{displaymath} 
    \end{column}
    \begin{column}{.5\textwidth}

  \begin{displaymath}
    λ_B = \begin{pmatrix}\frac{36}{5}\\- \frac{4}{5}\\\frac{1}{5}\end{pmatrix}, \,  
     d = \begin{pmatrix}- \frac{2}{5}\\\frac{3}{5}\\- \frac{2}{5}\end{pmatrix}. 
   \end{displaymath}      


\begin{displaymath}
  A ⋅ d = \begin{pmatrix}0\\-1\\0\\\frac{2}{5}\\- \frac{3}{5}\\\frac{2}{5}\end{pmatrix}, \, 
  b-Ax^* = 
\begin{pmatrix}0\\0\\0\\4\\0\\3\end{pmatrix}.
\end{displaymath}

    \end{column}       
  \end{columns}
\end{frame}


\begin{frame}{Non-degenerate LPs}

  \begin{definition}
    The LP $\max\{c^Tx : x ∈ ℝ^n, \, Ax≤b\}$ is \emph{non-degenerate} if number of zero-components in $Ax-b ∈ ℝ^m$ is at most  $n$ for each $x ∈ ℝ^n$. 
  \end{definition}
  
  \begin{columns}
    \begin{column}{.5\textwidth}
      
    \end{column}
    \begin{column}{.5\textwidth}
      
    \end{column}       
  \end{columns}
\end{frame}




\begin{frame}{Termination non-degenerate case}



\begin{theorem}
  \label{thr:s-5}
  If the linear program  is non-degenerate, then the
  simplex algorithm terminates.
\end{theorem}
  
  \begin{columns}
    \begin{column}{.5\textwidth}
      
    \end{column}
    \begin{column}{.5\textwidth}
      
    \end{column}       
  \end{columns}
\end{frame}



\begin{frame}{}

  \begin{columns}
    \begin{column}{.5\textwidth}
      
    \end{column}
    \begin{column}{.5\textwidth}
      
    \end{column}       
  \end{columns}
\end{frame}


\begin{frame}{Smallest index rule}

 \begin{tabbing}
      Start with feasible basis $B$ \\[1ex]
      {\tt while} \= $B$ is not optimal \\ [.7ex]
      \> Compute $\lambda \in \setR^n$ such that $\lambda^TA_B = c^T$ \\
      \> Let $i^* \in B$ be the \emph{ smallest index} with $\lambda_i<0$ \\
      \> Compute  $d \in \setR^n$ with $a_j^T d = 0, \, j \in B \setminus\{i^*\}$
      and $a_{i^*}^T d = -1$ \\ 
      \> Determine $K = \{ k \colon 1 \leq k \leq m, \, a_k^Td >0\}$\\[.7ex]  
      \> {\tt if} \= $K = \emptyset$ \\   
      \> \> \emph{assert LP unbounded} \\
      \> {\tt else} \\
      \> \> Let $k^* \in K$ the \emph{smallest index} where 
     $
        \displaystyle \min_{k \in K} (b_k - a_k^Tx^*)/a_k^Td
      $
      is attained \\ %(with  $x^* = A_B^{-1}b_B$)

      \> \>\emph{update} $B := B \setminus\{i^*\} \cup \{k^*\}$             
    \end{tabbing}
    

  
  \begin{columns}
    \begin{column}{.5\textwidth}
      
    \end{column}
    \begin{column}{.5\textwidth}
      
    \end{column}       
  \end{columns}
\end{frame}




\begin{frame}{Termination}



  \begin{theorem}
    \label{thr:3}
    The simplex algorithm with the smallest index rule terminates. 
  \end{theorem}

  
  \begin{columns}
    \begin{column}{.5\textwidth}
      
    \end{column}
    \begin{column}{.5\textwidth}
      
    \end{column}       
  \end{columns}
\end{frame}



\begin{frame}{}

  \begin{columns}
    \begin{column}{.5\textwidth}
      
    \end{column}
    \begin{column}{.5\textwidth}
      
    \end{column}       
  \end{columns}
\end{frame}


\begin{frame}{}

  \begin{columns}
    \begin{column}{.5\textwidth}
      
    \end{column}
    \begin{column}{.5\textwidth}
      
    \end{column}       
  \end{columns}
\end{frame}




\begin{frame}{}

  \begin{columns}
    \begin{column}{.5\textwidth}
      
    \end{column}
    \begin{column}{.5\textwidth}
      
    \end{column}       
  \end{columns}
\end{frame}








%%% Local Variables:
%%% mode: LaTeX
%%% TeX-master: "Slides"
%%% End:
