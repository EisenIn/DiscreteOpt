\pagestyle{empty}



\begin{frame}{Review: The simplex algorithm}

 
\end{frame}


\begin{frame}{Review: The simplex algorithm}

 
\end{frame}





\begin{frame}{Finding an initial feasible basis}

  \begin{columns}
    \begin{column}{.5\textwidth}
      
Linear program has equivalent form 
\begin{equation}
  \label{eq:s-3}
  \max\{ c^Tx \colon Ax \leq b, \, x \geq 0\}. 
\end{equation}

\bigskip

Split  $Ax \leq b$ as

\medskip 
$A_1x\leq b_1$ and $A_2x\leq b_2$

\medskip 
with $b_1 \geq 0$ and $b_2<0$.
    \end{column}
    \begin{column}{.5\textwidth}
      
    \end{column}       
  \end{columns}
\end{frame}





\begin{frame}{Duality}

  \begin{columns}
    \begin{column}{.5\textwidth}
      \begin{equation*}
        \max\{c^T x \colon x \in \setR^n, \, Ax \leq b\}, 
      \end{equation*}
      (Primal) 
    \end{column}
    \begin{column}{.5\textwidth}
      \begin{equation*}        
  \min\{b^Ty \colon y \in \setR^m, \, A^T y = c,\, y\geq0\}.
\end{equation*}
(Dual)
    \end{column}       
  \end{columns}
\end{frame}





\begin{frame}{Duality}

  \begin{columns}
    \begin{column}{.5\textwidth}
      
    \end{column}
    \begin{column}{.5\textwidth}
      
    \end{column}       
  \end{columns}
\end{frame}





\begin{frame}{Weak Duality}
\begin{theorem}[Weak duality]
  If $x^*$ and $y^*$ are primal and dual feasible solutions respectively, then
  $c^Tx^* \leq b^Ty^*$. 
\end{theorem}

  \begin{columns}
    \begin{column}{.5\textwidth}
      
    \end{column}
    \begin{column}{.5\textwidth}
      
    \end{column}       
  \end{columns}
\end{frame}





\begin{frame}{}

  \begin{columns}
    \begin{column}{.5\textwidth}
      
    \end{column}
    \begin{column}{.5\textwidth}
      
    \end{column}       
  \end{columns}
\end{frame}





\begin{frame}{Strong duality}

\begin{theorem}
  \label{thr:4}
  If the primal linear program is feasible and bounded, then so is 
  the dual linear program. Furthermore in this case, both linear
  programs have an optimal solution  and the optimal  values coincide. 
\end{theorem}

  \begin{columns}
    \begin{column}{.5\textwidth}
      
    \end{column}
    \begin{column}{.5\textwidth}
      
    \end{column}       
  \end{columns}
\end{frame}





\begin{frame}{}

  \begin{columns}
    \begin{column}{.5\textwidth}
      
    \end{column}
    \begin{column}{.5\textwidth}
      
    \end{column}       
  \end{columns}
\end{frame}





\begin{frame}{Dual of the Dual}

  \begin{columns}
    \begin{column}{.5\textwidth}
      \begin{displaymath}
   \min\{ b^Ty \colon y \in \setR^m, \, A^T y = c, \, y\geq0\}
 \end{displaymath}
    \end{column}
    \begin{column}{.5\textwidth}
      
    \end{column}       
  \end{columns}
\end{frame}





\begin{frame}{}

  \begin{columns}
    \begin{column}{.5\textwidth}
      
    \end{column}
    \begin{column}{.5\textwidth}
      
    \end{column}       
  \end{columns}
\end{frame}


\begin{frame}{10 Minutes break for Retour Indicatif}

  \begin{columns}
    \begin{column}{.5\textwidth}
      \begin{displaymath}
  % \min\{ b^Ty \colon y \in \setR^m, \, A^T y = c, \, y\geq0\}
 \end{displaymath}
    \end{column}
    \begin{column}{.5\textwidth}
      
    \end{column}       
  \end{columns}
\end{frame}






\begin{frame}{Dual of the Dual}

 \begin{corollary}
   \label{thr:5}
   If the dual linear program has an optimal solution, then so does the
   primal linear program and the objective values coincide.
 \end{corollary}

  \begin{columns}
    \begin{column}{.5\textwidth}
      
    \end{column}
    \begin{column}{.5\textwidth}
      
    \end{column}       
  \end{columns}
\end{frame}





\begin{frame}{Table of possibilities}

  \begin{columns}
    \begin{column}{.5\textwidth}
      
    \end{column}
    \begin{column}{.5\textwidth}
      
    \end{column}       
  \end{columns}
\end{frame}





\begin{frame}{Further example}



  
  \begin{columns}
    \begin{column}{.5\textwidth}
      \begin{equation*}
        \begin{array}{rcl}
          \max &c^Tx \\
          Bx &  =&  b\\
          Cx & \leq & d. 
        \end{array}
      \end{equation*}
      (Primal)
    \end{column}
    \begin{column}{.5\textwidth}
      \begin{equation*}
        \begin{array}{c}
          \min \, b^Ty_1 + d^T y_2 \\
          \begin{array}{rcl}
            B^Ty_1 + C^T  y_2 & = &  c \\  y_2& \geq & 0. 
          \end{array}
        \end{array}
      \end{equation*}
      (Dual)
    \end{column}       
  \end{columns}
\end{frame}





\begin{frame}{}

  \begin{columns}
    \begin{column}{.5\textwidth}
      
    \end{column}
    \begin{column}{.5\textwidth}
      
    \end{column}       
  \end{columns}
\end{frame}



\begin{frame}{Zero sum games}


\begin{equation*}
  \label{eq:34}
  A =
  \begin{pmatrix}
    5 & 1 & 3 \\
    3 & 2 & 4 \\
    -3 & 0 &1 
  \end{pmatrix} 
\end{equation*}

\medskip 

Row player: Chooses row $i$

\smallskip 
Column player: Chooses column $j$


  \begin{columns}
    \begin{column}{.5\textwidth}
      
    \end{column}
    \begin{column}{.5\textwidth}
      
    \end{column}       
  \end{columns}
\end{frame}



\begin{frame}{Rock --  paper -- scissors}

  
\end{frame}


\begin{frame}{Deterministic strategies}


  \begin{columns}
    \begin{column}{.5\textwidth}
      \begin{displaymath}
        \max_i \min_j
      \end{displaymath}
    \end{column}
    \begin{column}{.5\textwidth}
       \begin{displaymath}
        \max_j \min_i
      \end{displaymath}
    \end{column}       
  \end{columns}
  
\end{frame}


\begin{frame}{Mixed strategies}

\begin{definition}
  \label{def:m1}
  Let $A \in \setR^{m\times n}$ define a two-player matrix game. A mixed
  strategy for the row-player is a vector $x \in \setR_{\geq0}^m$ with
  $\sum_{i=1}^m x_i = 1$. A mixed strategy for the column player is a
  vector $y \in  \setR_{\geq0}^n$ with $\sum_{j=1}^n y_i = 1$. 
\end{definition}

\begin{equation}
  \label{eq:37}
  E[\text{Payoff}]=x^TAy. 
\end{equation}

  \begin{columns}
    \begin{column}{.5\textwidth}
      
    \end{column}
    \begin{column}{.5\textwidth}
      
    \end{column}       
  \end{columns}
\end{frame}



\begin{frame}{Example: Mixed strategy rock -- paper -- scissors}
  
\end{frame}

\begin{frame}{Weak duality}
\begin{lemma}
  \label{d:lem:10}
  Let $A \in \setR^{m\times n}$, then 
   \begin{displaymath}
   \max_{x \in X} \min_{y\in Y} x^TA y \leq \min_{y\in Y} \max_{x\in X}
   x^TAy,       
  \end{displaymath}
  where $X$ and $Y$ denote the set of mixed row and column-strategies
  respectively. 
\end{lemma}
  \begin{columns}
    \begin{column}{.5\textwidth}
      
    \end{column}
    \begin{column}{.5\textwidth}
      
    \end{column}       
  \end{columns}
\end{frame}



\begin{frame}{Minimax-Theorem}



  \begin{theorem}[von Neumann (1928)]
  \label{d:thr:7}
  \begin{displaymath}
   \max_{x \in X} \min_{y\in Y} x^TA y = \min_{y\in Y} \max_{x\in X}
   x^TAy,       
  \end{displaymath}
  where $X$ and $Y$ denote the set of mixed row and column-strategies
  respectively. 
\end{theorem}
   \begin{columns}
    \begin{column}{.5\textwidth}
      
    \end{column}
    \begin{column}{.5\textwidth}
      
    \end{column}       
  \end{columns}
\end{frame}


\begin{frame}{}
  
\end{frame}



%%% Local Variables:
%%% mode: LaTeX
%%% TeX-master: "Slides"
%%% End:
