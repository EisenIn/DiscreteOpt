\pagestyle{empty}



\begin{frame}{Review: The simplex algorithm}

 
\end{frame}


\begin{frame}{Review: The simplex algorithm}

 
\end{frame}





\begin{frame}{Finding an initial feasible basis}

  \begin{columns}
    \begin{column}{.5\textwidth}
      
Linear program has equivalent form 
\begin{equation}
  \label{eq:s-3}
  \max\{ c^Tx \colon Ax \leq b, \, x \geq 0\}. 
\end{equation}

\bigskip

Split  $Ax \leq b$ as

\medskip 
$A_1x\leq b_1$ and $A_2x\leq b_2$

\medskip 
with $b_1 \geq 0$ and $b_2<0$.
    \end{column}
    \begin{column}{.5\textwidth}
      
    \end{column}       
  \end{columns}
\end{frame}





\begin{frame}{Duality}

  \begin{columns}
    \begin{column}{.5\textwidth}
      \begin{equation*}
        \max\{c^T x \colon x \in \setR^n, \, Ax \leq b\}, 
      \end{equation*}
      (Primal) 
    \end{column}
    \begin{column}{.5\textwidth}
      \begin{equation*}        
  \min\{b^Ty \colon y \in \setR^m, \, A^T y = c,\, y\geq0\}.
\end{equation*}
(Dual)
    \end{column}       
  \end{columns}
\end{frame}





\begin{frame}{Duality}

  \begin{columns}
    \begin{column}{.5\textwidth}
      
    \end{column}
    \begin{column}{.5\textwidth}
      
    \end{column}       
  \end{columns}
\end{frame}





\begin{frame}{Weak Duality}
\begin{theorem}[Weak duality]
  If $x^*$ and $y^*$ are primal and dual feasible solutions respectively, then
  $c^Tx^* \leq b^Ty^*$. 
\end{theorem}

  \begin{columns}
    \begin{column}{.5\textwidth}
      
    \end{column}
    \begin{column}{.5\textwidth}
      
    \end{column}       
  \end{columns}
\end{frame}





\begin{frame}{}

  \begin{columns}
    \begin{column}{.5\textwidth}
      
    \end{column}
    \begin{column}{.5\textwidth}
      
    \end{column}       
  \end{columns}
\end{frame}





\begin{frame}{Strong duality}

\begin{theorem}
  \label{thr:4}
  If the primal linear program is feasible and bounded, then so is 
  the dual linear program. Furthermore in this case, both linear
  programs have an optimal solution  and the optimal  values coincide. 
\end{theorem}

  \begin{columns}
    \begin{column}{.5\textwidth}
      
    \end{column}
    \begin{column}{.5\textwidth}
      
    \end{column}       
  \end{columns}
\end{frame}





\begin{frame}{}

  \begin{columns}
    \begin{column}{.5\textwidth}
      
    \end{column}
    \begin{column}{.5\textwidth}
      
    \end{column}       
  \end{columns}
\end{frame}





\begin{frame}{Dual of the Dual}

  \begin{columns}
    \begin{column}{.5\textwidth}
      \begin{displaymath}
   \min\{ b^Ty \colon y \in \setR^m, \, A^T y = c, \, y\geq0\}
 \end{displaymath}
    \end{column}
    \begin{column}{.5\textwidth}
      
    \end{column}       
  \end{columns}
\end{frame}





\begin{frame}{}

  \begin{columns}
    \begin{column}{.5\textwidth}
      
    \end{column}
    \begin{column}{.5\textwidth}
      
    \end{column}       
  \end{columns}
\end{frame}


\begin{frame}{10 Minutes break for Retour Indicatif}

  \begin{columns}
    \begin{column}{.5\textwidth}
      \begin{displaymath}
  % \min\{ b^Ty \colon y \in \setR^m, \, A^T y = c, \, y\geq0\}
 \end{displaymath}
    \end{column}
    \begin{column}{.5\textwidth}
      
    \end{column}       
  \end{columns}
\end{frame}

%%% Local Variables:
%%% mode: LaTeX
%%% TeX-master: "Slides"
%%% End:
