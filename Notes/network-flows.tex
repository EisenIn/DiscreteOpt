\chapter{Network flows}
\label{sec:flows}


\section{Maximum $s-t$-flows}
\label{sec:maximum-s-t}

We now turn our attention to a linear programming problem which we
will solve by direct methods, motivated by the nature of the
problem. We often use the following notation. If $f:A\longrightarrow B$ denotes a
function and if $U\subseteq A$, then $f(U)$ is defined as $f(U)= \sum_{a\in U} f(a)$. 




\begin{definition}[Network, $s-t$-flow]
  A network with capacities  consists of a directed simple graph
  $D=(V,A)$ and a   \emph{capacity function} $u:A\to\setR_{\geq0}$.  A function
  $f:A\to\setR_{\geq0}$ is   called an \emph{$s-t$-flow}, if 
  \begin{equation}
    \sum_{e \in \delta^{out}(v)} f(e) = \sum_{e \in \delta^{in}(v)} f(e), \, \mbox{ for all } v \in V-\{s,t\},
  \end{equation}
  where $s,t\in V$. The flow is \emph{feasible}, if $f(e) \leq u(e)$ for all
  $e \in A$. 
  The \emph{value} of $f$ is defined as 
  $value(f) =  \sum_{e \in \delta^{out}(s)} f(e) - \sum_{e \in \delta^{in}(s)} f(e)$.
  The \emph{maximum $s-t$-flow problem} is the problem of determining
  a maximum feasible $s-t$-flow. 
\end{definition}


Here, for $U \subseteq V$, $\delta^{in}(U)$ denotes the arcs which are entering $U$
and $\delta^{out}(U)$ denotes the arcs which are leaving $U$.   Edge sets
of the form $\delta^{out}(U)$   are called a \emph{cut} of $D$. The
\emph{capacity of a cut } $c(\delta^{out}(U))$ is the sum of the capacities of
its edges. 


Thus the maximum  flows problem is a linear program of the form 

\begin{eqnarray}
  \max \sum_{e \in \delta^{out}(s)} x(e) & - &   \sum_{e \in \delta^{in}(s)} x(e) \\
   \sum_{e \in \delta^{out}(v)} x(e) & = &  \sum_{e \in \delta^{in}(v)} x(e), \, \mbox{ for all
   } v \in V-\{s,t\} \\
   x(e) & \leq & u (e) , \, \mbox{ for all
   } e \in A \\
   x(e) & \geq & 0 , \, \mbox{ for all
   } e \in A \\
\end{eqnarray}


\begin{definition}[excess function]
  For any $f:A \to \setR$, the excess function is the function 
  $excess_f:  2^V \to \setR$  defined by $excess_f(U) = \sum_{e \in \delta^{in}(U)}
  f(e) - \sum_{e \in \delta^{out}(U)} f(e)$. 
\end{definition}


\begin{theorem}
\label{thr:6}
Let $D = (V,A)$ be a digraph, let $f:A\to\setR$ and let $U\subseteq V$, then
\begin{equation}
  \label{eq:28}
  excess_f(U) = \sum_{v \in U} excess_f(v).
\end{equation}  
\end{theorem}


\begin{proof}
  An arc which has both endpoints in $U$ is counted twice with
  different parities on the right, and thus cancels out. An arc which
  has his tail in $U$ is subtracted once on the right and once on the
  left.  An arc which has his head in $U$ is added once on the right
  and once on the left.
\end{proof}

A cut $\delta^{out}(U)$ with $s \in U$  and $t \notin U$ is called an $s-t$-cut.  

\begin{theorem}[Weak duality]
  \label{thr:7}
  Let $f$ be a feasible $s-t$-flow and let $\delta^{out}(U)$ be an
  $s-t$-cut, then $value(f) \leq u(\delta^{out}(U))$. 
\end{theorem}

\begin{proof}
  $value(f) = -excess_f(s) = -excess_f(U) = f(\delta^{out}(U)) -
  f(\delta^{in}(U)) \leq f(\delta^{out}(U)) \leq u(\delta^{out}(U))$.
\end{proof}

For an arc $a = (u,v) \in A$ the arc $a^{-1}$ denotes the arc $(v,u)$. 

\begin{definition}[Residual graph]
  Let $f:A \to \setR$, $l:A \to \setR$ and $u: A \to \setR$ where $l \leq f \leq u$. Consider
  the sets of edges
  \begin{equation}
    \label{eq:30}
    A_f = \{ a \mid a \in A, \, f(a) < u(a)\} \cup \{ a^{-1} \mid a \in A, \, f(a) >    l(a)\}. 
  \end{equation}
  The digraph $D(f) = (V,A_f)$ is called the \emph{residual graph} of
  $f$ (for lower bound $l$ and capacities $u$).   
\end{definition}


\begin{definition}[walk, path, distance]
  \label{def:6}
  A \emph{walk} is a sequence of the form
  $P=(v_0,a_1,v_1,\ldots,v_{m-1},a_m,v_m)$, where  $a_i =
  (v_{i-1},v_i)\in A$ for $i=1,\ldots,m$. If the nodes $v_0,\ldots,v_m$ are all
  different, then $P$ is a \emph{path}. The \emph{length } of $P$ is
  $m$. The \emph{distance} of two nodes  $u$ and $v$ is the length of
  a shortest path from $u$ to $v$. 
\end{definition}


\begin{corollary}
  \label{co:6}
  Let $f$ be a feasible $s-t$-flow and suppose that $D(f)$ has no path
  from $s$ to $t$, then $f$ has maximum value.
\end{corollary}


\begin{proof}
  Let $U$ be the set of nodes which are reachable in $D(f)$ from
  $s$. Clearly $\delta(U)$ is an $s-t$-cut. Now $value(f) = f(\delta^{out}(U)) -
  f(\delta^{in}(U)$.  Each arc leaving $U$ is not an arc of $D(f)$ and thus
  $f(\delta^{out}(U)) = u (\delta^{out}(U))$. Each arc entering $U$ is not an
  arc of $D(f)$ and thus $f(\delta^{in}(U) = 0$. It follows that  
  $value(f) = u (\delta^{out}(U))$ and $f$ is optimal by
  Theorem~\ref{thr:7}.  
\end{proof}

\begin{definition}[undirected walk]
\label{def:7}
  An undirected walk is a  sequence of the form
  $P=(v_0,a_1,v_1,\ldots,v_{m-1},a_m,v_m)$, where  $a_i =\in A$ for
  $i=1,\ldots,m$ and $a_i = (v_{i-1},v_i)$ or $a_i = (v_{i},v_{i-1})$. If
  the nodes $v_0,\ldots,v_m$ are all 
  different, then $P$ is a \emph{path}. The \emph{length } of $P$ is
  $m$. The \emph{distance} of two nodes  $u$ and $v$ is the length of
  a shortest path from $u$ to $v$. 
\end{definition}


Any directed path $P$ in $D(f)$ yields an undirected path  in
$D$. Define for such a path $P$ $\chi^P \in \{0,\pm1\}^A$ as
\begin{equation}
  \label{eq:29}
  \chi^P(a) = 
  \begin{cases}
    1 & \mbox{ if $P$ traverses $a$},\\
    -1&  \mbox{ if $P$ traverses $a^{-1}$},\\
    0 & \mbox{ if $P$ traverses neither $a$ or $a^{-1}$}. 
  \end{cases}
\end{equation}


\begin{theorem}[max-flow min-cut theorem, strong duality]
  \label{thr:8}
  The maximum value of a feasible $s-t$-flow is equal to the minimum
  capacity of an $s-t$ cut.
\end{theorem}

\begin{proof}
  Let $f$ be a maximum $s-t$-flow. Consider the residual graph $D(f)$
  with lower bound $l=0$. If this residual graph contains an
  $s-t$-path $P$, then we can route flow along this path. More
  precisely, there exists an $\epsilon>0$ such that $f + \epsilon \, \chi^P$ is feasible.
  We have $value(f + \epsilon \, \chi^P) = value(f) + \epsilon $. This contradics the
  maximality of $f$ thus there exists no $s-t$-path in $D(f)$. 
  
  Let $U$ be the nodes reachable from $s$ in $D(f)$. Then 
  $value(f)=u(\delta^{out}(U))$.  
\end{proof}
 


This suggests the algorithm of Ford and Fulkerson to find a maximum
flow. Start with $f = 0$. Next iteratively apply the following
\emph{flow augmentation algorithm}. 


Let $P$ be a directed $s-t$-path in $D(f)$. Set $f \gets f + \epsilon\chi^P$, where
$\epsilon$ is as large as possible to maintain $0\leq f\leq u$. 

\begin{exercise}  
  Define a \emph{residual capacity} for $D(f)$. Then determine the
  maximum $\epsilon$ such that $0\leq f\leq u$. 
\end{exercise}


\begin{theorem}
  \label{thr:9}
  If all capacities are rational, this algorithm terminates. 
\end{theorem}


\begin{exercise}
  Provide a proof of Theorem~\ref{thr:9}. 
\end{exercise}

\vspace{2cm} Here a picture with diamond needing exponentially many
augmentations. Explain! 



\begin{corollary}[integrity theorem]
  If $u(a) \in \setN$ for each $a \in A$, then there exists an integer maximum
  flow ($f(a) \in \setN$ for all $a \in A$). 
\end{corollary}




See~\cite{Schrijver03}

\begin{theorem}
\label{thr:10}
If we choose in each iteration a shortest $s-t$-path in $D(f)$ as a
flow-augmenting path, the number of iterations is at most $|V| \cdot|A|$. 
  
\end{theorem}

\begin{definition}
  Let $D = (V,A)$ be a digraph, $s,t \in V$  and let  $\mu(D)$ denote the
  length of a  shortest path from $s$ to $t$. Let $\alpha(D)$ denote the set
  of arcs contained in at least one shortest $s-t$ path.
\end{definition}

\begin{theorem}
  \label{thr:11}
  Let  $D = (V,A)$ be a digraph and  $s,t \in V$. Define $D' = (V,A \cup
  \alpha(D)^{-1})$. Then $\mu(D) = \mu(D')$ and $\alpha(D) = \alpha(D')$.
\end{theorem}


\begin{proof}
  It suffices to show that $\mu(D)$ and $\alpha(D)$ are invariant if we add
  $a^{-1}$ to $D$ for one arc $a \in \alpha(D)$. Suppose not, then there is a
  directed $s-t$-path $P_1$ traversing $a^{-1}$ of length at most
  $\mu(D)$. As $a \in \alpha(D)$ there is a path $P_2$ traversing $a$ of length
  $\mu(D)$. If we follow $P_2$  until the tail of $a$ is reached  and
  from thereon follow $P_1$, we obtain a path of $D$ of length less
  than $\alpha(D)$. This is a contradiction. 
\end{proof}


\begin{proof}[of Theorem~\ref{thr:10}]
  Let us augment flow $f$ along a shortest $s-t$-path $P$ in $D(f)$
  obtaining flow $f'$. The residual graph $D_{f'}$ is a subgraph of
  $D' = (V, A_f \cup\alpha(D(f))^{-1})$. Hence $\mu(D_{f'}) \geq \mu(D')=\mu(D(f))$. If
  $\mu(D_{f'})=\mu(D(f))$, then $\alpha(D_{f'})\subseteq\alpha(D') = \alpha(D(f))$. At least one
  arc of $P$ does not belong to $D_{f'}$, (the arc of minimum residual
  capacity!) thus the inclusion is
  strict. Since $\mu(D(f))$ increases at most $|V|$~times and, as long as
  $\mu(D(f))$ does not change, $\alpha(D(f))$ decreases at most $2\, |A|$~times, we
  have the theorem.
\end{proof}

In the following let $m = |A|$ and $n = |V|$. 

\begin{corollary}
  \label{co:2}
  A maximum flow can be found in time $\bigO(n \, m^2)$.
\end{corollary}




\section{Shortest Paths}
\label{sec:shortest-paths}

\begin{definition}[Cycle]
  A walk in which starting node and end-node agree is called a
  \emph{cycle}. 
\end{definition}

Suppose we are given a directed graph $D=(V,A)$ and a length function
$c:A\longrightarrow\setR$. The \emph{length} of a path $P$ is defined as 
\begin{displaymath}
  c(P) = \sum_{\substack{a \in A\\ a\in P}}  c(a). 
\end{displaymath}

We now study how to determine a shortest path in the weighted graph
$D$ efficiently, in case of the absence of cycles of negative length. 



\begin{theorem}
  \label{thr:1}
  Suppose that each cycle in $D$ has non-negative length and suppose
  there exists an $s-t$-walk in $D$. Then there exists a path
  connecting $s$ with $t$ which has minimum length among all walks
  connecting $s$ and $t$. 
\end{theorem}

\begin{proof}
  If there exists an $s-t$-walk, then there exists an $s-t$path. Since
  the number of arcs in a path is at most $|A|$, there must exist a
  shortest \emph{path}  $P$  connecting $s$ and $t$. We claim that
  $c(P)\leq c(W)$ for all $s-t$-walks $W$. Suppose that there exists a
  an $s-t$walk $W$ with $c(W)<c(P)$. Then let $W$ be such a walk with
  a minimum number of edges. Clearly $W$ contains a cycle $C$.  If the
  cycle has nonnegative length, then it can be removed from $W$ to
  obtain a walk whose length is at most $c(W)$ and whose number of
  edges is strictly less than $|C|$.   
\end{proof}

We use the notation $|W|,|C|,|P|$ to denote the number of edges in a
walk $W$ a cycle $C$ or a path $P$. 



As a conclusion we can note here: 
\begin{quote}
  If there do not exist negative cycles in $D$, and $s$ and $t$ are
  connected, then there exists a shortest path traversing at most 
  $|V|  - 1$ arcs. 
\end{quote}


\subsection*{The Bellman-Ford algorithm}

Let $n=|V|$. We calculate functions $f_0,f_1,\ldots,f_n:V\longrightarrow\setR\cup\{\infty\}$
successively by the following rule. 

\begin{enumerate}[i)]
\item $f_0(s) = 0$, $f_0(v) = \infty$ for all $v \neq s$ 
\item For $k<n$ if $f_k$ has been found, compute 
  \begin{displaymath}
    \displaystyle f_{k+1}(v) = \min\{f_k(v), \min_{(u,v)\in A}\{f_k(u)+c(u,v)\}  
  \end{displaymath}
  for all $v \in V$. 
\end{enumerate}

\begin{theorem}
  \label{thr:1}
  For each $k=0,\ldots,n$ and for each $v \in V$ 
  \begin{displaymath}
    f_k(v)  = \min\{ c(P \mid P \text{ is an } s-v-\text{Path traversing
    at most } k \text{ arcs}.
  \end{displaymath}
\end{theorem}

\begin{corollary}
  \label{co:7}
  If $D = (V,A) $ does not contain negative cycles w.r.t. $c$, then
  $f_n(v)$ is equal to the length of a shortest $s-v$-Path. The
  numbers $f_n(v)$ can be computed in time $O( |V| \cdot |A| )$. 
\end{corollary}


\begin{corollary}
  \label{co:9}
  In time $O(|V|^2 |A| )$  one can test whether $D = (V,A)$ has a
  negative cycle w.r.t. $c$ and eventually return one. 
\end{corollary}


\section{Minimum cost network flows, MCNFP}
\label{sec:minimum-cots-flows}

In contrast to the maximum $s-t$-flow problem, the goal here is to
route a flow, which comes from several sources and sinks through a
network with capacities and \emph{costs} in such a way, that the total
cost is minimized. 

\begin{example}
  Suppose you are given a directed graph width edge weights $D =
  (V,A)$,  $c: A \to \setR_{\geq0}$ and your task is to compute a shortest path
  from a particular node $s$ to all other nodes in the graph and
  assume that such paths exist. Then one can model this as a MCNFP by
  sending a flow of value $|V|-1$ into the source node and by letting
  a flow of value $1$ leave each node. The edges have infinite
  capacities. 
\end{example}




Here is a formal definition of a minimum cost network flow problem. In
this notation, vertices are indexed with the letters $i,j,k$ and edges
(arcs)  are denoted by their tail and head respectively, for example
$(i,j)$ denotes the edge from $i$ to $j$. 
  
For a node $i\in V$, the subset $I(i)\subseteq V$ denotes the incoming nodes of
$i$, i.e., the set $I(i) = \{ j \mid (j,i) \in A\}\subseteq V$. Similarly $O(i) = \{ j
\mid (i,j) \in A\}\subseteq V$ denotes the set of outgoing nodes. A network is now a
directed graph $D = (V,A)$ together with a capacity function $u: A \to
\setQ_{\geq0}$, a cost function $c: A \to\setQ$ and an external flow $b: V \to \setQ$.
The value of $b(i)$ denotes the amount of flow which comes from the
exterior. If $b(i)>0$, then there is flow from the outside, entering
the network through node $i$. If $b(i)<0$, there is flow which leaves
the network through $i$.

In the following we often use the notation $f(i,j)$ for the flow-value
on the arc $(i,j)$ (instead of $f((i,j))$). Similarly we write
$c(i,j)$ and $u(i,j)$. 

A \emph{feasible flow} is a function $f:A \to \setQ_{\geq0}$ which
satisfies the following constraints. 
\begin{displaymath}  
  \begin{array}{cl}
    \sum_{j \in O(i)} f{(i,j)} - \sum_{j\in I(i)} f{(j,i)} =b_i & \text{ for all } i \in V,\\
    0 \leq f{(i,j)} \leq u{(i,j)}              & \text{ for all } (i,j) \in A.
  \end{array}
\end{displaymath}


The goal is to find a feasible flow with minimum cost: 


  \begin{displaymath}
    \begin{array}{rcl}
      \text{minimize}     & \sum_{(i,j) \in A} c(i,j) f(i,j) &     \\
      \text{ subject to}  & \sum_{j \in O(i)} f(i,j) - \sum_{j\in I(i)} f(j,i) =b(i) & \text{ for all } i \in V,\\
                          & 0 \leq f(i,j) \leq u(i,j)              &
                          \text{ for all } (i,j) \in A 
    \end{array}
  \end{displaymath}
  
  
  Without loss of generality we can make the following assumptions: 
  
  \begin{enumerate}
  \item All data (cost, supply, demand and capacity) are integral.
  \item The supplies/demands at the nodes satisfy the condition 
    $\sum_{i \in V} b(i)=0$ and the MCNFP has a feasible solution. 
  \item The network contains an incapacitated directed path between
    every pair of nodes. 
  \item All arc costs are nonnegative. 
  \item The graph does not contain a pair of reverse arcs. 
  \end{enumerate}
  
  \begin{exercise}
    Show how to transform a MCNFP on a digraph with pairs of reverse
    edges into a MCNFP on a digraph with no pairs of reverse
    edges.  The number of edges and nodes should asymptotically remain
    the same. 
  \end{exercise}
  

  \begin{exercise}
    Prove that there exists no feasible flow unless $\sum_{i \in V} b(i)=0$
    holds. 
  \end{exercise}

  \begin{exercise}
    Why can we assume that the network has a path from $i$ to $j$ for
    all $i\neq j\in V$   which is incapacitated?
  \end{exercise}

  
  An \emph{arc-flow} of $D$ is a flow vector, that satisfies the
  nonnegativity and capacity constraints. 
 
  \begin{eqnarray*}
        \sum_{j \in O(i)} f((i,j)) - \sum_{j\in I(i)} f((j,i)) = - e(i) & &   \text{ for all } i \in V,\\
        0 \leq f((i,j)) \leq u((i,j))                 & &   \text{ for all }
        (i,j) \in A. 
  \end{eqnarray*}


  \begin{itemize}
  \item  If $e(i) >0$, then $i$ is an \emph{excess node} (more inflow than
    outflow).
  \item If $e(i) <0$, then $i$ is a \emph{deficit node} (more outflow than
    inflow).
  \item  If $e(i) = 0$ then $i$ is called \emph{balanced}.
  \end{itemize}
  
  
  \begin{exercise}
    Prove that $\sum_{i \in V} e(i) = 0$. 
  \end{exercise}



  
  Let $\paths$ be the collection of directed paths  of $D$ and
  let $\cycles$ be the collection of directed cycles of $D$. A
  path-flow is a function $\beta: \paths \cup \cycles \to \setR_{\geq0}$ which assigns
  flow values to paths and cycles. 

  For $(i,j)\in A$ and $P \in \paths$ let $\delta_{(i,j)}(P)$ be $1$ if $(i,j) \in
  P$ and $0$ otherwise. For $C \in \cycles$ let $\delta_{(i,j)}(C)$ be $1$ if
  $(i,j)\in C$ and $0$ otherwise. 

  A path-flow $\beta$ determines a unique   arc-flow
  \begin{displaymath}
    f(i,j) = \sum_{P \in \paths} \delta_{(i,j)}(P) \beta(P) + \sum_{C \in \cycles} \delta_{(i,j)}(C) \beta(C). 
  \end{displaymath}
    

  
  \begin{theorem}
    \label{thr:Decomp}
    Every path and cycle flow  has a unique representation
    as a  nonnegative arc-flow. Conversely, every nonnegative arc flow
    $f$ can be represented as a path and cycle flow with the following
    properties:
    \begin{enumerate}
    \item Every directed path with positive flow connects a deficit
      node with an excess node.
    \item At most $n+m$ paths and cycles have nonzero flow and at most
      $m$ cycles have nonzero flow.
    \end{enumerate}
    If the arc flow $f$ is integral, then so are the path and cycle
    flows into which it decomposes. 
    
  \end{theorem}

  \begin{proof}
   
  
  ``$\Rightarrow$'' See discussion above.
  
  ``$\Leftarrow$'' 
  
  Let $f$ be an arc flow. Suppose $i_0$ is a deficit node. Then there
  exists an incident arc $(i_0,i_1)$ which carries a positive flow. If
  $i_1$ is an excess node, we have found a path from deficit to excess
  node. Otherwise, the flow balance constraint at $i_1$ implies that
  there exists an arc $(i_1,i_2)$ with positive flow. Repeating this
  procedure, we finally must arrive at an excess node or revisit a
  node. This means that we either have constructed a directed path $P$
  from deficit node to excess node or a directed cycle $C$, both
  involving only edges with strictly positive flow.
  
  In the first case, let $P = i_0,\ldots,i_k$ be the directed path from
  deficit node $i_0$ to excess node $i_k$. We set $\beta(P) =
  \min\{-e_{i_0}, e_{i_k}, \min\{f{(i,j)} \mid (i,j) \in P\} \}$ and $f{(i,j)} =
  f{(i,j)} - \beta(P), \, (i,j) \in P$.  In the second case, set $\beta(C) =
  \min\{f{(i,j)} \mid (i,j) \in C$ and $f{(i,j)} = f{(i,j)} - \beta(C), \, (i,j)
  \in C$.  Repeat this procedure until all node imbalances are zero.
  
  Now find an arc with positive flow and construct a cycle $C$ by
  following only positive arcs from there. Set 
  $\beta(C) = \min\{f{(i,j)} \mid  (i,j) \in C\}$ and 
  $f{(i,j)} = f{(i,j)} - \beta(C),\, (i,j) \in C\}$. Repeat this process until
  there are no positive flow-edges left. 

  Each time a path or a cycle is identified, the excess/deficit of
  some node is set to zero or some edge is set to zero. This implies
  that we decompose into at most $n+m$ paths and cycles. Since cycle
  detection sets an edge to zero we have at most $m$ cycles. 
\end{proof}
  
  An arc flow $f$ with $e(i)=0$ for each $i \in V$ is called a
  \emph{circulation}. 
  
  \begin{corollary}
    A circulation can be decomposed into at most $m$ cycle flows.
  \end{corollary}

  
  Let $D = (V,A)$ be a network with capacities 
  $u{(i,j)}, \,  (i,j) \in A$ and costs $c{(i,j)}, \, (i,j) \in A$ and let
  $f$ be a feasible flow of the network. The \emph{residual network} $D(f)$ is
  defined as follows.

  \begin{itemize}
  \item We replace each arc $(i,j) \in A$ with two arcs $(i,j)$ and
    $(j,i)$.
  \item The arc $(i,j)$ has cost $c{(i,j)}$ and \emph{residual capacity}
    $r{(i,j)} = u{(i,j)} - f{(i,j)}$.  
  \item The arc $(j,i)$    has cost $-c{(i,j)}$ and
    residual capacity $r{(j,i)}=f{(i,j)}$. 
  \item      Delete all arcs which do not have strictly positive residual
    capacity. 
  \end{itemize}

  
  A directed cycle in $D(f)$ is called an \emph{augmenting cycle} of
  $f$.  

 
  \begin{exercise}
    Suppose that $f$ and $f^\circ$ are feasible flows. 
    Prove that $f  - f^\circ$  is a circulation in $D(f^\circ)$.  Here $f  -
    f^\circ$ is the flow  
    \begin{displaymath}
      (f-f^\circ)(i,j) = 
      \begin{cases}
        \max\{0, f(i,j) - f^\circ(i,j)\}, & \text{ if } (i,j) \in  A(D)\\
        \max\{0, f^\circ(j,i) - f(j,i)\}, & \text{ if } (j,i) \in  A(D)\\
        0, & \text{ otherwise.}
      \end{cases}
    \end{displaymath}
  \end{exercise}
   
  


  \begin{theorem}[Augmenting Cycle Theorem]
    \label{thr:augcyc}
    Let $f$ and $f^\circ$ be any two feasible flows of a network flow
    problem. Then $f$ equals $f^\circ$ plus the flow of at most $m$
    directed cycles in $D(f^\circ)$.  Furthermore the cost of $f$ equals
    the cost of $f^\circ$ plus the cost of flow on these augmenting
    cycles. 
  \end{theorem}
  
  \begin{proof}
    This can be seen by applying flow decomposition on the  flow 
    $f - f^\circ$  in $D(f^\circ)$. 
  \end{proof}
  

 
  
  \begin{theorem}[Negative Cycle Optimality Conditions]
    \label{thr:13}
    A feasible flow $f^*$ is an optimal solution of the minimum cost
    network flow problem, if and only if it satisfies the negative
    cycle optimality conditions: The residual network $D(f^*)$
    contains no directed cycle of negative cost.    
  \end{theorem}

  \begin{proof}
    

  ``$\Rightarrow$'' Suppose that $f$ is a feasible flow and that $D(f)$ contains
  a negative directed cycle. Then $f$ cannot be optimal, since we can
  augment positive flow along the corresponding cycle in the
  network. Therefore, if $f^*$ is an optimal flow, then $D(f^*)$
  cannot contain a negative directed cycle. 

  ``$\Leftarrow$'' Suppose now that $f^*$ is a feasible flow and suppose that
  $D(f^*)$ does not contain a negative cycle. Let $f^\circ$ be an optimal
  flow with $f^\circ \neq f^*$. The vector $f^\circ-f^*$ is a circulation with
  nonpositive cost $c^T(f^\circ-f^*) \leq0$. It follows from
  Theorem~\ref{thr:augcyc} that the cost of $f^\circ$ equals the cost of
  $f^*$ plus  the cost of directed cycles in the residual network
  $D(f^*)$.  The cost of these cycles is nonnegative, and therefore
  $c(f^\circ) \geq c(f^*)$ which implies that $f^*$ is optimal. 

\end{proof}

\begin{exercise}
  What is the maximum amount of flow, which can be augmented along a
  cycle in the residual network. Describe the flow after the
  augmentation and prove that it is feasible. 
\end{exercise}









\begin{algorithm}
  \label{alg:1}
  ~\\
  \begin{enumerate}
  \item  establish a feasible flow $f$ in the network
  \item {\tt WHILE} $D(f)$ contains a negative cycle
    \begin{enumerate}
    \item  detect a negative cycle $C$ in $D(f)$
    \item $\delta=\min\{r{(i,j)} \mid (i,j) \in C\}$
    \item augment $\delta$ units of flow along the cycle $C$
    \item update $D(f)$
    \end{enumerate}
  \item {\tt RETURN}  $f$
  \end{enumerate}
\end{algorithm}


  \begin{exercise}
    Provide an example of a MCNFP for which the cycle-canceling
    algorithm from above can require an exponential number of
    cancels, if the cycles are chosen in a stupid way. 
  \end{exercise}


  \begin{theorem}
    \label{thr:12}
    The cycle canceling algorithm terminates after a finite number of
    steps. 
  \end{theorem}
  \begin{proof}   
  By the assumption we have made on the MCNFP, the problem has an
  optimal feasible flow $f$. 
  
  The cycle canceling algorithm reduces the cost in each iteration.
  We have assumed that the input data is integral.
  Therefore the number of iterations is finite.
  
  
  
\end{proof}


\begin{corollary}
  \label{co:3}
  If all data are integral and if the MCNFP has a optimal flow, then
  it has an optimal flow with integer values only. 
\end{corollary}



%Consider the MCNFP

%\begin{displaymath}
%    \begin{array}{rcl}
%      \text{minimize}     & \sum_{(i,j) \in A} c{(i,j)} f{(i,j)} &     \\
%      \text{ subject to}  & \sum_{j \in O(i)} f{(i,j)} - \sum_{j\in I(i)} f{(j,i)} =b_i & \text{ for all } i \in V,\\
%                          & 0 \leq f{(i,j)} \leq u{(i,j)}              & \text{ for all } (i,j) \in A.
%    \end{array}
%  \end{displaymath}
    

%We transform this problem into standardform via slackvariables 
%$z{(i,j)}\geq0,\, (i,j) \in A$:

%\begin{displaymath}
%    \begin{array}{rrcll}
%      \text{minimize}     & \sum_{(i,j) \in A} c{(i,j)} f{(i,j)} &     \\
%      \text{ subject to}  & \sum_{j \in O(i)} f{(i,j)} - \sum_{j\in I(i)} f{(j,i)}
%      & = & b_i & \text{ for all } i \in V,\\
%                          & f{(i,j)} + z{(i,j)} & = &  u{(i,j)}
%                          & \text{ for all } (i,j) \in A, \\
%                    & f{(i,j)},z{(i,j)} & \geq & 0. &       \\
%    \end{array}
%  \end{displaymath}



  
%  The dual of this problem has variables $\pi_i, \, i \in V$ and 
%  $\alpha{(i,j)},  \,(i,j) \in A$ and is defined as follows:
  
  
%  \begin{displaymath}
%    \begin{array}{rrcll}
%      \text{maximize}     & \sum_{i \in V} b_i \pi_i  - \sum_{(i,j) \in A} u{(i,j)} \alpha{(i,j)}      \\
%      \text{ subject to}  & \pi_i - \pi_j - \alpha{(i,j)} 
%      & \leq & c{(i,j)} & \text{ for all } (i,j) \in A,\\
%                          & \alpha{(i,j)} & \geq & 0
%                          & \text{ for all } (i,j) \in A.
%  \end{array}
%  \end{displaymath}


Let $\pi : V  \to \setR$ be a function (\emph{node potential}). The
\emph{reduced cost} of an arc $(i,j)$ w.r.t. $\pi$ is 
$c_\pi((i,j))=c((i,j))+\pi(i) - \pi(j)$. The potential $\pi$ is called
\emph{feasible} if $c_\pi((i,j))\geq0$ for all arcs $(i,j)\in A$. 

\begin{lemma}
  \label{lem:7}
  Let $D = (V,A)$ be a digraph with edge weights $c:A\to\setR$. Then $D$
  does not have a negative cycle if and only if there exists a
  feasible node potential $\pi$ of $D$ with edge weights $c$. 
\end{lemma}


\begin{proof}
  Consider a directed path $P = i_0,i_1,\ldots,i_k$. The cost of this
  path is 
  \begin{displaymath}
    c(P) = \sum_{j=1}^{k}c((i_{j-1},i_j)).
  \end{displaymath}
  The reduced cost of this path is equal to 
  \begin{displaymath}
    c_\pi(P) = \sum_{j=1}^{k}c((i_{j-1},i_j)) + \pi(i_0) - \pi(i_k).
  \end{displaymath}
  If $P$ is a cycle, then $i_0$ and $i_k$ are equal, which means that
  its cost and reduced cost coincide. Thus, if there exists a feasible
  node potential, then there does not exist a negative cycle. 

  
  On the other hand, suppose that $D,c$ does not contain a negative
  cycle. Add a vertex $s$ to $D$ and the edges $(s,i)$ for all 
  $i \in  V$. The weights (costs) of all these new edges is $0$. Notice
  that in this way, no new cycles are created, thus still there does
  not exist a negative cycle. This means we can compute the shortest
  paths from $s$ to all other nodes $i \in V$. Let $\pi$ be the function
  which assigns these shortest paths lengths. Clearly $c_\pi((i,j)) =
  \pi(i) - \pi(j) + c((i,j))\geq0$, since the shortest-path length to $j$ is
  at most the shortest-path length to $i + c((i,j))$. 
\end{proof}

  
This means that we have again a nice way to prove that a flow is
optimal. Simply equip  the residual network with a feasible node
potential.  



\begin{corollary}[Reduced Cost Optimality Condition]
  \label{co:1}
  A feasible flow $f^*$ is optimal if and only if there exists a node
  potential $\pi$ such that the reduced costs $c_\pi{(i,j)}$ of  each arch
  $(i,j)$ of $D(f)$ are nonnegative. 
\end{corollary}






  The cycle canceling algorithm is only pseudopolynomial. If we could
  always chose a minimum cycle (cycle with best improvement) as an
  augmenting cycle, we would have a polynomial number of
  iterations. Finding minimum cycles is $NP$-hard. Instead we augment
  along \emph{minimum mean cycles}. One can find minimum mean cycles
  in polynomial time.  

  The \emph{mean cost} of a cycle $C \in \cycles$ is the cost of $C$
  divided by the number of edges in $C$:
  \begin{displaymath}
    (\sum_{(i,j) \in C} c{(i,j)})  / |C|.
  \end{displaymath}
  

  \begin{algorithm}[Minimum Mean Cycle Canceling, MMCC]
    \label{alg:2}
    ~\\
    \begin{enumerate}
    \item establish a feasible flow $f$ in the network
    \item {\tt WHILE} $D(f)$ contains a negative cycle
      \begin{enumerate}
      \item  detect a minimum mean cycle $C$ in $D(f)$ 
      \item   $\delta=\min\{r{(i,j)} \mid (i,j) \in C\}$
      \item   augment $\delta$ units of flow along the cycle $C$
      \item   update $D(f)$ 
      \end{enumerate}
    \item {\tt RETURN} $f$
    \end{enumerate}
  \end{algorithm}

  




  We now analyze the MMCC-algorithm. Let $\mu(f)$ denote the minimum
  mean-weight of a cycle in $D(f)$. 
  
  \begin{lemma}[See Korte \& Vygen \cite{MR1897297}]
    \label{lem:8}
    Let $f_1,f_2,\ldots$ be a sequence of feasible flows such that
    $f_{i+1}$ results from $f_i$ by augmenting flow along $C_i$, where
    $C_i$ is a minimum mean cycle of $D(f_i)$, then
    \begin{enumerate}
    \item \label{item:5} $\mu(f_k)\leq\mu(f_{k+1})$ for all $k$.
    \item \label{item:6} $\mu(f_k) \leq \frac{n}{n-1} \mu(f_l)$, where $k<l$
      and $C_k \cup C_l$ contains a pair of reversed edges.
    \end{enumerate}
  \end{lemma}



  \begin{proof}
    
    \ref{item:5}): Suppose $f_k$ and $f_{k+1}$ are two subsequent flows in this
    sequence. Consider the multi-graph $H$ which results from $C_k$
    and $C_{k+1}$ by deleting pairs of opposing edges.  The arcs of
    $H$ are a subset of the arcs of $D(f_k)$, since an arc of
    $C_{k+1}$ which is not in $D(f_k)$ must be a reverse arc of $C_k$.

    Each node in $H$ has even degree.  Thus $H$ can be
    decomposed into cycles, each of mean weight at least $ \mu(f_k)$.  
    Thus we have $c(A(H)) \geq \mu(f_k) |A(H)|$. 

    Since the total weight of each reverse pair of edges is zero we
    have 
    \begin{displaymath}
      c(A(H)) = c(C_k) + c(C_{k+1}) = \mu(f_k) |C_k|   + \mu(f_{k+1}) |C_{k+1}|.
    \end{displaymath}

    Since $|A(H)| \leq |C_k| + |C_{k+1}|$ we conclude 
    \begin{eqnarray*}
       \mu(f_k) ( |C_k| + |C_{k+1}|) & \leq &  \mu(f_k)|A(H)| \\
                              & \leq & c(A(H)) \\
                              & = & \mu(f_k) |C_k|   + \mu(f_{k+1}) |C_{k+1}|.
    \end{eqnarray*}

    Thus $\mu(f_k) \leq \mu(f_{k+1})$. 

    
    \ref{item:6}): By the first part of the theorem, it is enough to
    prove the statement for $k,l$ such that $C_i \cup C_l$ does not
    contain a pair of reverse edges for each  $ i, \, k < i < l$.  

    Again, consider the graph $H$ resulting from  $C_k$
    and $C_{l}$ by deleting pairs of opposing edges. $H$ is a subgraph
    of $D(f_k)$, since any edge of $C_l$ which does not belong to
    $D(f_k)$ must be a reverse edge of $C_k,C_{k+1},\ldots,C_{l-1}$. But
    only $C_k$ contains a reverse edge of $C_l$. So as above we have 
    \begin{displaymath}
      c(A(H)) = c(C_k) + c(C_{l}) = \mu(f_k) |C_k|   + \mu(f_{l})      |C_{k+1}|. 
    \end{displaymath}
    
    Since $|A(H)| \leq |C_k| + |C_{l}| -2$ we have $|A(H)| \leq
    \frac{n-1}{n}( |C_k| + |C_{l}|)$. Thus we get
    \begin{eqnarray*}
       \mu(f_k)\frac{n-1}{n} ( |C_k| + |C_{l}|) & \leq &  \mu(f_k)|A(H)| \\
                              & \leq & c(A(H)) \\
                              & = & \mu(f_k) |C_k|   + \mu(f_{l})|C_{l}|\\          
                              & \leq & \mu(f_{l}) (|C_k|   + |C_{l}|) \\                       
    \end{eqnarray*}
    This implies that $\mu(f_k) \leq \frac{n}{n-1}  \mu(f_l)$.  
    
  \end{proof}






  \begin{corollary}
    \label{co:4}
    During the execution of the MMCC-algorithm, $|\mu(f)|$ decreases by
    a factor of $1/2$ every $n\cdot m$ iterations. 
  \end{corollary}

  \begin{proof}
    Let $C_1, C_2,\ldots$ be the sequence of augmenting cycles. Every
    $m$-th iteration, there must be an edge of the cycle, which is
    reverse to one of the succeeding $m-1$ cycles, because every
    iteration, one edge of the residual network will be deleted. 
    Thus after $n \, m$ iterations, the absolute value of $\mu$ has
    dropped by $\left(\frac{n-1}{n} \right)^n \leq e^{-1} \leq 1/2$. 
  \end{proof}

  \begin{corollary}
    \label{co:5}
    If all data are integral, then the
    MMCC-algorithm runs in polynomial time. 
  \end{corollary}
  
  \begin{proof}
    \begin{itemize}
    \item A lower bound on $\mu$ is the smallest cost $c_{min}$
    \item $|\mu|$ drops by $1/2$ every $m \, n$ iterations. 
    \item After $m n \log n |c_{min}|$ iterations, absolute value
    of minimum mean weight cycle   drops below $1/n$, thus is zero.
    \item {\bf We need to prove that a minimum mean cycle can be found        in polynomial time} 
    \end{itemize}
  \end{proof}


This is a so-called \emph{weakly polynomial} bound, since the binary
encoding length of the numbers in the input (here the costs)
influences the running time. We now prove that the MMCC-algorithm is
\emph{strongly polynomial}. 






\begin{theorem}[See Korte \& Vygen~\cite{MR1897297}]
  The MMCC-algorithm requires  $\bigO(m^2 \, n \log n)$ iterations
  (mean weight cycle cancellations). 
\end{theorem}

\begin{proof}
  One shows that every $m\, n  (\lceil\log n\rceil+1)$ iterations, at least one
  edge is \emph{fixed}, which means that the flow through  this edge
  does not change anymore. 
  
  Let $f_1$ be some flow at some iteration and let $f_2$ be the flow
  $m\, n  (\lceil\log n\rceil+1)$ iterations later. 
  It follows from Corollary~\ref{co:4} that 
  \begin{equation}
    \label{eq:33}
    \mu(f_1)\leq 2 \, n \, \mu(f_2) 
  \end{equation}
  %
  holds. 
  
  Define the costs $c'(e) = c(e) - \mu(f_2)$ for the residual network
  $D(f_2)$. There exists no negative cycle in $D(f_2)$ w.r.t. this
  cost $c'$. ( A cycle $C$ has weight $c'(C) = \sum_{e\in C} c(e) - |C|
  \mu(f_2)$ and thus $c'(C) / |C| = \sum_{e\in C} c(e) / |C| - \mu(f_2)\geq0$).
  By Lemma~\ref{lem:7} there exists a feasible node potential $\pi$ for
  these weights. One has $0 \leq c'_\pi(e) = c_\pi(e) - \mu(f_2)$ and thus
  \begin{equation}
    \label{eq:32}
     c_\pi(e) \geq \mu(f_2), \text{ for all } e \in A(D(f_2)). 
  \end{equation}

  Let $C$ be a minimum mean cycle of $D(f_1)$. One has
  \begin{equation}
    \label{eq:34}
    c_\pi(C) = c(C) = \mu(f_1) \, |C| \leq 2 \, n \, \mu(f_2) |C|.
  \end{equation}
  
  It follows that there exists an arc $e_0$ of $C$ such that
  \begin{equation}
    \label{eq:36}
    c_\pi(e_0)\leq 2 \, n \, \mu(f_2)
  \end{equation}
  %
  holds. The inequalities~\eqref{eq:32}
  imply that  $e_0 \notin A(D(f_2))$ 

  We now make the following
  claim:
  %
  \begin{quote}
    Let $f'$ be a feasible flow such that $e_0 \in D(f')$, then 
    $\mu(f') \leq    \mu(f_2)$. 
  \end{quote}
  
  If we have shown this claim, then it follows from Lemma~\ref{lem:8}
  that $e_0$ cannot be anymore in the residual network of a flow after
  $f_2$.  Thus the flow along the edge $e_0$ (or $e_0^{-1}$) is
  fixed. 

  Let $f'$ be a flow such that $e_0\in A(D(f'))$. Recall that $f' - f_2$
  is a circulation in $D(f_2)$ where $e_0\notin D(f_2), \,e_0^{-1}\in D(f_2)$  and this
  circulation sends flow over $e_0^{-1}$. This circulation can be
  decomposed into cycles and one of these cycles $C$ contains
  $e_0^{-1}$.  One has $c_\pi(e_0^{-1}) = -c_\pi(e_0) \geq -  2 \, n \,
  \mu(f_2)$ (eq.~\eqref{eq:36}). Using~\eqref{eq:32} one obtains
  \begin{eqnarray}
    \label{eq:37}
    c(C) & = &  \sum_{e \in C} c_\pi(e) \\
         & \geq &  -  2 \, n \,  \mu(f_2) + (n-1) \mu(f_2) \\
         & = & - (n +1)\, \mu(f_2) \\
         & > & -n \, \mu(f_2). 
  \end{eqnarray}
  %
  The reverse of $C$ is an augmenting cycle for $f'$ with total weight
  at most $n \, \mu(f_2)$ and thus with mean weight at most
  $\mu(f_2)$. Thus $\mu(f') \leq \mu(f_2)$. 
\end{proof}






\section{Computing a minimum cost-to-profit ratio cycle}
\label{sec:comp-minim-cost}

Given a digraph $D = (V,A)$ with costs $c:A\to\setZ$ and profit $p:A \to
\setN_{>0}$, the task is to compute a cycle $C \in \cycles$ with minimum
ratio 
\begin{equation}
  \label{eq:31}
  \frac{c(C)}{p(C)}.
\end{equation}

Notice that this is the largest number $\beta\in \setQ$ which satisfies 
\begin{equation}
  \beta\leq  \frac{c(C)}{p(C)}, \, \text{ for all } C \in \cycles. 
\end{equation}

By rewriting this inequality, we understand this to be the largest
number $\beta\in \setQ$ such that 
\begin{equation}
  c(C) - \beta\, p(C) \geq0 \, \text{ for all } C \in \cycles. 
\end{equation}
In other words, we search the largest number $\beta \in \setQ$ such that the
digraph $D= (V,A)$ with costs $c_\beta:A \to \setQ$, where 
$c_\beta(e) = c(e) - \beta\,p(e)$.  

We need a routine to check whether $D$ has a negative cycle for a
given weight function $c$. For this we assume w.l.o.g. that each 
vertex is reachible from the vertex $s$, if necessary by intruducing a
new vertex $s$ from which there is an edge with cost and profit $0$ to
all other nodes. The minimum cost-to-profit ration cycle w.r.t. this
new graph is then the minimum cost to profit ratio cycle w.r.t. the
original graph, since $s$ is not a vertex of any cycle. 

Recall the following single-source shortest-path algorithm  of
Bellman-Ford which we now apply with weights $c_\beta$: 


{\small
  \begin{quote}
    Let $n=|V|$. We calculate functions $f_0,f_1,\ldots,f_n:V\longrightarrow\setR\cup\{\infty\}$
    successively by the following rule. 
    
    \begin{enumerate}[i)]
    \item $f_0(s) = 0$, $f_0(v) = \infty$ for all $v \neq s$ 
    \item For $k<n$ if $f_k$ has been found, compute 
      \begin{displaymath}
        \displaystyle f_{k+1}(v) = \min\{f_k(v), \min_{(u,v)\in A}\{f_k(u)+c_\beta(u,v)\}  
      \end{displaymath}
      for all $v \in V$. 
    \end{enumerate}
  \end{quote}
}

There exists a negative cycle w.r.t. $c_\beta$ if and only if $f_n(v) <
f_k(v)$  for some $v \in V$ and $1\leq k<n$. Thus we can test in
$O(m\cdot n)$ steps whether $D,c_\beta$ contains a negative cycle. 

We now apply the following idea to  search for the correct value of
$\beta$. We keep an interval $I = [L,U]$ with the invariant that the
value  $\beta$ that we are searching lies in this interval $I$. As
starting values, we can chose $L = c_{min}$ and $U = c_{max}$, 
where $c_{min}$ and $c_{max}$ are the smallest and largest cost
respectively. In one iteration we compute $M = (L + U) /2$. We then
check whether $D$, together with $c_M$ contains a negative cycle. If
yes, we know that $\beta$ is at least $M$ and we set $L\gets M$. If not, then
$\beta$ is at most $M$ and we update the upper bound $U \gets M$. 

When can we stop this procedure? We can stop it, if we can assure that
only one valid cost-to-profit ratio cycle  lies in $[L,U]$. Suppose that $C_1$
and $C_2$ have different cost-to-profit ratios. Then 
\begin{eqnarray}
 | c(C_1) /  p(C_1) - c(C_2)/ p(C_2) | & = & \left| \frac{c(C_1)\, p(C_2) -
 c(C_2)p(C_1)}{  (p(C_1) \, p(C_2))}\right| \\
                                & \geq & 1/ (n^2 p_{max}^2). 
\end{eqnarray}
%
Thus we can stop our process, if $U-L< 1/ (n^2 p_{max}^2)$, since we
know then that there can be only one cycle $c \in \cycles$ with
$c(C)/p(C) \in [L,U]$.

Suppose that $[L,U]$ is the final interval. 
We know then that 
\begin{displaymath}
  L \leq c(C) / p(C) \text{ for all } C \in \cycles 
\end{displaymath}
and 
\begin{displaymath}
  U> c(C)/p(C) \text{ holds for some } C \in \cycles. 
\end{displaymath}
Let $C$ be a minimum weight cycle w.r.t. the edge costs $c_L$. Clearly
$ U > c(C)/p(C) \geq L$ holds and thus $C$ is the minimum cost-to-profit
cycle we have been looking for. 

Let us analyze the number of required iterations. We need to halve the
starting interval-length $2 \, c$, where $c$ is the largest absolute
value of a cost, until the length is at most $1/ (n^2 p_{max}^2)$. 
We search the minimal $i\in \setN$ such that
\begin{equation}
  \label{eq:35}
  (1/2)^i c \leq 1/(n^2 p_{max}^2).
\end{equation}
This shows us that we need $\bigO(\log( c \,p_{max}^2 n^2))$
iterations which is $\bigO(\log n \log K)$, where $K$ is the largest
absolute value of a cost or a profit. 



\begin{theorem}[Lawler~\cite{MR0439106}]
  \label{thr:15}
  Let $D$ be a digraph  with costs $c:A\to\setZ$ and profit $p: A\to \setN_{>0}$
  an let $K \in \setN$   such that $|c(e)| + |p(e)| \leq K$ for all $e \in \setN$. A
  minimum cost-to-profit ratio cycle of $G$ can be computed in time
  $\bigO(m\,n \, \log n\, \log K)$.
\end{theorem}



But we knew a weakly polynomial algorithm for MCNFP from the
exercises. So you surely ask: Can we do better for minimum
cost-to-profit cycle computation? The answer is ``Yes''!



\subsection{Parametric search} 
\label{sec:parametric-search}




Let us first roughly describe  the  idea on  how to obtain a
strongly polynomial algorithm, see~\cite{Megiddo79}.  The Bellman-Ford
algorithm tells us whether our  current $\beta$ is too large or too
small, depending on whether $D$ with weights $c_\beta$ contains a
negative cycle or not. Recall that the B-F algorithm computes labels
$f_i(v)$ for $v \in V$ and $1\leq i\leq n$.  If these labels are computed
with costs $c_\beta$, then they are \emph{piecewise linear} functions in
$\beta$ and we denote them by $f_i(v)[\beta]$. 

Denote the optimal $\beta$ that we look for by $\beta^*$ and suppose that we
know an interval $I$ with such that $\beta^* \in I$ and each function
$f_i(v)[\beta]$ is linear if it is restricted to this domain $I$. Then we
can determine $\beta^*$ as follows.

Let $I = [L,U]$ be the interval and  remember that we are searching for the
largest value of $\beta \in I$ such that $f_{n}(v)[\beta]=f_{n-1}(v)[\beta]$
holds for each $v \in V$. Clearly this holds for $\beta = L$. Thus we only
need to check whether $\beta = U$ by computing the values $f_{n}(v)[U]$
and $f_{n-1}(v)[U]$ for each $v \in V$ and check whether one of these
pairs consists of different numbers. 


The idea is now to compute such an interval $I = [L,U]$ in strongly
polynomial time. 

Consider the function $f_1(v)[\beta]$. Clearly one has 
\begin{displaymath}
f_1(v)[\beta] = 
  \begin{cases}
    c(s,v) - \beta \cdot p(s,v) &  \text{ if } (s,v)\in A, \\
    \infty                  & \text{ otherwise}. 
  \end{cases}
\end{displaymath}

This shows that $f_1(v)[\beta]$ is a linear function in $\beta$ for each $v
\in V$. 

Now suppose that $i\geq1$ and that we have computed an interval
$I=[L,U]$ with  $\beta^*\in I$ and each function $f_i(v)[\beta]$ is a linear
function if $\beta$ is restricted to $I$. 

Now consider the function  $f_{i+1}(v)[\beta]$ for a particular $v \in
V$. Recall the formula 
\begin{equation}
  \label{eq:1}
  f_{i+1}(v)[\beta] = \min\{ f_i(v)[\beta], \min_{(u,v)\in A} \{ f_i(u)[\beta] +
  c(u,v) - \beta \cdot p(u,v)\} \}. 
\end{equation}


Each of the functions $ f_i(v)[\beta]$ and $ f_i(u)[\beta] +  c(u,v) - \beta
\cdot p(u,v)$  are linear on $I$.  The function $f_i(v)[\beta]$ can be retrieved by
computing a shortest path $P_i(v)$ from $s$ to $v$ with edge weights $c_\beta$ for
some $\beta$ in $(L,U)$ which uses at most $i$ arcs. If $\beta$ is then
allowed to vary, the line which is defined by $f_i(v)[\beta]$ on $I$ is
then the length of this path $P$   with parameter $\beta$.  Similarly we
can retrieve the functions (lines) $ f_i(u)[\beta] +  c(u,v) - \beta
\cdot p(u,v)$  for each $(u,v) \in A$. With the Bellman-Ford algorithm,
this amounts to a running time of $O(m \cdot n)$. 

We now have $n$ lines and can now compute the lower envelope of these
lines in time $O(n \log n)$ alternatively we can also compute all
intersection points of these lines and sort them w.r.t. increasing
$\beta$-coordinate. This would amount to $O(n^2 \log n)$.
Let $\beta_1,\ldots,\beta_k$ be the sorted list of  these  $\beta$-coordinates.
Now  $\beta_{trial}:= \beta_{\lfloor k/2\rfloor}$ and
check whether $\beta^*>\beta_{trial}$. If yes, we can replace $L$ by
$\beta_{trial}$ and we can delete the numbers
$\beta_1,\ldots,\beta_{\lfloor k/2\rfloor-1}$. Otherwise, we replace $U$ by $\beta_{trial}$ and
delete $\beta_{\lfloor k/2\rfloor+1},\ldots,b_k$. In
any case, 
we halved the number of possible $\beta$-coordinates and continue in this
way.  Such a check requires a negative cycle test in the graph
$D$ with edge weights $\beta_{trial}$ and costs $O(m \cdot n)$. In the end we have two consecutive $\beta$-coordinates and have an
interval $[L,U]$ on which  $f_{i+1}(v)[\beta]$ is linear.  To  find
an interval $I$ such that $f_{i+1}(v)[\beta]$ is linear on $I$ and $\beta^*
\in I$ costs thus $O(m \cdot n \log n)$ steps. 


We now continue to tighten \emph{this interval} such that all functions
$f_{i+1}(v)[\beta], v \in V$ are linear on $[L,U]$.  Thus in step $i+1$ this
amounts to a running time of 
\begin{displaymath}
  O\left(n \cdot ( m \cdot n \log n) \right). 
\end{displaymath}

The total running time  is thus 
\begin{displaymath}
  O(n^3 \cdot m \cdot \log n).  
\end{displaymath}

\begin{theorem}
  \label{thr:2}
  Let $D=(V,A)$ be a directed graph and let $c: A \longrightarrow\setR$ and
  $p:A\longrightarrow\setR_{>0}$ be functions. One can compute a cycle $C$ of $D$
  minimizing $c(C)/p(C)$ in time  $O(n^3 \cdot m \cdot \log n)$.  
\end{theorem}


%%% Local Variables: 
%%% mode: latex
%%% TeX-master: "lecture"
%%% TeX-master: "lecture"
%%% End: 
