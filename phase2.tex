\chapter{The two-phase simplex method}
\label{cha:two-phase-simplex}

We now deal with the first question raised at the end of
Chapter~\ref{cha:introduction}. How do we find an initial basic
feasible solution with which the simplex algorithm is started? \emph{Phase
one} of the simplex method deals with the computation of an initial
feasible basis, which is then handed over to \emph{phase two}, the simplex
method as we described it so far. 

\section{Phase one}
\label{sec:phase-1}

Suppose we have to solve a linear program 
\begin{equation}
  \label{eq:11}
  \begin{array}{rcl}
    \min c^Tx \\
    A x & = & b \\
    x & \geq& 0.
  \end{array}
\end{equation}
By multiplying some rows with $-1$ if necessary, we can achive that
the right-hand-side $b$ satisfies $b\geq0$. From this, we construct a
linear program from which an initial basic solution is readily
available
\begin{equation}
  \label{eq:12}
  \begin{array}{rcl}
    \min z_1+ \cdots+ z_m \\
    A x +z & = & b \\
    x,z & \geq& 0.
  \end{array}
\end{equation}
The set $B = \{n+1,\ldots,n+m\}$ is a feasible basis with basic feasible
solution $(x^*,z^*)$ defined by $x^* = 0$  and $z^* = b$. The next
lemma is immediate. 
\begin{lemma}
  \label{lem:4}
  The linear program~\eqref{eq:11} is infeasible if and only if the
  optimum value of the linear program~\eqref{eq:12} is larger than
  zero. 
\end{lemma}
\begin{proof}
  If $x^*$ is a feasible solution of~\eqref{eq:11}, then $(x^*,0)$ is
  a feasible solution of~\eqref{eq:12} with objective function value
  equal to zero. Thus if~\eqref{eq:11} is feasible, then the optimum
  value of \eqref{eq:12} is  zero (it cannot be smaller, since the
  $z_i$ are nonnegative).  

  If now, on the other hand, the optimum value of~\eqref{eq:12} is
  zero, we can show that~\eqref{eq:11} is feasible. Since then, if
  $(x^*,z^*)$ is an optimum solution of~\eqref{eq:12}, then $z^* = 0$
  and thus $x^*$ is a feasible solution of~\eqref{eq:11}. \qed
\end{proof}


{Phase one} of the simplex method consists of solving the linear
program~\eqref{eq:12} with the initial feasible basis
$B=\{n+1,\ldots,n+m\}$.  If the objective value of the computed solution
of~\eqref{eq:12} is larger than zero, we assert that \eqref{eq:11} is
infeasible. 


Otherwise, the goal is now to extract an initial feasible basis of
this solution. 
Let $B' \subseteq\{1,\ldots,m+n\}$ be the basis returned by the simplex
algorithm and let $(x^*,0)$ be the associated basic feasible solution. If
$B' \subseteq\{1,\ldots,m\}$, then this $B'$ is already a feasible  basis
of~\eqref{eq:11}. It could however be that $B' \cap \{n+1,\ldots,n+m\} \neq
\emptyset$. We could augment $B'
\cap\{1,\ldots,m\}$ to a basis $B$ of $Ax = b, x\geq0$, provided that $A$ has
full row-rank. 

Instead we describe  an alternative method which uses tableau mechanism to
\emph{drive the artificial variables out} of the basis $B'$. We
describe it now. Let $B' = \{B_1,\ldots,B_m\}$ and suppose that $B_\ell\in
\{n+1,\ldots,n+m\}$. We inspect the tableau of the basis $B'$
of~\eqref{eq:11} 
\begin{equation}
  \label{eq:13}
    \begin{array}{c|c}
    * & * \\ \hline 
    A_{B'}^{-1}A        & x^*_{B'}.
  \end{array}
\end{equation}
If the $\ell$-th row of $ A_{B'}^{-1}A $ is zero, then the $\ell$-th row
of the system $ A_{B'}^{-1}A x =  A_{B'}^{-1}b$ can be deleted from
the tableau and $B':= B' - \{B_\ell\}$. 

Otherwise there exists an entry in this row, which is unequal to zero.
Let this entry be in column $j$. We now let $j$ enter the basis and
let $B_\ell$ leave the basis by the appropriate pivot operations. 
Observe that you may have to multiply the corresponding row by $-1$,
if the pivot entry would be negative. 
We continue these steps until there are no artificial variables left in
the basis. 


\subsection{An example}

The following example is taken from~\cite[p.~114]{BertsimasTsitsiklis97}.

Consider the linear program
\begin{equation*}
  \begin{array}{rccccccccc}
    \min &  x_1 & + & x_2 & + & x_3 & & &  & \\
    &x_1&+&2x_2&+&3x_3&&&=&3\\ 
    &-x_1&+&2x_2&+&6x_3&&&=&2\\ 
    &&-&4x_2&-&9x_3&&&=&-5\\ 
    &&&&&3x_3&+&x_4&=&1 \\
    &&&&x_1,&x_2,&x_3,&x_4&\geq&0\\
  \end{array}
\end{equation*}
We form the auxiliary linear program to initialize phase one of the
simplex algorithm. 

\begin{displaymath}
  \begin{array}{l}
     \min   \quad \quad \quad \quad\quad \quad \quad \quad\quad  \quad \quad  x_5+x_6+x_7+x_8\\
  \begin{array}{ccccccccccccccccc}   
   x_1&+&2x_2&+&3x_3&&&+ &x_5&&&&&&&=&3\\ 
   -x_1&+&2x_2&+&6x_3&&&&&+&x_6&&&&&=&2\\ 
   &&4x_2&+&9x_3&&&&&&&+&x_7&&&=&5\\ 
   &&&&3x_3&+&x_4&&&&&&&+&x_8&=&1 
  \end{array}\\
  x_1,x_2,x_3,x_4,x_5,x_6,x_7,x_8\geq0.
\end{array}
\end{displaymath}
%
By evaluating the reduced cost of the basis $B = \{5,6,7,8\}$ we
arrive at the tableau 




\begin{displaymath}
  \begin{tabularx}{\textwidth}{XXXXXXXX|X}   
   0&  -8& -21&  -1&   0&   0&   0&   0& -11\\\hline
   1&   2&   3&   0&   1&   0&   0&   0&   3\\
  -1&   2&   6&   0&   0&   1&   0&   0&   2\\
   0&   4&   9&   0&   0&   0&   1&   0&   5\\
   0&   0&   3&   1&   0&   0&   0&   1&   1\\
 \end{tabularx}
\end{displaymath}
We let $4$ enter the basis and $8$ leaves the basis. The tableau for
$B = \{4,5,6,7\}$ is 
\begin{displaymath}
  \begin{tabularx}{\textwidth}{XXXXXXXX|X}   
   0&  -8& -18&   0&   0&   0&   0&   1& -10\\\hline
   1&   2&   3&   0&   1&   0&   0&   0&   3\\
  -1&   2&   6&   0&   0&   1&   0&   0&   2\\
   0&   4&   9&   0&   0&   0&   1&   0&   5\\
   0&   0&   3&   1&   0&   0&   0&   1&   1\\
 \end{tabularx}
\end{displaymath}
Next, $3$ enters the basis and $4$ leaves again. We obtain the tableau
of $B = \{3,5,6,7\}$. 
\begin{displaymath}
  \begin{tabularx}{\textwidth}{XXXXXXXX|X}   
   0&  -8&   0&   6&   0&   0&   0&   7&  -4\\\hline
   0&   0&   1& 1/3&   0&   0&   0& 1/3& 1/3\\
   1&   2&   0&  -1&   1&   0&   0&  -1&   2\\
  -1&   2&   0&  -2&   0&   1&   0&  -2&   0\\
   0&   4&   0&  -3&   0&   0&   1&  -3&   2\\
 \end{tabularx}
\end{displaymath} 
Next, $2$ enters the basis and $6$ leaves the basis. We obtain the
tableau of $B = \{2,3,5,7\}$ 
\begin{displaymath}
  \begin{tabularx}{\textwidth}{XXXXXXXX|X}   
   -4&    0&    0&   -2&    0&    4&    0&   -1&   -4\\\hline
   -1/2&    1&    0&   -1&    0&  1/2&    0&   -1&    0\\
   0&    0&    1&  1/3&    0&    0&    0&  1/3&  1/3\\
   2&    0&    0&    1&    1&   -1&    0&    1&    2\\
   2&    0&    0&    1&    0&   -2&    1&    1&    2\\

  \end{tabularx}  
\end{displaymath}
Next, $1$ enters the basis and $5$ leaves. The tableau of $B =
\{1,2,3,7\}$ is 
\begin{displaymath}
  \begin{tabularx}{\textwidth}{XXXXXXXX|X}  
       0&    0&    0&    0&    2&    2&    0&    1&    0\\\hline
       1&    0&    0&  1/2&  1/2& -1/2&    0&  1/2&    1\\
       0&    1&    0& -3/4&  1/4&  1/4&    0& -3/4&  1/2\\
       0&    0&    1&  1/3&    0&    0&    0&  1/3&  1/3\\
       0&    0&    0&    0&   -1&   -1&    1&    0&    0\\
  \end{tabularx}  
\end{displaymath}
Now we see that the $4$-th row of the matrix $A_B^{-1}A$ is
zero. Thus, we can delete the last constraint. We arrive at the
tableau 
\begin{displaymath}
  \begin{tabularx}{.8\textwidth}{XXXX|X}  
       0&    0&    0&    0&    0\\\hline
       1&    0&    0&  1/2&    1\\
       0&    1&    0& -3/4&  1/2\\
       0&    0&    1&  1/3&  1/3\\
  \end{tabularx}  
\end{displaymath}
which is yields a feasible basic solution $x_1=1, x_2=1/2,
x_3=1/3$. Phase two of the simplex algorithm is initiated with the
tableau belonging to the basis $\{1,2,3\}$. 
\begin{displaymath}
  \begin{tabularx}{.8\textwidth}{XXXX|X}  
       0&    0&    0&    -1/12&    -11/6\\\hline
       1&    0&    0&  1/2&    1\\
       0&    1&    0& -3/4&  1/2\\
       0&    0&    1&  1/3&  1/3\\
  \end{tabularx}  
\end{displaymath}



After one more iteration one arrives at the tableau  corresponding of
the basis $B=\{1,2,4\}$. 
\begin{displaymath}
   \begin{tabularx}{.8\textwidth}{XXXX|X}  
        0&    0&  1/4&    0& -7/4\\\hline
        1&    0& -3/2&    0&  1/2\\
        0&    1&  9/4&    0&  5/4\\
        0&    0&    3&    1&    1\\
   \end{tabularx}  
\end{displaymath}


\section{The dual simplex method}
We briefly discuss also the dual simplex method. 

\begin{definition}[Dual feasible basis]
  A basis $B$ which yields nonnegative reduced cost $\overline{c}^T =
  c^T - c_B^TA_B^{-1}A \geq0$ is called a \emph{dual feasible basis}.
\end{definition}

In the dual simplex method, we are given a tableau 
\begin{equation}
  \label{eq:14}
  \begin{array}{c|c}
    c^T - c_B^TA_B^{-1}A & -c_B^Tx^*_B \\ \hline 
    A_B^{-1}A        & x^*_B.
  \end{array}
\end{equation}
together with a dual feasible basis $B$ defining the tableau. 
Suppose the basic solution $x^*$ with $x^*_B = A_B^{-1}b$ and $x^*_{\overline{B}} =
0$ is not feasible, or in other words, the basis $B$ is not (primal)
feasible, however dual feasible. We want to pivot to a new basis $B'$
which remains dual feasible, whose objective function value
$c_{B'}^Tx^*_{B'}$ is larger compared to $c_B^Tx^*_B$. Remember that
this value is equal to the objective value of the corresponding dual
solution and that  the dual linear program $\max\{b^Ty \colon A^Ty\leq c, \, x \in
\setR^n\}$  is a maximization problem. 

We now describe the pivot step which yields an improved dual feasible
solution if the linear program does not have any degenerate basic
solutions. By applying perturbation, or a suitable pivot rule, we can
thereby achieve that the dual simplex method also eventually
terminates. 


Suppose that $x^*(B_i)$ is less than zero. We choose $B_i$ to
\emph{exit} the basis $B$ and search for a $j \in \{1,\ldots,n\}$ that
should enter the basis $B$ to form the new basis $B'$. This is in
contrast to the primal simplex method, where we are choosing an
entering variable first and then determine an exiting variable. 


Let $a_1,\ldots,a_n, x^*(B_i)$ be the $i$-th row of the system matrix of
the tableau~\eqref{eq:14}. 

\begin{lemma}\label{lem:5}
  If $a_i\geq0$ for each $i=1,\ldots,n$, then the linear program is
  infeasible and the dual linear program is unbounded.   
\end{lemma}

\begin{proof}
  The proof is straightforward with linear programming duality.
  Clearly, there cannot be a feasible solution $\wt{x}\geq0$ of the
  primal, since with this $\wt{x}$ one has $ \sum_{i=1}^n a_i \wt{x}_i =
  x^*(B_i)<0$ which is impossible, since the $a_i$ are all greater
  than or equal to $0$. The theorem of strong duality then implies
  that the dual linear program is unbounded.  \qed
\end{proof}


Now let $j$ be an index with $a_j<0$. If we let $j$ enter the basis,
then we must transform the tableau in such a way that the $j$-th
column of the tableau becomes the $i$-th unit vector. We want to
ensure that the tableau is still dual feasible. 

To this end, we consider the $i$-th  row of the system matrix. 
\begin{displaymath}
  a_1,\ldots,a_j,\ldots,a_n,x^*(B_i)
\end{displaymath}
and multiply it with $1/a_j$ to obtain. 
\begin{displaymath}
  a_1/a_j,\ldots,a_{j-1}/a_j,1,a_{j+1}/a_j,\ldots,a_n/a_j,x^*(B_i)/a_j
\end{displaymath}
If we perform now elementary row operations on the tableau to have the
$i$-th unit vector in the $j$-th column, then we would want the new
reduced costs to remain nonnegative. The $k$-th reduced cost is 
\begin{displaymath}
  \overline{c}_k - \overline{c}_j a_k/a_j.
\end{displaymath}
If $a_k$ was positive, then the new reduced cost of $k$ is clearly
positive as well. If $a_k<0$, then the new reduced cost of $k$ is
nonnegative if and only if 
\begin{displaymath}
    \overline{c}_k /a_k\leq \overline{c}_j /a_j. 
\end{displaymath}
Let $K = \{j \colon a_j<0\}$ be the indices corresponding to negative
entries of the $i$-th row of the system matrix of the tableau. 
If $j\in K$ is an index with $a_j<0$ such that 
\begin{displaymath}
  \overline{c}_k /a_k\leq \overline{c}_j /a_j ,\, \text{ for all }k \in K,
\end{displaymath}
then $j$ can enter the basis and the dual objective value has strictly
increased if there are no degenerate basic solutions. 

This yields \emph{one iteration of the dual simplex method} which
operates on a dual feasible basis $B$:
\begin{enumerate}
\item If $x^*_B\geq0$, then output the optimal basis $B$. 
\item Else let $i$ be an index with $x^*(B_i)<0$. If the $i$-th row of
  the system matrix contains only nonnegative entries, the assert that
  the linear program is infeasible and that the dual linear program is
  unbounded. 
\item Let $K = \{i \colon a_i <0\}$ and let $j \in K$ be an index with 
  \begin{displaymath}
    \overline{c}_j /a_j = \max\{ \overline{c}_k /a_k \colon k \in K\}.
  \end{displaymath}
\item The index  $i$ leaves the basis and $j$ enters the basis. 
\item Update the tableau. 
\end{enumerate}

Notice that the update step strictly reduces the upper-right entry in
the tableau if $\overline{c}_j >0$. Again, by a perturbation argument,
we can achieve that such a degeneracy never occurs and the dual
simplex method terminates. 


\begin{example}[See p. 162 in~\cite{BertsimasTsitsiklis97}]
  
  \begin{displaymath}
  \begin{tabularx}{.5 \textwidth}{XXXX|X}
    1   &      1 &      0  &    0   &   0 \\ \hline 
    -1   &     -2 &    1  &    0   &    -2\\
    -1   &     0   &     0  &    1   &    -1
  \end{tabularx} 
 \quad \quad \quad  B = \{3,4\}
\end{displaymath} 

 \begin{displaymath}
  \begin{tabularx}{.5 \textwidth}{XXXX|X}
    1/2   &      0 &    1/2  &    0   &   -1 \\ \hline 
    1/2   &     1 &    -1/2  &    0   &    1\\
    -1   &     0   &     0   &    1   &    -1
  \end{tabularx} 
 \quad \quad \quad  B = \{2,4\}
\end{displaymath} 

 \begin{displaymath}
  \begin{tabularx}{.5 \textwidth}{XXXX|X}
    0   &      0 &    1/2  &    1/2   &   -3/2 \\ \hline 
    1   &     0   &     0   &    -1   &    1\\
    0   &     1  &    -1/2  &    1/2   &    1/2 
  \end{tabularx} 
 \quad \quad \quad  B = \{1,2\}
\end{displaymath} 



\end{example}

\subsection*{Exercises}

\begin{enumerate}
\item Solve the following linear program using the two-phase simplex method:
\begin{eqnarray*}
\min &  2x_1 +3x_2 +3x_3 + x_4 - 2 x_5   & \\
\text{s.t.}   &  x_1 + x_2 + 4 x_4 + x_5    = 2  &  \\
    &   x_1 + 2x_2 +  - 3 x_4 + x_5    = 2 &  \\
    &    x_1 -4 x_2 + 3 x_3 = 1 &  \\
    &   x_1,x_2,x_3,x_4 \geq 0
\end{eqnarray*}
During the first phase, let the following indices enter the basis in this order: $1,2,5$.
Use the lexicographic pivoting rule to decide which index will leave the basis in each step.
\item Solve the following simplex tableau using the dual simplex method.
\begin{center}
\begin{tabular}[ht]{|c|c|c|c|c||c|}
  \hline
   $x_1$ & $x_2$ & $x_3$ & $x_4$ & $x_5$ &  \\
  \hline
    $3$ & $2$ & $1$ & $0$ & $0$  & $0$ \\
  \hline
  \hline
    $-2$ & $2$ & $-1$ & $1$ & 0 & $-3$\\
  \hline
	$-2$ & $-1$ & $1$ & $0$ & $1$ &  $-1$ \\
  \hline
\end{tabular}
\end{center}
\item Consider the following simplex tableau:
\begin{center}%This exercise is taken from the Bertsimas book
\begin{tabular}[ht]{|c|c|c|c|c||c|}
  \hline
   $x_1$ & $x_2$ & $x_3$ & $x_4$ & $x_5$ &  \\
  \hline
    $\delta$ & $-2$ & $0$ & $0$ & $0$  & $-10$ \\
  \hline
  \hline
    $-1$ & $\eta$ & $1$ & $0$ & 0 & $4$\\
  \hline
	$\alpha$ & $-4$ & $0$ & $1$ & $0$ &  $1$ \\
  \hline
	$\gamma$ & $3$ & $0$ & $0$ & $1$ &  $\beta$ \\
  \hline
\end{tabular}
\end{center}
The current basic variables are $x_3,x_4,x_5$.
The entries $\alpha, \beta, \gamma, \delta, \eta$ in the tableau are unknown parameters.

For each one of the following statements, find some parameter values that will make the statement true.
\begin{enumerate}
\item The current solution is feasible but not optimal.
\item The current solution is optimal.
\item The optimal cost is $-\infty$.
\end{enumerate}
\end{enumerate}


\bibliographystyle{abbrv}

\bibliography{mybib,papers,books,my_publications}


%%% Local Variables: 
%%% mode: latex
%%% TeX-master: "lecture"
%%% End: 
