\documentclass[11pt]{article}

\usepackage{ppackage}
\usepackage{bbm}
\usepackage{import}






\usepackage{tikz}
\usetikzlibrary{arrows.meta,patterns}
\usetikzlibrary{ipe} % ipe compatibility library

\usepackage{../../Notes/tikzit}
\usepackage{../../Notes/utf8math}

\input{../../Notes/TIKZ/digraph.tikzstyles}



\usepackage{url}


\newcommand{\solution}{
\bigskip\noindent
	\textbf{Solution: \\}}
	


\DeclareMathOperator{\size}{size}
\DeclareMathOperator{\conv}{conv}
\newcommand{\SV}{\mathrm{SV}}
\newcommand{\bigO}{O}
\newcommand{\cut}{\mathrm{cut}}
\newcommand{\LLL}{\mathrm{LLL}}
\newcommand{\setR}{\mathbb{R}}
\newcommand{\setZ}{\mathbb{Z}}
\newcommand{\setQ}{\mathbb{Q}}
\newcommand{\setC}{\mathbb{C}}
\newcommand{\setN}{\mathbb{N}}
\newcommand{\wt}[1]{\widetilde{#1}}
\newcommand{\opt}{{\sc 0/1-opt}\xspace}
\newcommand{\aug}{{\sc 0/1-aug}\xspace}
\newcommand{\psep}{{\sc 0/1-psep}\xspace}
\newcommand{\sep}{{\sc 0/1-sep}\xspace}
\newcommand{\fopt}{{\sc 0/1-testopt\xspace} }

\newcommand{\hpp}{\mathrm{HPP}}
\newcommand{\nodes}{\mathcal{V}}
\newcommand{\vol}{\mathrm{vol}}
\newcommand{\diag}{\mathrm{diag}}
\newcommand{\arcs}{\mathcal{A}}
\newcommand{\edges}{\mathcal{E}}
\newcommand{\paths}{\mathscr{P}}
\newcommand{\cycles}{\mathcal{C}}




\newcommand{\K}{{\mathcal K}}
\newcommand{\A}{{A}}
\newcommand{\B}{{B}}
\newcommand{\T}{\mathscr{T}}
\newcommand{\eE}{\mathscr{E}}
\newcommand{\eS}{\mathscr{S}}
\newcommand{\eP}{\mathscr{P}}
\newcommand{\eM}{\mathscr{M}}



\newcommand{\transp}{^{\mathrm{T}}}

\newcommand{\smallmat}[1]{\left( \begin{smallmatrix} #1 \end{smallmatrix}\right)}

\newcommand{\mat}[1]{ \begin{pmatrix} #1 \end{pmatrix}}
\newcommand{\smat}[1]{ \big(\begin{smallmatrix} #1 \end{smallmatrix}\big)}

\newcommand{\pc}{\mathscr{P}}
\newcommand{\ob}{\mathscr{O}}
\newcommand{\odds}{\mathscr{W}}
\newcommand{\up}{\mathscr{U}}
\newcommand{\ef}{\mathscr{F}}
\newcommand{\eh}{\mathscr{H}}
\newcommand{\ev}{\mathscr{V}}
\newcommand{\ec}{\mathscr{C}}
\newcommand{\eu}{\mathscr{U}}

\newcommand{\lex}{\mathrm{lex}}

\renewcommand{\leq}{\leqslant}
\renewcommand{\geq}{\geqslant}









\newcommand{\linhull}{\mathrm{lin.hull}}
\newcommand{\affhull}{\mathrm{affine.hull}}
\newcommand{\charcone}{\mathrm{char.cone}}
\newcommand{\cone}{\mathrm{cone}}
\newcommand{\rank}{\mathrm{rank}}
\newcommand{\wb}[1]{\overline{#1}}



\usepackage{enumerate}

      
\institute{\'Ecole Polytechnique F\'ed\'erale de Lausanne}
\lecture{Discrete Optimization}
\faculty{Prof. Eisenbrand}
\term{Spring 2025}
\publishdate{April 8, 2025}
\duedate{ }
\problemset{Assignment~8}

\begin{document}
\makeheader

\begin{enumerate}[1)]



\item Let $a, e, k ∈ \setN$ be three given natural numbers.
\begin{enumerate}
\item Argue that $a^{2^k}$ can be computed using $Θ(k)$ multiplications.
\item How many bits ($Θ$-notation) does $a^{2^k}$ have?
\item Let $(e_0, \hdots, e_l)$ be the bit representation of $e$, that is, $e = \displaystyle\sum_{i= 0}^l e_i2^i$ with $e_i ∈ \{0, 1\}$ for $i= 0, \hdots, l$.
Complete the following algorithm by replacing each occurrence of three question marks (???) so
that it computes $a^e$ using $O(l)$ many arithmetic operations.
\begin{align*}
& E = 1\\
& S = a\\
&\text{For }(i=0 \text{ to }l) \\
& \quad \text{if  }(e_i == 1)\\
& \quad \quad E = E\cdot ??? \\
&\quad S = ??? \\
&\text{return }???
\end{align*}
\item Show that for given $a, e, N ∈ \setN$ one can compute $a^e (\mod N)$ in time polynomial in the binary
encoding length of $a, e$ and $N$.
\item Let $a, b, c, N ∈ \setN$ be given and suppose that $N$ is a prime number. Show that $a^{b^c} (\mod N)$ can
be computed in polynomial time in the binary encoding length of $a, b, c$ and $N$. You may use
Fermat’s little theorem: $a^N ≡ a (\mod N)$.
\end{enumerate}





\item There are n types of animals, and you want to assign them to two stables. Unfortunately,
some animals would eat other animals when left unattended. Therefore you need to assign
the animals carefully. There are m relations of the form “u eats v“, where u and v are animals.
Find an $O(n + m)$ algorithm that decides whether there is an assignment of animals to the
two stables such that no animal eats another one of the same stable, and outputs a feasible
assignment.\\
\textit{Hint: Breadth-first-search might be useful.}



\item Let $M_{2^k}$ be a matrix of order $n:= 2^k$, where $k∈\setN_{>0}$ such that it is recursively defined as follows:
$$M_{2^k} = \begin{pmatrix} M_{2^{k−1}} &  M_{2^{k−1}} \\ M_{2^{k−1}} & −M_{2^{k−1}} \end{pmatrix} $$

and $M_1 = [1]$, a $1 \times 1$ matrix. Prove that $|\det(M_n)|= n^{n/2}$, i.e. that the Hadamard bound is tight.




\item The determinant of a matrix $A∈\setR^{n×n}$ can be computed by the recursive formula
$$\det(A) = \displaystyle\sum_{j=1}^n (-1)^{1+j} a_{1j}\det(A_{1j}), $$
where $A_{1j}$ is the $(n−1)\times(n−1)$ matrix that is obtained from $A$ by deleting its first row and j-th
column. This yields the following recursive algorithm.\\
Input: $A∈\setR^{n×n}$ \\
Output: $\det(A)$\\
\begin{align*}
& \text{if } (n= 1) \\
&\text{return }a_{11}\\
&\text{else} \\
& \quad d:= 0\\
& \quad\text{for }j = 1,...,n\\
& \quad \quad d:= (−1)^{1+j} \det(A_{1j} ) + d \\
&\quad \text{return }d
\end{align*}
Let $A∈\setR^{n×n}$ and suppose that the $n^2$ components of $A$ are pairwise different.
\begin{enumerate}
\item Suppose that $B$ is a matrix that can be obtained from $A$ by deleting the first k rows and k of
the columns of A. How many (recursive) calls of the form $\det(B)$ does the algorithm create?
\item How many different submatrices can be obtained from $A$ by deleting the first k rows and some
set of k columns? Conclude that the algorithm remains exponential, even if it does not expand
repeated subcalls.
\end{enumerate}


\end{enumerate}



  

\end{document}

%%% Local Variables:
%%% mode: latex
%%% TeX-master: t
%%% End:
