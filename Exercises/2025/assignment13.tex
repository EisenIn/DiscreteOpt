\documentclass[11pt]{article}

\usepackage{ppackage}
\usepackage{bbm}
\usepackage{import}






\usepackage{tikz}
\usetikzlibrary{arrows.meta,patterns}
\usetikzlibrary{ipe} % ipe compatibility library

\usepackage{../../Notes/tikzit}
\usepackage{../../Notes/utf8math}
\usetikzlibrary{positioning, 
                quotes}

\input{../../Notes/TIKZ/digraph.tikzstyles}



\usepackage{url}


\newcommand{\solution}{
\bigskip\noindent
	\textbf{Solution: \\}}
	


\DeclareMathOperator{\size}{size}
\DeclareMathOperator{\conv}{conv}
\newcommand{\SV}{\mathrm{SV}}
\newcommand{\bigO}{O}
\newcommand{\cut}{\mathrm{cut}}
\newcommand{\LLL}{\mathrm{LLL}}
\newcommand{\setR}{\mathbb{R}}
\newcommand{\setZ}{\mathbb{Z}}
\newcommand{\setQ}{\mathbb{Q}}
\newcommand{\setC}{\mathbb{C}}
\newcommand{\setN}{\mathbb{N}}
\newcommand{\wt}[1]{\widetilde{#1}}
\newcommand{\opt}{{\sc 0/1-opt}\xspace}
\newcommand{\aug}{{\sc 0/1-aug}\xspace}
\newcommand{\psep}{{\sc 0/1-psep}\xspace}
\newcommand{\sep}{{\sc 0/1-sep}\xspace}
\newcommand{\fopt}{{\sc 0/1-testopt\xspace} }

\newcommand{\hpp}{\mathrm{HPP}}
\newcommand{\nodes}{\mathcal{V}}
\newcommand{\vol}{\mathrm{Vol}}
\newcommand{\diag}{\mathrm{diag}}
\newcommand{\arcs}{\mathcal{A}}
\newcommand{\edges}{\mathcal{E}}
\newcommand{\paths}{\mathscr{P}}
\newcommand{\cycles}{\mathcal{C}}




\newcommand{\K}{{\mathcal K}}
\newcommand{\A}{{A}}
\newcommand{\B}{{B}}
\newcommand{\T}{\mathscr{T}}
\newcommand{\eE}{\mathscr{E}}
\newcommand{\eS}{\mathscr{S}}
\newcommand{\eP}{\mathscr{P}}
\newcommand{\eM}{\mathscr{M}}



\newcommand{\transp}{^{\mathrm{T}}}

\newcommand{\smallmat}[1]{\left( \begin{smallmatrix} #1 \end{smallmatrix}\right)}

\newcommand{\mat}[1]{ \begin{pmatrix} #1 \end{pmatrix}}
\newcommand{\smat}[1]{ \big(\begin{smallmatrix} #1 \end{smallmatrix}\big)}

\newcommand{\pc}{\mathscr{P}}
\newcommand{\ob}{\mathscr{O}}
\newcommand{\odds}{\mathscr{W}}
\newcommand{\up}{\mathscr{U}}
\newcommand{\ef}{\mathscr{F}}
\newcommand{\eh}{\mathscr{H}}
\newcommand{\ev}{\mathscr{V}}
\newcommand{\ec}{\mathscr{C}}
\newcommand{\eu}{\mathscr{U}}

\newcommand{\lex}{\mathrm{lex}}

\renewcommand{\leq}{\leqslant}
\renewcommand{\geq}{\geqslant}









\newcommand{\linhull}{\mathrm{lin.hull}}
\newcommand{\affhull}{\mathrm{affine.hull}}
\newcommand{\charcone}{\mathrm{char.cone}}
\newcommand{\cone}{\mathrm{cone}}
\newcommand{\rank}{\mathrm{rank}}
\newcommand{\wb}[1]{\overline{#1}}



\usepackage{enumerate}

      
\institute{\'Ecole Polytechnique F\'ed\'erale de Lausanne}
\lecture{Discrete Optimization}
\faculty{Prof. Eisenbrand}
\term{Spring 2025}
\publishdate{May 20, 2025}
\duedate{ }
\problemset{Assignment~13}

\begin{document}
\makeheader

\begin{enumerate}[1)]

%\item A \emph{simplex}  is a polytope of the form
  %\begin{displaymath}
    %Σ = \conv\{v_0,\dots,v_n\}, 
    %\end{displaymath}
   % where $v_0,\dots,v_n ∈ ℝ^n$ are affinely independent, i.e. $v_1-v_0, \dots,v_n - v_0$ are linearly independent.

    
\item Let $A ∈ ℤ^{m ×n}$ be a matrix of rank $n$ and let $b ∈ ℤ^m$. Show the following: If $P = \{ x ∈ ℝ^n : Ax ≤ b\}$ is full-dimensional, then there exist $n+1$ vertices of $P$ that are \textit{affinely} independent. 


\item Given vectors $c∈\setR^n$ and $a∈\setR^n$, and a symmetric positive definite matrix $A∈\setR^{n×n}$, provide a formula for the ellipsoid containing the half-ball $H= \{x∈\setR^n : ∥x∥_2 ≤1, c^Tx≥0\}$.
  
  
\item Let $n ∈ ℕ$ and consider the space $ℝ^{n^2}$. An element $x ∈ℝ^{n^2}$  be interpreted as a matrix $A ∈ ℝ^{n ×n}$  in the obvious way as
  \begin{displaymath}
    A =
    \begin{pmatrix}
      x_1 & x_2 & \cdots & x_n \\
      x_{n+1} & x_{n+2} & \cdots & x_{2n} \\
      & &  \vdots \\
      x_{n^2-n+1} & x_{n^2-n+2} & \cdots &x_{n^2} \\
    \end{pmatrix}
  \end{displaymath}

  Let  $X ⊆ ℝ^{n^2}$ be the subset of $ℝ^{n^2}$ consisting of symmetric and positive semidefinite matrices. 
  \begin{enumerate}[i)]
  \item Show that $X$ is convex.
  \item Let $A ∈ ℝ^{n ×n}$, $A ∉ X$. Describe a hyperplane $a^Tx = β$ that separates $A$ from $X$. 
  \end{enumerate}
  
\item Give an example for a linear program with no maximum (in other
words, unbounded linear program) such that the corresponding integer
program is not unbounded.


\item Let $E ⊂\setR^3$ be the ellipsoid
  $E= \{(x,y,z) :x^2 + \frac{y^2}{4} + \frac{z^2} {9} ≤1\}$.  Let
  $H^+$ be the half-space $H^+ = \{(x,y,z) : x+ y+ z ≥0\}$.  Find an
  ellipsoid $E'$ such that $E'⊃E∩H^+$ and
  $\vol(E') ≤\vol(E) ⋅e^{−1/(2(3+1))}$.


\item (Bonus Question) Suppose we are given an oracle that tells us whether a
   polyhedron defined by $Ax ≤b$ is full-dimensional. Based on this oracle,
   this exercise develops a method to find an inequality $a^Tx ≤ β$ of
   $Ax ≤b$ that is satisfied by every feasible solution with
   \emph{equality} in the case where $P = \{ x ∈ ℝ^n : Ax≤b\}$ is not
   full-dimensional.


   First we split $Ax ≤ b$ into two systems $A_1x ≤ b_1$ and
   $A_2 x ≤ b_2$, where $A_1x ≤ b_1$ are the inequalities that are
   satisfied with equality by every feasible $x^* ∈ ℝ^n$. The system $A_1 x ≤b_1$ is
   called the \emph{implicit equalities} of $Ax ≤b$.
   \begin{enumerate}[i)] 
   \item If $Ax ≤b$ is feasible, then there exists a feasible solution $x^*$ such that $A_2x^* < b_2$ holds.
     \item Argue that the implicit equalities of $A_1x≤ b_1$ are  $A_1x ≤b_1$. 
   \item Suppose that $Ax ≤ b$ is not full-dimensional and that
     $A'x ≤ b'$ is full-dimensional, where $A'x ≤ b'$ stems from
     $Ax ≤ b$ by deleting one inequality $a^Tx ≤ β$. Show that
     $a^Tx ≤ β$ is an implicit equality of $Ax ≤b$.
   \item If $a^Tx ≤ β$ is an implicit equality of $Ax≤ b$, then describe a feasibility problem in $ℝ^{n-1}$ that is equivalent to the one of $Ax ≤b$. 
   \end{enumerate}

\end{enumerate}




  

\end{document}

%%% Local Variables:
%%% mode: latex
%%% TeX-master: t
%%% End:
