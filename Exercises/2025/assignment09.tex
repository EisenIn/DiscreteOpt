\documentclass[11pt]{article}

\usepackage{ppackage}
\usepackage{bbm}
\usepackage{import}






\usepackage{tikz}
\usetikzlibrary{arrows.meta,patterns}
\usetikzlibrary{ipe} % ipe compatibility library

\usepackage{../../Notes/tikzit}
\usepackage{../../Notes/utf8math}

\input{../../Notes/TIKZ/digraph.tikzstyles}



\usepackage{url}


\newcommand{\solution}{
\bigskip\noindent
	\textbf{Solution: \\}}
	


\DeclareMathOperator{\size}{size}
\DeclareMathOperator{\conv}{conv}
\newcommand{\SV}{\mathrm{SV}}
\newcommand{\bigO}{O}
\newcommand{\cut}{\mathrm{cut}}
\newcommand{\LLL}{\mathrm{LLL}}
\newcommand{\setR}{\mathbb{R}}
\newcommand{\setZ}{\mathbb{Z}}
\newcommand{\setQ}{\mathbb{Q}}
\newcommand{\setC}{\mathbb{C}}
\newcommand{\setN}{\mathbb{N}}
\newcommand{\wt}[1]{\widetilde{#1}}
\newcommand{\opt}{{\sc 0/1-opt}\xspace}
\newcommand{\aug}{{\sc 0/1-aug}\xspace}
\newcommand{\psep}{{\sc 0/1-psep}\xspace}
\newcommand{\sep}{{\sc 0/1-sep}\xspace}
\newcommand{\fopt}{{\sc 0/1-testopt\xspace} }

\newcommand{\hpp}{\mathrm{HPP}}
\newcommand{\nodes}{\mathcal{V}}
\newcommand{\vol}{\mathrm{vol}}
\newcommand{\diag}{\mathrm{diag}}
\newcommand{\arcs}{\mathcal{A}}
\newcommand{\edges}{\mathcal{E}}
\newcommand{\paths}{\mathscr{P}}
\newcommand{\cycles}{\mathcal{C}}




\newcommand{\K}{{\mathcal K}}
\newcommand{\A}{{A}}
\newcommand{\B}{{B}}
\newcommand{\T}{\mathscr{T}}
\newcommand{\eE}{\mathscr{E}}
\newcommand{\eS}{\mathscr{S}}
\newcommand{\eP}{\mathscr{P}}
\newcommand{\eM}{\mathscr{M}}



\newcommand{\transp}{^{\mathrm{T}}}

\newcommand{\smallmat}[1]{\left( \begin{smallmatrix} #1 \end{smallmatrix}\right)}

\newcommand{\mat}[1]{ \begin{pmatrix} #1 \end{pmatrix}}
\newcommand{\smat}[1]{ \big(\begin{smallmatrix} #1 \end{smallmatrix}\big)}

\newcommand{\pc}{\mathscr{P}}
\newcommand{\ob}{\mathscr{O}}
\newcommand{\odds}{\mathscr{W}}
\newcommand{\up}{\mathscr{U}}
\newcommand{\ef}{\mathscr{F}}
\newcommand{\eh}{\mathscr{H}}
\newcommand{\ev}{\mathscr{V}}
\newcommand{\ec}{\mathscr{C}}
\newcommand{\eu}{\mathscr{U}}

\newcommand{\lex}{\mathrm{lex}}

\renewcommand{\leq}{\leqslant}
\renewcommand{\geq}{\geqslant}









\newcommand{\linhull}{\mathrm{lin.hull}}
\newcommand{\affhull}{\mathrm{affine.hull}}
\newcommand{\charcone}{\mathrm{char.cone}}
\newcommand{\cone}{\mathrm{cone}}
\newcommand{\rank}{\mathrm{rank}}
\newcommand{\wb}[1]{\overline{#1}}



\usepackage{enumerate}

      
\institute{\'Ecole Polytechnique F\'ed\'erale de Lausanne}
\lecture{Discrete Optimization}
\faculty{Prof. Eisenbrand}
\term{Spring 2025}
\publishdate{April 15, 2025}
\duedate{ }
\problemset{Assignment~9}

\begin{document}
\makeheader

\begin{enumerate}[1)]

\item Let $M ∈ \setZ^{n×m}$ be totally unimodular. Prove that the following matrices are totally unimodular as well. 
\begin{enumerate}
\item $M^T$
\item $\begin{pmatrix} M & I_n \end{pmatrix}$
\item $\begin{pmatrix} M & -M \end{pmatrix}$
\item $M\cdot \begin{pmatrix} I_n− 2e_j e_j^T\end{pmatrix}$ for any $j \in [n]$. 
\end{enumerate}




\item A family $\mathcal{F}$ of subsets of a finite groundset $E$ is laminar, if for all $C ,D ∈ \mathcal{F}$ , one of the following holds:
\begin{enumerate}
\item $C ∩ D = \emptyset$ 
\item $C ⊆ D$
\item $D ⊆ C$.
\end{enumerate}
Let $F_1$ and $F_2$ be two laminar families of the same groundset $E$ and consider its union
$F_1 ∪ F_2$. Define the $|F_1 ∪ F_2| × |E |$ adjacency matrix $A$ as follows: For $F ∈ F_1 ∪ F_2$ and $e ∈ E$
we have $A_{F,e} = 1$, if $e ∈ F$ and $A_{F,e} = 0$ otherwise.\\
Show that A is totally unimodular.


\item Let $G$ be a graph and let $A$ be its node-edge incidence matrix. We have seen that if $G$ is
bipartite then $A$ is totally unimodular. Prove the converse, i.e., if $A$ is totally unimodular then $G$
is bipartite.


\item Given a graph $G=(V,E)$, the spanning tree polytope $PST(G)$ is defined as follows:
$$PST (G) = \{x∈\setR^E : x(E(U)) ≤|U|−1 \ ∀U ⊂V, x(E) = |V|−1, x≥0\}.$$
We will show that each vertex of $PST(G)$ is integral (i.e. $PST (G)$ is the convex hull of the incidence
vectors of the spanning trees of G) by an uncrossing argument. 
Given $x^*$ a vertex of $PST (G)$, let $F= \{U ⊂V : x^*(E(U)) = |U|−1\}$.
\begin{enumerate}
\item Let $A,B ∈F$, show that $A∩B,A∪B ∈F$.
\item Show that if $L$ is a maximal laminar subfamily of $F$, then $span(L) = span(F)$ (where
$span(F) = span\{χ^{E(A)},A∈F\}$, and similarly for L).
\end{enumerate}







\end{enumerate}



  

\end{document}

%%% Local Variables:
%%% mode: latex
%%% TeX-master: t
%%% End:
