\documentclass[11pt]{article}

\usepackage{ppackage}
\usepackage{bbm}
\usepackage{import}


\newcommand{\solution}{
\bigskip\noindent
	\textbf{Solution: \\}}


\institute{\'Ecole Polytechnique F\'ed\'erale de Lausanne}
\lecture{Discrete Optimization}
\faculty{Prof. Eisenbrand}
\term{Spring 2018}
\publishdate{March 1, 2018}
\duedate{\textbf{Problem 5} can be \textbf{submitted} until March 9 12:00 noon
  into the box in front of MA C1 563. }
\problemset{Assignment~2}

\begin{document}
\makeheader

%\problem
%Show that the recursion 
%$T(n) = 8 \cdot T(n/2) + \theta(n^2)$, with the initial condition $T(1)=\theta(1)$,
%has the solution $T(n) = \theta(n^3)$.

\problem
Describe an algorithm that multiplies two $n$-bit integers in time $O(n^2)$. You may assume to have a subroutine $Sum(d,e)$ which returns the sum of two $n$-bit natural numbers $d$ and $e$ in time $O(n)$.
 
\problem Suppose $a,b \in \N$ are two $n$-bit integers, where $n$ is a power of $2$. Consider the first and the last $n/2$ bits of $a$, and denote their corresponding decimal numbers with $a'$ and $a''$, respectively. Likewise decimal numbers $b'$ and $b''$ correspond to the first and the second half of the bit-representation of $b$. 
  \begin{enumerate}[i)]
  \item Show that $a = a' + a'' \cdot 2^{n/2}$ and  $b = b' + b'' \cdot 2^{n/2}$.
  \item Show that $a \cdot b = a' \cdot b' + (a' \cdot b'' + a'' \cdot b') \cdot 2^{n/2} + a'' \cdot b'' \cdot 2^n $. 
  \item Show that $(a'b''+a''b') = (a'+a'')(b'+b'') - a'\cdot b' - a'' \cdot b''$. 
  \item Design a recursive algorithm for $n$-bit integer multiplication whose running time $T(n)$  satisfies the recursion 
    \begin{displaymath}
      T(n) \leq  3 \cdot T( n/2  ) + c \cdot n,
    \end{displaymath}
where $c>1$ is some constant. 

\emph{Hint: You can assume that there is a constant $c'$ such that two $n$-bit numbers can be added and subtracted using at most $c'\, n$ basic operations. }

\item \emph{Unroll} the recursion above three times. 

\item Conclude that two $n$-bit
    numbers can be computed in $O(n^{\log_2(3)})$
    elementary bit operations. 
  \end{enumerate}
  
\problem 
The \emph{determinant} of a matrix $A \in \R^{n \times n}$ can be computed by the recursive formula 
\begin{displaymath}
  \det(A) = \sum_{j=1}^n (-1)^{1+j}a_{1j} \det(A_{1j}),
\end{displaymath}
where $A_{1j}$ is the $(n-1)×(n-1)$ matrix that is obtained from $A$ by deleting its first row and $j$-th column.  This yields the following recursive algorithm (see the lecture notes, Example 1.4). 

\begin{tabbing}
  Input: $A \in \R^{n \times n}$ \\
  Output: $\det(A)$ \\
  
  {\bf if} \= $(n=1)$ \\
           \> {\bf return} $a_{11}$ \\
  {\bf else} \\
           \> $d:=0$  \\
           \> {\bf for } \= $j=1,\dots,n$ \\
           \>            \> $d:= (-1)^{1+j}⋅ \det(A_{1j}) +d$\\
           \> {\bf return} $d$   
\end{tabbing}

Let $A \in \R^{n \times n}$  and suppose that the $n^2$ components of $A$ are pairwise different.
\begin{enumerate}[i)]
\item
Suppose that $B$ is a matrix that can be obtained from $A$ by deleting the first $k$ rows and $k$ of the columns of $A$. How many (recursive) calls of the form $\det(B)$ does the algorithm create? 

\item How many different submatrices can be obtained from $A$ by deleting the first $k$ rows and some set of $k$ columns? Conclude that the algorithm remains exponential, even if it does not expand repeated subcalls. 
\end{enumerate}
  
\problem In this exercise, you will see that matrix multiplication is in some sense not harder than matrix inversion. 

Suppose that $I(n)$ with $I(n)=\Omega(n^2)$ is a function that satisfies $I(3\,n) = O(I(n))$ and that a non-singular  $n \times n$ matrix can be inverted using $I(n)$ arithmetic operations. Show that two $n \times n $ matrices $A$ and $B$ can be multiplied using $O(I(n))$ arithmetic operations. 

\emph{Hint: Construct an upper triangular $3n \times 3n$-matrix that contains $A$ and $B$. }

\problemstar Let $M_{2^k}$ be a matrix of order $n:=2^k$, where $\ k \in \mathbb{N}_{>0}$ such that it is recursively defined as follows:
\begin{equation}
M_{2^k}=
\begin{pmatrix}
    M_{2^{k-1}} & M_{2^{k-1}} \\ 
    M_{2^{k-1}} & -M_{2^{k-1}} 
  \end{pmatrix}
\end{equation}
and $M_1=[1]$. Prove that $|\det(M_{n})|=n^{n/2}$, i.e. that the Hadamard bound is tight.
\end{document}

%%% Local Variables:
%%% mode: latex
%%% TeX-master: t
%%% End:
