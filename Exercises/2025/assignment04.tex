\documentclass[11pt]{article}

\usepackage{ppackage}
\usepackage{bbm}
\usepackage{import}






\usepackage{tikz}
\usetikzlibrary{arrows.meta,patterns}
\usetikzlibrary{ipe} % ipe compatibility library

\usepackage{../../Notes/tikzit}
\usepackage{../../Notes/utf8math}

\input{../../Notes/TIKZ/digraph.tikzstyles}



\usepackage{url}


\newcommand{\solution}{
\bigskip\noindent
	\textbf{Solution: \\}}
	


\DeclareMathOperator{\size}{size}
\DeclareMathOperator{\conv}{conv}
\newcommand{\SV}{\mathrm{SV}}
\newcommand{\bigO}{O}
\newcommand{\cut}{\mathrm{cut}}
\newcommand{\LLL}{\mathrm{LLL}}
\newcommand{\setR}{\mathbb{R}}
\newcommand{\setZ}{\mathbb{Z}}
\newcommand{\setQ}{\mathbb{Q}}
\newcommand{\setC}{\mathbb{C}}
\newcommand{\setN}{\mathbb{N}}
\newcommand{\wt}[1]{\widetilde{#1}}
\newcommand{\opt}{{\sc 0/1-opt}\xspace}
\newcommand{\aug}{{\sc 0/1-aug}\xspace}
\newcommand{\psep}{{\sc 0/1-psep}\xspace}
\newcommand{\sep}{{\sc 0/1-sep}\xspace}
\newcommand{\fopt}{{\sc 0/1-testopt\xspace} }

\newcommand{\hpp}{\mathrm{HPP}}
\newcommand{\nodes}{\mathcal{V}}
\newcommand{\vol}{\mathrm{vol}}
\newcommand{\diag}{\mathrm{diag}}
\newcommand{\arcs}{\mathcal{A}}
\newcommand{\edges}{\mathcal{E}}
\newcommand{\paths}{\mathscr{P}}
\newcommand{\cycles}{\mathcal{C}}




\newcommand{\K}{{\mathcal K}}
\newcommand{\A}{{A}}
\newcommand{\B}{{B}}
\newcommand{\T}{\mathscr{T}}
\newcommand{\eE}{\mathscr{E}}
\newcommand{\eS}{\mathscr{S}}
\newcommand{\eP}{\mathscr{P}}
\newcommand{\eM}{\mathscr{M}}



\newcommand{\transp}{^{\mathrm{T}}}

\newcommand{\smallmat}[1]{\left( \begin{smallmatrix} #1 \end{smallmatrix}\right)}

\newcommand{\mat}[1]{ \begin{pmatrix} #1 \end{pmatrix}}
\newcommand{\smat}[1]{ \big(\begin{smallmatrix} #1 \end{smallmatrix}\big)}

\newcommand{\pc}{\mathscr{P}}
\newcommand{\ob}{\mathscr{O}}
\newcommand{\odds}{\mathscr{W}}
\newcommand{\up}{\mathscr{U}}
\newcommand{\ef}{\mathscr{F}}
\newcommand{\eh}{\mathscr{H}}
\newcommand{\ev}{\mathscr{V}}
\newcommand{\ec}{\mathscr{C}}
\newcommand{\eu}{\mathscr{U}}

\newcommand{\lex}{\mathrm{lex}}

\renewcommand{\leq}{\leqslant}
\renewcommand{\geq}{\geqslant}









\newcommand{\linhull}{\mathrm{lin.hull}}
\newcommand{\affhull}{\mathrm{affine.hull}}
\newcommand{\charcone}{\mathrm{char.cone}}
\newcommand{\cone}{\mathrm{cone}}
\newcommand{\rank}{\mathrm{rank}}
\newcommand{\wb}[1]{\overline{#1}}



\usepackage{enumerate}

      
\institute{\'Ecole Polytechnique F\'ed\'erale de Lausanne}
\lecture{Discrete Optimization}
\faculty{Prof. Eisenbrand}
\term{Spring 2025}
\publishdate{March 11, 2025}
\duedate{ }
\problemset{Assignment~4}

\begin{document}
\makeheader

\begin{enumerate}[1)]
\item Let $K \subseteq \setR^n$ be a convex set. Show that $x^\ast$ is an extreme point $\iff \forall x_1 \neq x_2 \in K, x^\ast \neq \frac{1}{2}x_1 + \frac{1}{2}x_2$. 


\item Suppose that the linear program $\max \{c^Tx \colon x \in \R^n, \, Ax \leq b\}$ is non-degenerate and $B$ is an optimal basis. Show that the linear program has a unique optimal solution if and only if $\lambda_B>0$. 


\item For each of the following assertion, provide a proof or a counterexample. 
  \begin{enumerate}[i)]
  \item An index that has just left the basis $B$ in the simplex
    algorithm cannot enter in the very next iteration.
  \item An index that has just entered the basis $B$ in the simplex
    algorithm cannot leave again in the very next iteration. 
  \end{enumerate}
  
  
 \item Consider the auxiliary linear program to find an initial feasible basis in (4.10). The constraint matrix of this linear program is of the form
  \begin{displaymath}
    \begin{pmatrix}
      A & 0 \\
      -I_n & 0\\ 
      0 & -I_{m_2}\\
      0 & I_{m_2}
    \end{pmatrix},
  \end{displaymath}
  where $m_2$ is the number of rows of $A_2$. This matrix has $m+n+2\cdot m_2$ rows. Describe an initial feasible basis that corresponds to the basic feasible solution $x = 0$ and $y=0$. 

Suppose that the optimal value of the auxiliary linear program is $0$ and let $B'$   be an optimal basis found by the simplex algorithm. Prove that $B' \setminus \{m+n+1,\dots,m+n+m_2\}$ is a feasible \emph{basis} of the linear program (4.9). 
  
 \item Fill in the blanks to complete the code for the Simplex.py file which runs a simplex algorithm. 




\end{enumerate}



  

\end{document}

%%% Local Variables:
%%% mode: latex
%%% TeX-master: t
%%% End:
