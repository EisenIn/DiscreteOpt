\documentclass[11pt]{article}

\usepackage{ppackage}
\usepackage{bbm}
\usepackage{import}






\usepackage{tikz}
\usetikzlibrary{arrows.meta,patterns}
\usetikzlibrary{ipe} % ipe compatibility library

\usepackage{../../Notes/tikzit}
\usepackage{../../Notes/utf8math}

\input{../../Notes/TIKZ/digraph.tikzstyles}



\usepackage{url}


\newcommand{\solution}{
\bigskip\noindent
	\textbf{Solution: \\}}
	


\DeclareMathOperator{\size}{size}
\DeclareMathOperator{\conv}{conv}
\newcommand{\SV}{\mathrm{SV}}
\newcommand{\bigO}{O}
\newcommand{\cut}{\mathrm{cut}}
\newcommand{\LLL}{\mathrm{LLL}}
\newcommand{\setR}{\mathbb{R}}
\newcommand{\setZ}{\mathbb{Z}}
\newcommand{\setQ}{\mathbb{Q}}
\newcommand{\setC}{\mathbb{C}}
\newcommand{\setN}{\mathbb{N}}
\newcommand{\wt}[1]{\widetilde{#1}}
\newcommand{\opt}{{\sc 0/1-opt}\xspace}
\newcommand{\aug}{{\sc 0/1-aug}\xspace}
\newcommand{\psep}{{\sc 0/1-psep}\xspace}
\newcommand{\sep}{{\sc 0/1-sep}\xspace}
\newcommand{\fopt}{{\sc 0/1-testopt\xspace} }

\newcommand{\hpp}{\mathrm{HPP}}
\newcommand{\nodes}{\mathcal{V}}
\newcommand{\vol}{\mathrm{vol}}
\newcommand{\diag}{\mathrm{diag}}
\newcommand{\arcs}{\mathcal{A}}
\newcommand{\edges}{\mathcal{E}}
\newcommand{\paths}{\mathscr{P}}
\newcommand{\cycles}{\mathcal{C}}




\newcommand{\K}{{\mathcal K}}
\newcommand{\A}{{A}}
\newcommand{\B}{{B}}
\newcommand{\T}{\mathscr{T}}
\newcommand{\eE}{\mathscr{E}}
\newcommand{\eS}{\mathscr{S}}
\newcommand{\eP}{\mathscr{P}}
\newcommand{\eM}{\mathscr{M}}



\newcommand{\transp}{^{\mathrm{T}}}

\newcommand{\smallmat}[1]{\left( \begin{smallmatrix} #1 \end{smallmatrix}\right)}

\newcommand{\mat}[1]{ \begin{pmatrix} #1 \end{pmatrix}}
\newcommand{\smat}[1]{ \big(\begin{smallmatrix} #1 \end{smallmatrix}\big)}

\newcommand{\pc}{\mathscr{P}}
\newcommand{\ob}{\mathscr{O}}
\newcommand{\odds}{\mathscr{W}}
\newcommand{\up}{\mathscr{U}}
\newcommand{\ef}{\mathscr{F}}
\newcommand{\eh}{\mathscr{H}}
\newcommand{\ev}{\mathscr{V}}
\newcommand{\ec}{\mathscr{C}}
\newcommand{\eu}{\mathscr{U}}

\newcommand{\lex}{\mathrm{lex}}

\renewcommand{\leq}{\leqslant}
\renewcommand{\geq}{\geqslant}









\newcommand{\linhull}{\mathrm{lin.hull}}
\newcommand{\affhull}{\mathrm{affine.hull}}
\newcommand{\charcone}{\mathrm{char.cone}}
\newcommand{\cone}{\mathrm{cone}}
\newcommand{\rank}{\mathrm{rank}}
\newcommand{\wb}[1]{\overline{#1}}



\usepackage{enumerate}

      
\institute{\'Ecole Polytechnique F\'ed\'erale de Lausanne}
\lecture{Discrete Optimization}
\faculty{Prof. Eisenbrand}
\term{Spring 2025}
\publishdate{March 25, 2025}
\duedate{ }
\problemset{Assignment~6}

\begin{document}
\makeheader

\begin{enumerate}[1)]
\item Determine the value of the matrix game defined by 
  \begin{displaymath}
    A =
    \begin{pmatrix}
      6 & 6 \\
      7 & 4
    \end{pmatrix}
  \end{displaymath}
and determine optimal strategies for both players with
\begin{enumerate}
\item pure strategies and 
\item mixed strategies.
\end{enumerate} 
  
\item This exercise is a continuation of exercise 2) from the sheet of last week.  Here we find  the \emph{Chebychev center} of a polyhedron $P = \{x ∈ ℝ^n :Ax ≤ b\}$ with $A ∈ ℝ^{m ×n}$ and $b ∈ ℝ^m$. This is the center $z ∈ ℝ^n$  of the largest euclidean ball $B(z,R) = \{ x ∈ ℝ^n : \| x - z \|_2 ≤ R\}$ that satisfies $B(z,R) ⊆ P$.
  \begin{enumerate}[i)]
  \item Let $H = (a^T x = β) ⊆  ℝ^n$ be a hyperplane and $x^*∈ ℝ^n$. What is the \emph{euclidean distance} of $x^*$ from $H$?
  \item
    Assume now that every row of $A$ has euclidean norm $\|⋅\|_2$ equal to one.  
    Prove that the following linear program finds the Chebychev center $z$ and the radius $R ∈ ℝ_{≥0}$  of the largest ball $B(z,R) ⊆ P$:
    \begin{displaymath}
      \begin{array}[t]{c}
        \max R \\
        A z +  \mathbf{1} R ≤ b
      \end{array},
    \end{displaymath}
   and $\mathbf{1}∈ ℝ^m$ is the vector of all ones.
  \item Show that there is a subsystem $A'x ≤ b'$ of $Ax≤b$ with at most $n+1$ inequalities whose corresponding polyhedron has the same Chebychev center as $P$.
  \item Write down the dual of the linear program above. 
  \end{enumerate}

\item (Complementary slackness)

  Consider the primal/dual pair
  \begin{displaymath}
    \begin{array}[t]{c}
      \max c^Tx \\
           Ax ≤ b
    \end{array} \quad \text{ and }\quad
    \begin{array}[t]{c}
      \min b^Ty \\
      y^T A = c^T\\
      y ≥ 0      
    \end{array} 
  \end{displaymath}
  defined by $A ∈ ℝ^{m ×n}$, $b ∈ ℝ^m$ and $c ∈ ℝ^n$. 
  Let $x^* ∈ ℝ^n$ and $y^*∈ ℝ^m$ be feasible primal and dual solutions respectively.

  \medskip
  \noindent

  Show the following: $x^*$ and $y^*$ are both optimal solutions respectively if and only if $y^\ast_i > 0 \implies
A_ix^\ast = b_i$ for each $i \in [m]$.

\item Consider the linear programming problems    
  \begin{displaymath}
    \begin{array}[t]{c}
      \max c^Tx \\
      Ax ≤ b \\
      x≥ 0
    \end{array} \quad \text{ and }\quad
    \begin{array}[t]{c}
      \min b^Ty \\
      y^T A ≥ c^T\\
      y ≥ 0      
    \end{array}
  \end{displaymath}
  \begin{enumerate}[i)]
  \item Show that the minimization problem on the right is equivalent to the dual of the maximization problem.
  \item Let $x^*$ and $y^*$ be feasible solutions of the maximization and minimization problem respectively. Show that they are both optimal solutions respectively if and only if the following condition holds:% , where $a_i$ and $a^j$ denote the $i$-th row and $j$-th column of $A$ respectively:
    \begin{displaymath}
      (y^*)^T (b - Ax^*) = 0 \text{ and } (y^T A - c^T) x^* = 0. 
    \end{displaymath}
    
    \end{enumerate}

\end{enumerate}

\end{document}

%%% Local Variables:
%%% mode: latex
%%% TeX-master: t
%%% End:
