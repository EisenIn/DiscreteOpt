\documentclass[11pt]{article}

\usepackage{ppackage}
\usepackage{bbm}
\usepackage{import}






\usepackage{tikz}
\usetikzlibrary{arrows.meta,patterns}
\usetikzlibrary{ipe} % ipe compatibility library

\usepackage{../../Notes/tikzit}
\usepackage{../../Notes/utf8math}
\usetikzlibrary{positioning, 
                quotes}

\input{../../Notes/TIKZ/digraph.tikzstyles}



\usepackage{url}


\newcommand{\solution}{
\bigskip\noindent
	\textbf{Solution: \\}}
	


\DeclareMathOperator{\size}{size}
\DeclareMathOperator{\conv}{conv}
\newcommand{\SV}{\mathrm{SV}}
\newcommand{\bigO}{O}
\newcommand{\cut}{\mathrm{cut}}
\newcommand{\LLL}{\mathrm{LLL}}
\newcommand{\setR}{\mathbb{R}}
\newcommand{\setZ}{\mathbb{Z}}
\newcommand{\setQ}{\mathbb{Q}}
\newcommand{\setC}{\mathbb{C}}
\newcommand{\setN}{\mathbb{N}}
\newcommand{\wt}[1]{\widetilde{#1}}
\newcommand{\opt}{{\sc 0/1-opt}\xspace}
\newcommand{\aug}{{\sc 0/1-aug}\xspace}
\newcommand{\psep}{{\sc 0/1-psep}\xspace}
\newcommand{\sep}{{\sc 0/1-sep}\xspace}
\newcommand{\fopt}{{\sc 0/1-testopt\xspace} }

\newcommand{\hpp}{\mathrm{HPP}}
\newcommand{\nodes}{\mathcal{V}}
\newcommand{\vol}{\mathrm{vol}}
\newcommand{\diag}{\mathrm{diag}}
\newcommand{\arcs}{\mathcal{A}}
\newcommand{\edges}{\mathcal{E}}
\newcommand{\paths}{\mathscr{P}}
\newcommand{\cycles}{\mathcal{C}}




\newcommand{\K}{{\mathcal K}}
\newcommand{\A}{{A}}
\newcommand{\B}{{B}}
\newcommand{\T}{\mathscr{T}}
\newcommand{\eE}{\mathscr{E}}
\newcommand{\eS}{\mathscr{S}}
\newcommand{\eP}{\mathscr{P}}
\newcommand{\eM}{\mathscr{M}}



\newcommand{\transp}{^{\mathrm{T}}}

\newcommand{\smallmat}[1]{\left( \begin{smallmatrix} #1 \end{smallmatrix}\right)}

\newcommand{\mat}[1]{ \begin{pmatrix} #1 \end{pmatrix}}
\newcommand{\smat}[1]{ \big(\begin{smallmatrix} #1 \end{smallmatrix}\big)}

\newcommand{\pc}{\mathscr{P}}
\newcommand{\ob}{\mathscr{O}}
\newcommand{\odds}{\mathscr{W}}
\newcommand{\up}{\mathscr{U}}
\newcommand{\ef}{\mathscr{F}}
\newcommand{\eh}{\mathscr{H}}
\newcommand{\ev}{\mathscr{V}}
\newcommand{\ec}{\mathscr{C}}
\newcommand{\eu}{\mathscr{U}}

\newcommand{\lex}{\mathrm{lex}}

\renewcommand{\leq}{\leqslant}
\renewcommand{\geq}{\geqslant}









\newcommand{\linhull}{\mathrm{lin.hull}}
\newcommand{\affhull}{\mathrm{affine.hull}}
\newcommand{\charcone}{\mathrm{char.cone}}
\newcommand{\cone}{\mathrm{cone}}
\newcommand{\rank}{\mathrm{rank}}
\newcommand{\wb}[1]{\overline{#1}}



\usepackage{enumerate}

      
\institute{\'Ecole Polytechnique F\'ed\'erale de Lausanne}
\lecture{Discrete Optimization}
\faculty{Prof. Eisenbrand}
\term{Spring 2025}
\publishdate{May 13, 2025}
\duedate{ }
\problemset{Assignment~12}

\begin{document}
\makeheader

\begin{enumerate}[1)]

\item A \emph{simplex}  is a polytope of the form
  \begin{displaymath}
    Σ = \conv\{v_0,\dots,v_n\}, 
    \end{displaymath}
    where $v_0,\dots,v_n ∈ ℝ^n$ are affinely independent, i.e. $v_1-v_0, \dots,v_n - v_0$ are linearly independent.

    
\item Let $A ∈ ℤ^{m ×n}$ be a matrix of rank $n$ and let $b ∈ ℤ^m$. Show the following: If $P = \{ x ∈ ℝ^n : Ax ≤ b\}$ is full-dimensional, then
  \begin{enumerate}
  \item   there exist $n+1$ vertices of $P$ that are affinely independent. 
  \end{enumerate}
\item Let $Ax ≤b$ be a system of inequalities.  % where $A ∈ R^{m ×n}$ and $b ∈ ℝ^m$ and let $P = \{ x ∈ ℝ^n : Ax ≤b\}$.
  In inequality $a^T x ≤ β$ from $Ax ≤b$ is called an \emph{implicit equality} if
    $a^T x^*  = β$ holds for each $x^* ∈ ℝ^n$ satisfying $Ax ≤b$. %  \, \quad \text{ for all } x^* ∈ P. 

  

\item Show that the unit simplex $∆ = conv\{0,e_1,\hdots,e_n\}⊂\setR^n$ has volume $\frac{1}{n!}$.



\item Let $P= \{x∈\setR^n : Ax≤b\}$ be a full dimensional $0/1$ polytope and $c ∈\setZ^n$. We will show how we
can use the ellipsoid method to solve the optimization problem $\max c^Tx: x∈P$.
Define $z^*:= \max c^Tx: x∈P$ and $c_{max} := \max \{|c_i|: 1 ≤i≤n\}$.
\begin{enumerate}[i)]
\item Show that the ellipsoid method requires $O(n^3 \log(n)c_{max})$ iterations to decide whether $P ∩
(c^⊤x≥β) = ∅$ for some integer $β$ (i.e. find a suitable initial ellipsoid and stopping value L).
\item Show that we can use binary search to find $z^*$ with $\log(nc_{max})$ calls to the ellipsoid method.
\item Show how you can find an optimal solution $x^*$ such that $c^Tx^*= z^*$ in polynomial time.
\end{enumerate}



\item Consider the complete graph $G_n$ with 3 vertices, i.e., $G= (\{1,2,3\}, \binom{3}
{2} )$. Is the polyhedron of the linear programming relaxation of the vertex-cover integer program integral?


\item Consider the polyhedron $P = \{x ∈ \setR^3 : x_1 + 2 x_2 + 4 x_3 ≤ 4, x ≥ 0\}$. Show that this polyhedron is integral. 




\end{enumerate}




  

\end{document}

%%% Local Variables:
%%% mode: latex
%%% TeX-master: t
%%% End:
