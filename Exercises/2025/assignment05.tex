\documentclass[11pt]{article}

\usepackage{ppackage}
\usepackage{bbm}
\usepackage{import}






\usepackage{tikz}
\usetikzlibrary{arrows.meta,patterns}
\usetikzlibrary{ipe} % ipe compatibility library

\usepackage{../../Notes/tikzit}
\usepackage{../../Notes/utf8math}

\input{../../Notes/TIKZ/digraph.tikzstyles}



\usepackage{url}


\newcommand{\solution}{
\bigskip\noindent
	\textbf{Solution: \\}}
	


\DeclareMathOperator{\size}{size}
\DeclareMathOperator{\conv}{conv}
\newcommand{\SV}{\mathrm{SV}}
\newcommand{\bigO}{O}
\newcommand{\cut}{\mathrm{cut}}
\newcommand{\LLL}{\mathrm{LLL}}
\newcommand{\setR}{\mathbb{R}}
\newcommand{\setZ}{\mathbb{Z}}
\newcommand{\setQ}{\mathbb{Q}}
\newcommand{\setC}{\mathbb{C}}
\newcommand{\setN}{\mathbb{N}}
\newcommand{\wt}[1]{\widetilde{#1}}
\newcommand{\opt}{{\sc 0/1-opt}\xspace}
\newcommand{\aug}{{\sc 0/1-aug}\xspace}
\newcommand{\psep}{{\sc 0/1-psep}\xspace}
\newcommand{\sep}{{\sc 0/1-sep}\xspace}
\newcommand{\fopt}{{\sc 0/1-testopt\xspace} }

\newcommand{\hpp}{\mathrm{HPP}}
\newcommand{\nodes}{\mathcal{V}}
\newcommand{\vol}{\mathrm{vol}}
\newcommand{\diag}{\mathrm{diag}}
\newcommand{\arcs}{\mathcal{A}}
\newcommand{\edges}{\mathcal{E}}
\newcommand{\paths}{\mathscr{P}}
\newcommand{\cycles}{\mathcal{C}}




\newcommand{\K}{{\mathcal K}}
\newcommand{\A}{{A}}
\newcommand{\B}{{B}}
\newcommand{\T}{\mathscr{T}}
\newcommand{\eE}{\mathscr{E}}
\newcommand{\eS}{\mathscr{S}}
\newcommand{\eP}{\mathscr{P}}
\newcommand{\eM}{\mathscr{M}}



\newcommand{\transp}{^{\mathrm{T}}}

\newcommand{\smallmat}[1]{\left( \begin{smallmatrix} #1 \end{smallmatrix}\right)}

\newcommand{\mat}[1]{ \begin{pmatrix} #1 \end{pmatrix}}
\newcommand{\smat}[1]{ \big(\begin{smallmatrix} #1 \end{smallmatrix}\big)}

\newcommand{\pc}{\mathscr{P}}
\newcommand{\ob}{\mathscr{O}}
\newcommand{\odds}{\mathscr{W}}
\newcommand{\up}{\mathscr{U}}
\newcommand{\ef}{\mathscr{F}}
\newcommand{\eh}{\mathscr{H}}
\newcommand{\ev}{\mathscr{V}}
\newcommand{\ec}{\mathscr{C}}
\newcommand{\eu}{\mathscr{U}}

\newcommand{\lex}{\mathrm{lex}}

\renewcommand{\leq}{\leqslant}
\renewcommand{\geq}{\geqslant}









\newcommand{\linhull}{\mathrm{lin.hull}}
\newcommand{\affhull}{\mathrm{affine.hull}}
\newcommand{\charcone}{\mathrm{char.cone}}
\newcommand{\cone}{\mathrm{cone}}
\newcommand{\rank}{\mathrm{rank}}
\newcommand{\wb}[1]{\overline{#1}}



\usepackage{enumerate}

      
\institute{\'Ecole Polytechnique F\'ed\'erale de Lausanne}
\lecture{Discrete Optimization}
\faculty{Prof. Eisenbrand}
\term{Spring 2025}
\publishdate{March 18, 2025}
\duedate{ }
\problemset{Assignment~5}

\begin{document}
\makeheader

\begin{enumerate}[1)]

\item Suppose you are given an oracle algorithm, which for a given polyhedron
$$P= \{\bar{x}∈\setR^n: \bar{A}\bar{x} \leq \bar{b}\}$$
gives you a feasible solution or asserts that there is none. Show that using a single call of this oracle
one can obtain an optimum solution for the LP
$$\max\{c^T x : x∈\setR^n; Ax≤b\}$$
assuming that the LP is feasible and bounded.




\item Determine the dual program for the following linear program:
\begin{align*}
\min \quad& 3x_1 + 2x_2−3x_3 + 4x_4 \\
&2x_1−2x_2 + 3x_3 + 4x_4 ≤ 3 \\
& x2 + 3x3 + 4x4 ≥ −5 \\
&2x_1−3x_2−7x_3−4x_4 = 2 \\
&x_1 ≥0 \\
& x_4 ≤0
\end{align*}



\item Let $P= \{x ∈\setR^n : Ax ≤b\}$ be a bounded, non-empty polyhedron. Formulate a linear program that
computes the largest ball inside P.




\item Consider the following linear program:
\begin{align*}
\max \quad & x_1 + x_2 \\
\text{subject to }&2x_1 + x_2 ≤6 \\
&x_1 + 2x_2 ≤8 \\
& 3x_1 + 4x_2 ≤ 22\\
& x_1 + 5x_2 ≤ 23
\end{align*}
Show that $(4/3, 10/3)$ is an optimal solution by using weak duality.




\end{enumerate}



  

\end{document}

%%% Local Variables:
%%% mode: latex
%%% TeX-master: t
%%% End:
