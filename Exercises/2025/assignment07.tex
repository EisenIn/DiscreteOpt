\documentclass[11pt]{article}

\usepackage{ppackage}
\usepackage{bbm}
\usepackage{import}






\usepackage{tikz}
\usetikzlibrary{arrows.meta,patterns}
\usetikzlibrary{ipe} % ipe compatibility library

\usepackage{../../Notes/tikzit}
\usepackage{../../Notes/utf8math}

\input{../../Notes/TIKZ/digraph.tikzstyles}



\usepackage{url}


\newcommand{\solution}{
\bigskip\noindent
	\textbf{Solution: \\}}
	


\DeclareMathOperator{\size}{size}
\DeclareMathOperator{\conv}{conv}
\newcommand{\SV}{\mathrm{SV}}
\newcommand{\bigO}{O}
\newcommand{\cut}{\mathrm{cut}}
\newcommand{\LLL}{\mathrm{LLL}}
\newcommand{\setR}{\mathbb{R}}
\newcommand{\setZ}{\mathbb{Z}}
\newcommand{\setQ}{\mathbb{Q}}
\newcommand{\setC}{\mathbb{C}}
\newcommand{\setN}{\mathbb{N}}
\newcommand{\wt}[1]{\widetilde{#1}}
\newcommand{\opt}{{\sc 0/1-opt}\xspace}
\newcommand{\aug}{{\sc 0/1-aug}\xspace}
\newcommand{\psep}{{\sc 0/1-psep}\xspace}
\newcommand{\sep}{{\sc 0/1-sep}\xspace}
\newcommand{\fopt}{{\sc 0/1-testopt\xspace} }

\newcommand{\hpp}{\mathrm{HPP}}
\newcommand{\nodes}{\mathcal{V}}
\newcommand{\vol}{\mathrm{vol}}
\newcommand{\diag}{\mathrm{diag}}
\newcommand{\arcs}{\mathcal{A}}
\newcommand{\edges}{\mathcal{E}}
\newcommand{\paths}{\mathscr{P}}
\newcommand{\cycles}{\mathcal{C}}




\newcommand{\K}{{\mathcal K}}
\newcommand{\A}{{A}}
\newcommand{\B}{{B}}
\newcommand{\T}{\mathscr{T}}
\newcommand{\eE}{\mathscr{E}}
\newcommand{\eS}{\mathscr{S}}
\newcommand{\eP}{\mathscr{P}}
\newcommand{\eM}{\mathscr{M}}



\newcommand{\transp}{^{\mathrm{T}}}

\newcommand{\smallmat}[1]{\left( \begin{smallmatrix} #1 \end{smallmatrix}\right)}

\newcommand{\mat}[1]{ \begin{pmatrix} #1 \end{pmatrix}}
\newcommand{\smat}[1]{ \big(\begin{smallmatrix} #1 \end{smallmatrix}\big)}

\newcommand{\pc}{\mathscr{P}}
\newcommand{\ob}{\mathscr{O}}
\newcommand{\odds}{\mathscr{W}}
\newcommand{\up}{\mathscr{U}}
\newcommand{\ef}{\mathscr{F}}
\newcommand{\eh}{\mathscr{H}}
\newcommand{\ev}{\mathscr{V}}
\newcommand{\ec}{\mathscr{C}}
\newcommand{\eu}{\mathscr{U}}

\newcommand{\lex}{\mathrm{lex}}

\renewcommand{\leq}{\leqslant}
\renewcommand{\geq}{\geqslant}









\newcommand{\linhull}{\mathrm{lin.hull}}
\newcommand{\affhull}{\mathrm{affine.hull}}
\newcommand{\charcone}{\mathrm{char.cone}}
\newcommand{\cone}{\mathrm{cone}}
\newcommand{\rank}{\mathrm{rank}}
\newcommand{\wb}[1]{\overline{#1}}



\usepackage{enumerate}

      
\institute{\'Ecole Polytechnique F\'ed\'erale de Lausanne}
\lecture{Discrete Optimization}
\faculty{Prof. Eisenbrand}
\term{Spring 2025}
\publishdate{April 1, 2025}
\duedate{ }
\problemset{Assignment~7}

\begin{document}
\makeheader

\begin{enumerate}[1)]

\item Consider the following problem. We are given $B ∈\mathbb{N}$, and a set of integer points $S= \{p ∈\setZ^n :
0 ≤p_i ≤B ∀ i = 1,...,n\}$, whose points are all colored blue but one, which is red. We have an
oracle that, given vectors $l,r ∈\setR^n$, tells us whether the red point in $S$ is contained in the box
$S∩\{x ∈\setR^n : l_i ≤x_i ≤r_i ∀i = 1,...,n\}$ or not. Give an algorithm to find the red point using
$O(n \log(B))$ many oracle calls.



\item Let $P := \{x ∈\setR^n : Ax= b, x ≥0\}$ be a polyhedron and $\min\{c^Tx : x ∈P\}$ be the corresponding
primal linear program. Assume that all the coefficients of $A, b$ and $c$ are integral and bounded in
absolute value by given $B ∈\setN$, and furthermore let $L:= B^n n^{n/2}$.
\begin{enumerate}
\item Show the following: If $x_1,x_2$ are vertices of $P$ and $c^Tx_1\neq c^Tx_2$, then $|c^Tx_1−c^Tx_2|≥1/L^2$.
\item  Let $x^∗$and $y^∗$ be feasible solutions of the primal and dual linear program respectively. Conclude
the following from the above: If $|c^Tx^∗ −b^Ty^∗|<1/L^2$, then each vertex $x$ of $P$ with $c^Tx≤c^Tx^∗$ is
an optimal solution of the primal.
\end{enumerate}




\item Let $Ax≤b$ be a system of inequalities where each component of $A$ and $b$ is an integer bounded by
$B$ in absolute value. Show that $Ax≤b$ is feasible if and only if $Ax≤b ,−B^n \cdot n^{n/2}\cdot n\cdot B ≤x_i ≤ B^n\cdot n^{n/2}\cdot n\cdot B$, $∀i∈[n]$ is feasible. \\

\textit{Hint: Consider a feasible point $x^∗$ and the index sets $I= \{i: x^∗_i ≥0\}$ and $J= \{j : x^∗_j ≤0\}$. The
polyhedron defined by $Ax≤b, x_i ≥0, i∈I, x_j ≤0, j ∈J$ is feasible and has vertices. Estimate the infinity norm of a vertex.}



\item Suppose that there exists an algorithm that on input $A ∈\setZ^{m×n}$ and $b ∈\setZ^m$ decides
the feasibility of the system $Ax≤b$, in time $poly(n,m,\log B)$, where $B$ is an upper bound on each
absolute value of an entry of $A$ and $b$. 


Let the system $P = \{Ax ≤b\}$ be feasible where $P$ contains vertices. Let $c ∈\setZ^n$ such that $\max\{c^Tx : Ax ≤b\}< \infty$ and $\Vert c \Vert_\infty \leq B$. Using binary search, show that there exists a polynomial time (in $n,m$ and $\log B$) algorithm that on input $A,b,c$ determines the value of $\max\{c^Tx: Ax≤b\}$.



\end{enumerate}



  

\end{document}

%%% Local Variables:
%%% mode: latex
%%% TeX-master: t
%%% End:
