\documentclass[11pt]{article}

\usepackage{../ppackage}
\usepackage{bbm}
\usepackage{import}






\usepackage{tikz}
\usetikzlibrary{arrows.meta,patterns}
\usetikzlibrary{ipe} % ipe compatibility library

\usepackage{../../../Notes/tikzit}
\usepackage{../../../Notes/utf8math}

\input{../../../Notes/TIKZ/digraph.tikzstyles}



\usepackage{url}


\newcommand{\solution}{
\bigskip\noindent
	\textbf{Solution: \\}}
	


\DeclareMathOperator{\size}{size}
\DeclareMathOperator{\conv}{conv}
\newcommand{\SV}{\mathrm{SV}}
\newcommand{\bigO}{O}
\newcommand{\cut}{\mathrm{cut}}
\newcommand{\LLL}{\mathrm{LLL}}
\newcommand{\setR}{\mathbb{R}}
\newcommand{\setZ}{\mathbb{Z}}
\newcommand{\setQ}{\mathbb{Q}}
\newcommand{\setC}{\mathbb{C}}
\newcommand{\setN}{\mathbb{N}}
\newcommand{\wt}[1]{\widetilde{#1}}
\newcommand{\opt}{{\sc 0/1-opt}\xspace}
\newcommand{\aug}{{\sc 0/1-aug}\xspace}
\newcommand{\psep}{{\sc 0/1-psep}\xspace}
\newcommand{\sep}{{\sc 0/1-sep}\xspace}
\newcommand{\fopt}{{\sc 0/1-testopt\xspace} }

\newcommand{\hpp}{\mathrm{HPP}}
\newcommand{\nodes}{\mathcal{V}}
\newcommand{\vol}{\mathrm{vol}}
\newcommand{\diag}{\mathrm{diag}}
\newcommand{\arcs}{\mathcal{A}}
\newcommand{\edges}{\mathcal{E}}
\newcommand{\paths}{\mathscr{P}}
\newcommand{\cycles}{\mathcal{C}}




\newcommand{\K}{{\mathcal K}}
\newcommand{\A}{{A}}
\newcommand{\B}{{B}}
\newcommand{\T}{\mathscr{T}}
\newcommand{\eE}{\mathscr{E}}
\newcommand{\eS}{\mathscr{S}}
\newcommand{\eP}{\mathscr{P}}
\newcommand{\eM}{\mathscr{M}}



\newcommand{\transp}{^{\mathrm{T}}}

\newcommand{\smallmat}[1]{\left( \begin{smallmatrix} #1 \end{smallmatrix}\right)}

\newcommand{\mat}[1]{ \begin{pmatrix} #1 \end{pmatrix}}
\newcommand{\smat}[1]{ \big(\begin{smallmatrix} #1 \end{smallmatrix}\big)}

\newcommand{\pc}{\mathscr{P}}
\newcommand{\ob}{\mathscr{O}}
\newcommand{\odds}{\mathscr{W}}
\newcommand{\up}{\mathscr{U}}
\newcommand{\ef}{\mathscr{F}}
\newcommand{\eh}{\mathscr{H}}
\newcommand{\ev}{\mathscr{V}}
\newcommand{\ec}{\mathscr{C}}
\newcommand{\eu}{\mathscr{U}}

\newcommand{\lex}{\mathrm{lex}}

\renewcommand{\leq}{\leqslant}
\renewcommand{\geq}{\geqslant}









\newcommand{\linhull}{\mathrm{lin.hull}}
\newcommand{\affhull}{\mathrm{affine.hull}}
\newcommand{\charcone}{\mathrm{char.cone}}
\newcommand{\cone}{\mathrm{cone}}
\newcommand{\rank}{\mathrm{rank}}
\newcommand{\wb}[1]{\overline{#1}}



\usepackage{enumerate}

      
\institute{\'Ecole Polytechnique F\'ed\'erale de Lausanne}
\lecture{Discrete Optimization}
\faculty{Prof. Eisenbrand}
\term{Spring 2025}
\publishdate{March 25, 2025}
\duedate{ }
\problemset{Assignment~6}

\begin{document}
\makeheader

\begin{enumerate}[1)]

\item Consider the following problem. We are given $B ∈\mathbb{N}$, and a set of integer points $S= \{p ∈\setZ^n :
0 ≤p_i ≤B ∀ i = 1,...,n\}$, whose points are all colored blue but one, which is red. We have an
oracle that, given vectors $l,r ∈\setR^n$, tells us whether the red point in $S$ is contained in the box
$S∩\{x ∈\setR^n : l_i ≤x_i ≤r_i ∀i = 1,...,n\}$ or not. Give an algorithm to find the red point using
$O(n \log(B))$ many oracle calls.


\begin{solution}
We split the problem in $n$ subproblems: for $i∈\{1,...,n\}$, we want to obtain the i-th component
of the red point. This is a simple binary search problem, and we illustrate how to solve it for $i= 1$:
we first call the oracle with $l = (0,...,0),r = (B/2,B,...,B)$. Then we call the oracle with $l=
(0,...,0),r = (B/4,B,...,B)$ (if the answer is positive) or $l = (B/2,...,0),r = (3B/2,B,...,B)$
(if the answer is negative), and so on. In this way, we are guaranteed to find each component of the
red point with $O(\log(B))$ oracle calls, for a total of $O(n\log(B))$ many oracle calls.
\end{solution}


\item Let $P := \{x ∈\setR^n : Ax= b, x ≥0\}$ be a polyhedron and $\min\{c^Tx : x ∈P\}$ be the corresponding
primal linear program. Assume that all the coefficients of $A, b$ and $c$ are integral and bounded in
absolute value by given $B ∈\setN$, and furthermore let $L:= B^n n^{n/2}$.
\begin{enumerate}
\item Show the following: If $x_1,x_2$ are vertices of $P$ and $c^Tx_1\neq c^Tx_2$, then $|c^Tx_1−c^Tx_2|≥1/L^2$.
\item  Let $x^∗$and $y^∗$ be feasible solutions of the primal and dual linear program respectively. Conclude
the following from the above: If $|c^Tx^∗ −b^Ty^∗|<1/L^2$, then each vertex $x$ of $P$ with $c^Tx≤c^Tx^∗$ is
an optimal solution of the primal.
\end{enumerate}


\begin{solution}
\begin{enumerate}
\item  Let $B_1$ and $B_2$ be the sets of row indices corresponding to basic feasible solutions $x_1,x_2$, then
$x_1 = \begin{pmatrix} A \\ I \end{pmatrix}_{B_1}^{-1} \begin{pmatrix} b \\ 0 \end{pmatrix}_{B_1}$ and $x_2 = \begin{pmatrix} A \\ I \end{pmatrix}_{B_2}^{-1} \begin{pmatrix} b \\ 0 \end{pmatrix}_{B_2}$.  A basic result from linear algebra states that given an $n×n$ invertible matrix $M$ one has that $M^{−1} =\frac{1}{\det(M)}adj(M)$. If one has that the entries of $M$ are integral and bounded in absolute value by $B$ then Hadamard’s bound gives that $\det(M) ≤ B^n n^{n/2} = L$. Combining the two statements gives that the entries of $M^{−1}$ are integer multiples of
$1/ |det(M)|≥1/L$.  Applying this statement to $\begin{pmatrix} A \\ I \end{pmatrix}_{B_1}$ and since $b$ is integral we obtain that each entry of $x_1$ is an integer multiple of $1/\left|\det\left( \begin{pmatrix} A \\ I \end{pmatrix}_{B_1}\right)\right|≥1/L$. Analogously obtain the same bound for $x_2$. Thus we obtain that $|c^Tx_1−c^Tx_2|= |c^T(x_1−x_2)|$ is a positive integer multiple of some $\delta  ≥1/L^2$ since $c∈\setZ^n$.

\item  We prove the statement by contradiction. Assume that $x$ is not an optimal solution, then there
is a vertex $\bar{x}$ such that $c^T\bar{x}<c^Tx$. Then by weak duality one has $b^Ty^∗≤c^T\bar{x} <c^Tx ≤c^Tx^∗$ which further implies $|c^T\bar{x} −c^Tx|<1/L^2$ contradicting part (a).

\end{enumerate}


\end{solution}



\end{enumerate}



  

\end{document}

%%% Local Variables:
%%% mode: latex
%%% TeX-master: t
%%% End:
