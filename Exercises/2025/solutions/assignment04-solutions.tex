\documentclass[11pt]{article}

\usepackage{../ppackage}
\usepackage{bbm}
\usepackage{import}






\usepackage{tikz}
\usetikzlibrary{arrows.meta,patterns}
\usetikzlibrary{ipe} % ipe compatibility library

\usepackage{../../../Notes/tikzit}
\usepackage{../../../Notes/utf8math}

\input{../../../Notes/TIKZ/digraph.tikzstyles}



\usepackage{url}


\newcommand{\solution}{
\bigskip\noindent
	\textbf{Solution: \\}}
	


\DeclareMathOperator{\size}{size}
\DeclareMathOperator{\conv}{conv}
\newcommand{\SV}{\mathrm{SV}}
\newcommand{\bigO}{O}
\newcommand{\cut}{\mathrm{cut}}
\newcommand{\LLL}{\mathrm{LLL}}
\newcommand{\setR}{\mathbb{R}}
\newcommand{\setZ}{\mathbb{Z}}
\newcommand{\setQ}{\mathbb{Q}}
\newcommand{\setC}{\mathbb{C}}
\newcommand{\setN}{\mathbb{N}}
\newcommand{\wt}[1]{\widetilde{#1}}
\newcommand{\opt}{{\sc 0/1-opt}\xspace}
\newcommand{\aug}{{\sc 0/1-aug}\xspace}
\newcommand{\psep}{{\sc 0/1-psep}\xspace}
\newcommand{\sep}{{\sc 0/1-sep}\xspace}
\newcommand{\fopt}{{\sc 0/1-testopt\xspace} }

\newcommand{\hpp}{\mathrm{HPP}}
\newcommand{\nodes}{\mathcal{V}}
\newcommand{\vol}{\mathrm{vol}}
\newcommand{\diag}{\mathrm{diag}}
\newcommand{\arcs}{\mathcal{A}}
\newcommand{\edges}{\mathcal{E}}
\newcommand{\paths}{\mathscr{P}}
\newcommand{\cycles}{\mathcal{C}}




\newcommand{\K}{{\mathcal K}}
\newcommand{\A}{{A}}
\newcommand{\B}{{B}}
\newcommand{\T}{\mathscr{T}}
\newcommand{\eE}{\mathscr{E}}
\newcommand{\eS}{\mathscr{S}}
\newcommand{\eP}{\mathscr{P}}
\newcommand{\eM}{\mathscr{M}}



\newcommand{\transp}{^{\mathrm{T}}}

\newcommand{\smallmat}[1]{\left( \begin{smallmatrix} #1 \end{smallmatrix}\right)}

\newcommand{\mat}[1]{ \begin{pmatrix} #1 \end{pmatrix}}
\newcommand{\smat}[1]{ \big(\begin{smallmatrix} #1 \end{smallmatrix}\big)}

\newcommand{\pc}{\mathscr{P}}
\newcommand{\ob}{\mathscr{O}}
\newcommand{\odds}{\mathscr{W}}
\newcommand{\up}{\mathscr{U}}
\newcommand{\ef}{\mathscr{F}}
\newcommand{\eh}{\mathscr{H}}
\newcommand{\ev}{\mathscr{V}}
\newcommand{\ec}{\mathscr{C}}
\newcommand{\eu}{\mathscr{U}}

\newcommand{\lex}{\mathrm{lex}}

\renewcommand{\leq}{\leqslant}
\renewcommand{\geq}{\geqslant}









\newcommand{\linhull}{\mathrm{lin.hull}}
\newcommand{\affhull}{\mathrm{affine.hull}}
\newcommand{\charcone}{\mathrm{char.cone}}
\newcommand{\cone}{\mathrm{cone}}
\newcommand{\rank}{\mathrm{rank}}
\newcommand{\wb}[1]{\overline{#1}}



\usepackage{enumerate}

      
\institute{\'Ecole Polytechnique F\'ed\'erale de Lausanne}
\lecture{Discrete Optimization}
\faculty{Prof. Eisenbrand}
\term{Spring 2025}
\publishdate{March 11, 2025}
\duedate{ }
\problemset{Assignment~4}

\begin{document}
\makeheader

\begin{enumerate}[1)]
\item Let $K \subseteq \setR^n$ be a convex set. Show that $x^\ast$ is an extreme point $\iff \forall x_1 \neq x_2 \in K, x^\ast \neq \frac{1}{2}x_1 + \frac{1}{2}x_2$. 


\begin{solution}
We show $x^\ast$ is an extreme point $\implies \forall x_1 \neq x_2 \in K, x^\ast \neq \frac{1}{2}x_1 + \frac{1}{2}x_2$.
\\

Since $x^\ast$ is an extreme point, there exists an inequality $a^T x≤β$ valid for $K$ such that $\{x^\ast\}= K∩\{x∈
\setR^n : a^T x= β\}$. Assume that $x^\ast$ can be written as a midpoint of two points $x_1,x_2∈K$. We obtain
that
$$β= a^T x^\ast= a^T (1/2x_1+ 1/2x_2) = 1/2a^T x_1+ 1/2a^T x_2 <1/2β+ 1/2 β= β$$
which gives a contradiction. We used that $a^T x_1 < β$ since $a^T x_1 ≤β (a^T x ≤β$ is valid for $K$ and $x_1 ∈K$) and $a^T x_1\neq β$ ($x^\ast$ is the only point in $K$ satisfying $a^T x≤β$ with equality).\\



We show $\forall x_1 \neq x_2 \in K, x^\ast \neq \frac{1}{2}x_1 + \frac{1}{2}x_2 \implies $ $x^\ast$ is an extreme point. \\


Let $A_{x∗} $be the a (potentially infinite) set of valid inequalities for $K$ which
are tight at $x^∗$. If $rank(A_{x∗} ) < n$ then there exists a vector $d$ and $\epsilon ∈\setR>0$ such that $ad = 0$ for
any vector $a$ of $A_{x∗}$ and $x^\ast ±\epsilon d \in K$. Since $x^\ast=  1/2(x^\ast + \epsilon d) + 1/2(x^\ast
−\epsilon d)$ we have a contradiction. As this cannot hold, here exist $n$ linearly independent vectors that form the rows of some matrix $A$ such that each row is a valid inequality $a_i^T x = b_i$ for $K$ and $Ax^\ast = b$ at $x^\ast$. Let $c= 1^T A$. Then $x^\ast$ is the unique point x such that $cx= 1^T b$ and $Ax≤b$. Thus $x^\ast$ is an extreme point.

\end{solution}


\item Suppose that the linear program $\max \{c^Tx \colon x \in \R^n, \, Ax \leq b\}$ is non-degenerate and $B$ is an optimal basis. Show that the linear program has a unique optimal solution if and only if $\lambda_B>0$. 


\begin{solution}
Let $x$ denote the basic feasible solution to $B$ and $λ^TA_B = c^T$ with $λ>0$. Assume there is another optimal solution $x' \neq x$. This gives
$$0 = c^T(x−x') = \lambda^T(A_B x - A_Bx') = \lambda^T(b_B - A_Bx').$$
Since $x'$ is feasible we have $A_B x' ≤b_B$ and since $x\neq x'$ there exists an index $i$ such that
$A_ix' < b_i$. This means all components of $(b_B−A_B x')$ are non-negative and one is strictly
positive. Thus, $λ^T (b_B−A_B x') >0$, a contradiction to $c^T(x-x') = 0$.


Now, assume that $x^\ast$ is the unique optimal solution. Assume for the sake of contradiction that $λ= λ_B$ has a zero component $λ_j = 0$. Since $A_B$ is invertible, we can choose a direction $d= (−1)A^{−1}_B e_j$ where $e_j$ is the $j$th unit vector. We first show that there is a $δ >0$ such that $x^
\ast+ δd$ is feasible. Recall that the only constraints that are active/tight
at $x^\ast$ are in $B$. Hence, we can always choose $δ$ small enough such that all the constraints outside
$B$ are not violated in $x^\ast+ δd$. Now, consider the constraints in B:
$$A_B (x^\ast+ δd) = A_B x^\ast+ δA_B d ≤b_B−δe_j ≤b_B.$$
Thus, $x^\ast+ δd$ is feasible.
Lastly, we prove that $x^\ast+ δd$ is also an optimal solution:
$$c^T (x^\ast+ δd) = c^T x^\ast+ c^T δd= c^T x^\ast+ δλ^T A_B d= c^T x^\ast−δλ^T e_j= c^T x^\ast$$
where the last inequality follows because we assumed $λ_j = 0$. Hence, we have found another optimal
and feasible solution and this contradicts the uniqueness of the optimum.


\end{solution}


\item For each of the following assertion, provide a proof or a counterexample. 
  \begin{enumerate}[i)]
  \item An index that has just left the basis $B$ in the simplex
    algorithm cannot enter in the very next iteration.
  \item An index that has just entered the basis $B$ in the simplex
    algorithm cannot leave again in the very next iteration. 
  \end{enumerate}
  
  
  
  \begin{solution}
   \begin{enumerate}[i)]
  \item An index that has left the basis can enter in the very next iteration. An example is a triangle
in the plane. Maybe the simplex method does not decide to walk to the neighboring optimal
vertex in one step but makes a detour (while improving) via the other vertex. In this case, the
inequality that has just left re-enters again.
  
  
  
  \item We use the fact that Simplex always chooses a direction that augments the objective function. Let $B$ be a feasible basis and let Simplex move from $B$ to
$\tilde{B}= B \setminus \{i\} \cup \{j\}$, i.e., $i$ leaves the basis and $j$ enters it. Note that $B$ and
$\tilde{B}$ have $n−1$ common indices. Assume that $j$ leaves the basis in the next iteration. Let $d$ and
$\tilde{d}$ be the directions that Simplex chooses to move from $B$ to $\tilde{B}$ and away from $\tilde{B}$ respectively. Then, $d\cdot a_k = 0 = \tilde{d}\cdot a_k$ for all $ k ∈B\setminus\{i\}$. Since the set of vectors $a_k : k ∈B\setminus\{i\}$ are n−1 linearly independent vectors and $d, \tilde{d} ∈\setR^n$, this means that $d$ and $\tilde{d}$ are parallel. Since $j$ entered the basis, this means that $a_j^T d>0$. Since j leaves the basis in the next step, $a^T_j \tilde{d}=−1$. Thus, $d= -w \tilde{d}$ for some $w >0$. In particular, this means that the Simplex is moving in the opposite direction. Now, due to the choice of the direction in Simplex we know that $c^Td>0$ and $c^T\tilde{d}>0$. But this is impossible as $d = -w \tilde{d}$. 
  \end{enumerate}
  
  \end{solution}
  
  
\item Consider the following linear program:
\begin{align*}
\max  \quad & 6a+ 9b+ 2c \\
\text{subject to } & a+ 3b+ c ≤−4 \\
&b+ c ≤−1 \\
& 3a+ 3b−c ≤1 \\
& a ≤0  \\
& b ≤0 \\
& c ≤0. 
\end{align*}
Solve the linear program with the Simplex method and initial vertex $(−1,−1,0)^T$ . For each iteration indicate all the parameters including the optimal value and the proof of optimality.

\begin{solution}
$$\begin{matrix}  \text{iteration} & \text{basis} & \text{vertex} & \lambda & \text{direction} & \epsilon & \text{index exchange} \\
1 & \{1, 2, 6\} & (-1, -1, 0) & (6, -9, 5) & (3, -1, 0) & 1/3 & 2\rightarrow 4 \\
2 & \{1, 4, 6\} & (0, -4/3, 0) & (3, 3, -1) & (0, 1/3, -1) & 5/2 & 6 \rightarrow 3 \\
3 & \{1, 3, 4\} & (0, -1/2, -5/2) & (5/2, 1/2, 2)  \end{matrix}$$

so that the last vertex is optimal with objective value $-19/2$. 

\end{solution}



  
 \item Fill in the blanks to complete the code for the Simplex.py file which runs a simplex algorithm for a non-degenerate LP. 




\end{enumerate}



  

\end{document}

%%% Local Variables:
%%% mode: latex
%%% TeX-master: t
%%% End:
