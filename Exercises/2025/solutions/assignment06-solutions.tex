\documentclass[11pt]{article}

\usepackage{../ppackage}
\usepackage{bbm}
\usepackage{import}






\usepackage{tikz}
\usetikzlibrary{arrows.meta,patterns}
\usetikzlibrary{ipe} % ipe compatibility library

\usepackage{../../../Notes/tikzit}
\usepackage{../../../Notes/utf8math}

\input{../../../Notes/TIKZ/digraph.tikzstyles}



\usepackage{url}


\newcommand{\solution}{
\bigskip\noindent
	\textbf{Solution: \\}}
	


\DeclareMathOperator{\size}{size}
\DeclareMathOperator{\conv}{conv}
\newcommand{\SV}{\mathrm{SV}}
\newcommand{\bigO}{O}
\newcommand{\cut}{\mathrm{cut}}
\newcommand{\LLL}{\mathrm{LLL}}
\newcommand{\setR}{\mathbb{R}}
\newcommand{\setZ}{\mathbb{Z}}
\newcommand{\setQ}{\mathbb{Q}}
\newcommand{\setC}{\mathbb{C}}
\newcommand{\setN}{\mathbb{N}}
\newcommand{\wt}[1]{\widetilde{#1}}
\newcommand{\opt}{{\sc 0/1-opt}\xspace}
\newcommand{\aug}{{\sc 0/1-aug}\xspace}
\newcommand{\psep}{{\sc 0/1-psep}\xspace}
\newcommand{\sep}{{\sc 0/1-sep}\xspace}
\newcommand{\fopt}{{\sc 0/1-testopt\xspace} }

\newcommand{\hpp}{\mathrm{HPP}}
\newcommand{\nodes}{\mathcal{V}}
\newcommand{\vol}{\mathrm{vol}}
\newcommand{\diag}{\mathrm{diag}}
\newcommand{\arcs}{\mathcal{A}}
\newcommand{\edges}{\mathcal{E}}
\newcommand{\paths}{\mathscr{P}}
\newcommand{\cycles}{\mathcal{C}}




\newcommand{\K}{{\mathcal K}}
\newcommand{\A}{{A}}
\newcommand{\B}{{B}}
\newcommand{\T}{\mathscr{T}}
\newcommand{\eE}{\mathscr{E}}
\newcommand{\eS}{\mathscr{S}}
\newcommand{\eP}{\mathscr{P}}
\newcommand{\eM}{\mathscr{M}}



\newcommand{\transp}{^{\mathrm{T}}}

\newcommand{\smallmat}[1]{\left( \begin{smallmatrix} #1 \end{smallmatrix}\right)}

\newcommand{\mat}[1]{ \begin{pmatrix} #1 \end{pmatrix}}
\newcommand{\smat}[1]{ \big(\begin{smallmatrix} #1 \end{smallmatrix}\big)}

\newcommand{\pc}{\mathscr{P}}
\newcommand{\ob}{\mathscr{O}}
\newcommand{\odds}{\mathscr{W}}
\newcommand{\up}{\mathscr{U}}
\newcommand{\ef}{\mathscr{F}}
\newcommand{\eh}{\mathscr{H}}
\newcommand{\ev}{\mathscr{V}}
\newcommand{\ec}{\mathscr{C}}
\newcommand{\eu}{\mathscr{U}}

\newcommand{\lex}{\mathrm{lex}}

\renewcommand{\leq}{\leqslant}
\renewcommand{\geq}{\geqslant}









\newcommand{\linhull}{\mathrm{lin.hull}}
\newcommand{\affhull}{\mathrm{affine.hull}}
\newcommand{\charcone}{\mathrm{char.cone}}
\newcommand{\cone}{\mathrm{cone}}
\newcommand{\rank}{\mathrm{rank}}
\newcommand{\wb}[1]{\overline{#1}}



\usepackage{enumerate}

      
\institute{\'Ecole Polytechnique F\'ed\'erale de Lausanne}
\lecture{Discrete Optimization}
\faculty{Prof. Eisenbrand}
\term{Spring 2025}
\publishdate{March 25, 2025}
\duedate{ }
\problemset{Assignment~6}

\begin{document}
\makeheader

\begin{enumerate}[1)]
\item Determine the value of the matrix game defined by 
  \begin{displaymath}
    A =
    \begin{pmatrix}
      6 & 6 \\
      7 & 4
    \end{pmatrix}
  \end{displaymath}
and determine optimal strategies for both players with
\begin{enumerate}
\item pure strategies and 
\item mixed strategies.
\end{enumerate} 


\begin{solution}
\begin{enumerate}
\item Strategy for the row player is $\max_i \min_j A(i, j) = 6$ by choosing row $1$ since the minimum value for row $1$ is $6$ which is higher than the minimum value for row $2$. The strategy for the column player is then to choose either column 1 or 2 once the row player has chosen row $1$. 

\item The row player chooses $x$ such that $\sum_i x_i = 1$ and $x$ is the solution of $\max_x \min_y x^TAy$. Note that for any vector $x \geq 0$, the minimizing vector $y$ such that $y_1 + y_2 = 1$ is given by $y = (0, 1)$. Then we have that $x^TAy = 6x_1 + 4x_2$ such that the maximizing vector $x$ is $(1, 0)$. This gives the optimal mixed strategy which is actually a pure strategy.  
\end{enumerate}
\end{solution}

  
\item This exercise is a continuation of exercise 2) from the sheet of last week.  Here we find  the \emph{Chebychev center} of a polyhedron $P = \{x ∈ ℝ^n :Ax ≤ b\}$ with $A ∈ ℝ^{m ×n}$ and $b ∈ ℝ^m$. This is the center $z ∈ ℝ^n$  of the largest euclidean ball $B(z,R) = \{ x ∈ ℝ^n : \| x - z \|_2 ≤ R\}$ that satisfies $B(z,R) ⊆ P$.
  \begin{enumerate}[i)]
  \item Let $H = (a^T x = β) ⊆  ℝ^n$ be a hyperplane and $x^*∈ ℝ^n$. What is the \emph{euclidean distance} of $x^*$ from $H$?
  \item
    Assume now that every row of $A$ has euclidean norm $\|⋅\|_2$ equal to one.  
    Prove that the following linear program finds the Chebychev center $z$ and the radius $R ∈ ℝ_{≥0}$  of the largest ball $B(z,R) ⊆ P$:
    \begin{displaymath}
      \begin{array}[t]{c}
        \max R \\
        A z +  \mathbf{1} R ≤ b
      \end{array},
    \end{displaymath}
   and $\mathbf{1}∈ ℝ^m$ is the vector of all ones.
  \item Show that there is a subsystem $A'x ≤ b'$ of $Ax≤b$ with at most $n+1$ inequalities whose corresponding polyhedron has the same Chebychev center as $P$.
  \item Write down the dual of the linear program above. 
  \end{enumerate}
  
  
  \begin{solution}
  \begin{enumerate}[i)]
\item The distance of $x^\ast$ to $H$ is the distance of $x^\ast$ to the orthogonal projection onto $H$. This is given by $\Vert x^\ast - proj_H(x^\ast)\Vert = \frac{|\beta - \langle x^\ast, \alpha\rangle|}{\Vert \alpha\Vert}$. 

\item Let $z^\ast$ be the Chebyshev center and $R^\ast$ be the corresponding radius. The ball $B(z^\ast, R^\ast) = \{x \in \setR^n: \Vert x- z^\ast \Vert \leq R^\ast \} = \{z^\ast + x : \Vert x \Vert \leq R^\ast\}.$ Then the Chebyshev center is defined by 
\begin{align*}
\max_{z^\ast, R^\ast} R^\ast &\\
\text{s.t. } & B(z^\ast, R^\ast) \subseteq P.
\end{align*}

This. is equivalent to 
\begin{align*}
\max_{z^\ast, R^\ast}  R^\ast &\\
\text{s.t. } & A(z^\ast + x) \leq b \quad \quad \text{ for every } x \in \setR^n \text{ s.t. } \Vert x\Vert \leq R^\ast.
\end{align*}

Note that for any row $i$, $a_i^T(z^\ast + x) \leq a_i^T(z^\ast + a_i / \Vert a_i \Vert R^\ast)$ since $\Vert x \Vert \leq R^\ast$ and $a_i^T x \leq \Vert x\Vert \frac{a_i^Ta_i}{\Vert a_i \Vert}$ (this is just the projection from part (i)). Then, the Chebyshev center is defined by 
\begin{align*}
\max_{z^\ast, R^\ast} & R^\ast \\
\text{s.t. } &  a_i^T(z^\ast + a_i / \Vert a_i \Vert R^\ast) \leq b_i \quad i \in [m]
\end{align*} 


since if the above constraints hold, then the constraints over all $x$ with $\Vert x\Vert \leq R^\ast$ also hold. Then this LP is equivalent to 
\begin{align*}
\max_{z^\ast, R^\ast} & R^\ast \\
\text{s.t. } & A^Tz^\ast + 1 R^\ast \leq b
\end{align*}

since $\Vert a_i \Vert = 1$ for every $i$. 

\item The vector $(z^\ast, R^\ast) \in \setR^{n+1}$ is the maximizer of the LP given in part (ii). Then there exist a set of at most $n+1$ tight inequalities of the LP so that the subsystem of these inequalities uniquely define the optimizer $(z^\ast, R^\ast)$. Then the polyhedron for this subsystem also has $(z^\ast, R^\ast)$ as an optimizer such that this system defines the same Chebyshev center. 
  
  \end{enumerate}
  \end{solution}

\item (Complementary slackness)

  Consider the primal/dual pair
  \begin{displaymath}
    \begin{array}[t]{c}
      \max c^Tx \\
           Ax ≤ b
    \end{array} \quad \text{ and }\quad
    \begin{array}[t]{c}
      \min b^Ty \\
      y^T A = c^T\\
      y ≥ 0      
    \end{array} 
  \end{displaymath}
  defined by $A ∈ ℝ^{m ×n}$, $b ∈ ℝ^m$ and $c ∈ ℝ^n$. 
  Let $x^* ∈ ℝ^n$ and $y^*∈ ℝ^m$ be feasible primal and dual solutions respectively.

  \medskip
  \noindent

  Show the following: $x^*$ and $y^*$ are both optimal solutions respectively if and only if $(y^\ast)i > 0 \implies
A_ix^\ast = b_i$ for each $i \in [m]$.



\begin{solution}
First we show $\Leftarrow$. \\

By the assumption, we have that 
\begin{align*}
{y^\ast}^T b &= \displaystyle\sum_{j = 1}^m y^\ast_j b_j \\
& = \displaystyle\sum_{j = 1}^m y^\ast_j(A_j x^\ast) \\
& =\displaystyle\sum_{j = 1}^m  y^\ast_j \left( \displaystyle\sum_{i = 1}^n A_{ji} x^\ast_i \right) \\
&= \displaystyle\sum_{i = 1}^n  x^\ast_i \left( \displaystyle\sum_{j = 1}^m A_{ji} y^\ast_j \right) \\
& = {x^\ast}^T c \\
& = c^T x^\ast
\end{align*} 
so that $x^\ast, y^\ast$ achieve the same primal/dual objective value and are therefore optimal by strong duality. For the second equality, we have used that whenever the summand is nonzero, $A_j x^\ast = b_j$. \\

We then show $\Rightarrow$. \\

We have that as $x^\ast, y^\ast$ are optimal, $c^Tx^\ast = b^Ty^\ast$ by strong duality. Then 
\begin{align*}
c^Tx^\ast & = \displaystyle\sum_{i =1}^n x^\ast_i c_i \\
& =  \displaystyle\sum_{i =1}^n x^\ast_i \left( \displaystyle\sum_{j =1}^m A_{ji}y^\ast_j \right) \\
& = \displaystyle\sum_{j =1}^m y^\ast_j \left( \displaystyle\sum_{i =1}^n A_{ji}x^\ast_i \right) \\
& = \displaystyle\sum_{j =1}^m y^\ast_j (A_j x^\ast) \\
& \leq \displaystyle\sum_{j =1}^m y^\ast_j  b_j \\
& = {y^\ast}^T b
\end{align*}
where we have used that $y^\ast_j \geq 0$ for every $j$ in the last inequality. Then as the above inequalities must hold with equality everywhere since $c^Tx^\ast = b^Ty^\ast$, it must be that if $y^\ast_j >0$ then $A_j x^\ast = b_j$.

\end{solution}

\item Consider the linear programming problems    
  \begin{displaymath}
    \begin{array}[t]{c}
      \max c^Tx \\
      Ax ≤ b \\
      x≥ 0
    \end{array} \quad \text{ and }\quad
    \begin{array}[t]{c}
      \min b^Ty \\
      y^T A ≥ c^T\\
      y ≥ 0      
    \end{array}
  \end{displaymath}
  \begin{enumerate}[i)]
  \item Show that the minimization problem on the right is equivalent to the dual of the maximization problem.
  \item Let $x^*$ and $y^*$ be feasible solutions of the maximization and minimization problem respectively. Show that they are both optimal solutions respectively if and only if the following condition holds:% , where $a_i$ and $a^j$ denote the $i$-th row and $j$-th column of $A$ respectively:
    \begin{displaymath}
      (y^*)^T (b - Ax^*) = 0 \text{ and } (y^T A - c^T) x^* = 0. 
    \end{displaymath}
    
    \end{enumerate}

  \begin{solution}
    \begin{enumerate}[i)]
      \item We rewrite the maximization problem as:
      \begin{displaymath}
        \begin{array}[t]{c}
          \max c^Tx \\
          \tilde{A}x ≤ \tilde{b}\\
        \end{array} 
      \end{displaymath}
      where $\tilde{A} = \begin{pmatrix}
        A \\
        -I
      \end{pmatrix}$ and $\tilde{b} = \begin{pmatrix}
        b \\
        \mathbf{0}
      \end{pmatrix}$. Then the dual of this maximization problem is
      \begin{displaymath}
        \begin{array}[t]{c}
          \min \tilde{b}^T \tilde{y} \\
          \tilde{y}^T \tilde{A} = c^T\\
          \tilde{y} ≥ 0      
        \end{array}
      \end{displaymath}
      Let $\tilde{y} = \begin{pmatrix}
        y \\
        y^\prime
      \end{pmatrix}$ then the objective becomes $\tilde{b}^T \tilde{y} = b^Ty$ and the constraint becomes $\tilde{y}^T \tilde{A} = y^T A - (y^\prime)^T = c^T, y,y^\prime \geq 0 \Leftrightarrow y^T A = c^T, y\geq 0$. 
      Therefore the dual problem is equivalent to
      \begin{displaymath}
        \begin{array}[t]{c}
          \min b^Ty \\
          y^T A ≥ c^T\\
          y ≥ 0      
        \end{array}
      \end{displaymath}
      which is the minimization problem on the right.
      \item By weak duality we have 
      \[ c^T x^* \leq (y^*)^T A x^* \leq (y^*)^T b. \]
      Both $x^*$ and $y^*$ are optimal solutions if and only if both of inequalities above are equalities, which is equivalent to
      \[ (y^T A - c^T) x^* = 0 \text{ and } (y^*)^T (b - Ax^*) = 0. \]
    \end{enumerate}
  \end{solution}

\end{enumerate}

\end{document}

%%% Local Variables:
%%% mode: latex
%%% TeX-master: t
%%% End:
