\documentclass[11pt]{article}

\usepackage{../ppackage}
\usepackage{bbm}
\usepackage{import}






\usepackage{tikz}
\usetikzlibrary{arrows.meta,patterns}
\usetikzlibrary{ipe} % ipe compatibility library

\usepackage{../../../Notes/tikzit}
\usepackage{../../../Notes/utf8math}

\input{../../../Notes/TIKZ/digraph.tikzstyles}



\usepackage{url}


\newcommand{\solution}{
\bigskip\noindent
	\textbf{Solution: \\}}
	


\DeclareMathOperator{\size}{size}
\DeclareMathOperator{\conv}{conv}
\newcommand{\SV}{\mathrm{SV}}
\newcommand{\bigO}{O}
\newcommand{\cut}{\mathrm{cut}}
\newcommand{\LLL}{\mathrm{LLL}}
\newcommand{\setR}{\mathbb{R}}
\newcommand{\setZ}{\mathbb{Z}}
\newcommand{\setQ}{\mathbb{Q}}
\newcommand{\setC}{\mathbb{C}}
\newcommand{\setN}{\mathbb{N}}
\newcommand{\wt}[1]{\widetilde{#1}}
\newcommand{\opt}{{\sc 0/1-opt}\xspace}
\newcommand{\aug}{{\sc 0/1-aug}\xspace}
\newcommand{\psep}{{\sc 0/1-psep}\xspace}
\newcommand{\sep}{{\sc 0/1-sep}\xspace}
\newcommand{\fopt}{{\sc 0/1-testopt\xspace} }

\newcommand{\hpp}{\mathrm{HPP}}
\newcommand{\nodes}{\mathcal{V}}
\newcommand{\vol}{\mathrm{vol}}
\newcommand{\diag}{\mathrm{diag}}
\newcommand{\arcs}{\mathcal{A}}
\newcommand{\edges}{\mathcal{E}}
\newcommand{\paths}{\mathscr{P}}
\newcommand{\cycles}{\mathcal{C}}




\newcommand{\K}{{\mathcal K}}
\newcommand{\A}{{A}}
\newcommand{\B}{{B}}
\newcommand{\T}{\mathscr{T}}
\newcommand{\eE}{\mathscr{E}}
\newcommand{\eS}{\mathscr{S}}
\newcommand{\eP}{\mathscr{P}}
\newcommand{\eM}{\mathscr{M}}



\newcommand{\transp}{^{\mathrm{T}}}

\newcommand{\smallmat}[1]{\left( \begin{smallmatrix} #1 \end{smallmatrix}\right)}

\newcommand{\mat}[1]{ \begin{pmatrix} #1 \end{pmatrix}}
\newcommand{\smat}[1]{ \big(\begin{smallmatrix} #1 \end{smallmatrix}\big)}

\newcommand{\pc}{\mathscr{P}}
\newcommand{\ob}{\mathscr{O}}
\newcommand{\odds}{\mathscr{W}}
\newcommand{\up}{\mathscr{U}}
\newcommand{\ef}{\mathscr{F}}
\newcommand{\eh}{\mathscr{H}}
\newcommand{\ev}{\mathscr{V}}
\newcommand{\ec}{\mathscr{C}}
\newcommand{\eu}{\mathscr{U}}

\newcommand{\lex}{\mathrm{lex}}

\renewcommand{\leq}{\leqslant}
\renewcommand{\geq}{\geqslant}









\newcommand{\linhull}{\mathrm{lin.hull}}
\newcommand{\affhull}{\mathrm{affine.hull}}
\newcommand{\charcone}{\mathrm{char.cone}}
\newcommand{\cone}{\mathrm{cone}}
\newcommand{\rank}{\mathrm{rank}}
\newcommand{\wb}[1]{\overline{#1}}



\usepackage{enumerate}

      
\institute{\'Ecole Polytechnique F\'ed\'erale de Lausanne}
\lecture{Discrete Optimization}
\faculty{Prof. Eisenbrand}
\term{Spring 2025}
\publishdate{March 4, 2025}
\duedate{ }
\problemset{Assignment~3}

\begin{document}
\makeheader

\begin{enumerate}[1)]

\item Using Theorem 3.11, prove the following variant of Farkas' lemma:
  Let $A\in\setR^{m\times n}$ be a matrix and $b\in\setR^m$ be a vector.
  The system $Ax \leq b$, $x\in\setR^n$ has a solution if and only if
  for all $\lambda\in\setR^m_{\geq0}$ with $\lambda^T A = 0$ one has $\lambda^T b \geq 0$.
  
  
  \item Provide an example of a convex and closed  set $K\subseteq\setR^2$ and a
  linear objective function $c^Tx$ such that $\inf\{c^Tx \colon
  x\in K\}>-\infty$ but there does not exist an $x^* \in K$ with $c^Tx^* \leq
  c^Tx$ for all $x \in K$. 
  
  \item Consider the vectors
$$ x_1 = \left(\begin{array}{c} 3 \\ 1 \\ 2\end{array}\right), x_2 = \left(\begin{array}{c} 1 \\ 2 \\ 5 \end{array}\right), x_3 = \left(\begin{array}{c} 2 \\ 0 \\ 1 \end{array}\right), x_4 = \left(\begin{array}{c}  2 \\ 4 \\ 3 \end{array}\right), x_5 = \left(\begin{array}{c}  1 \\ 1 \\ 1 \end{array}\right). $$

The vector 
$$ v= x_1 + 3 x_2 + 2 x_3 + x_4 + 3 x_5 =  \left(\begin{array}{c}  15\\ 14 \\ 25 \end{array}\right)$$
is a conic combination of the $x_i$.

Write $v$ as a conic combination using only three vectors of the $x_i$.

\emph{Hint: Recall the proof of Carath\'eodory's theorem}

\item In this exercise, assume that a linear program $\max\{c^Tx \mid
  Ax\leq b \}$ can be solved in constant time $O(1)$. Suppose that $P(A,b)$
  has vertices and that the linear program is bounded. Show how to
  compute an optimal \emph{vertex} solution of the linear
  program in polynomial time in $n$ and $m$ where $A \in \mathbb{R}^{m \times n}$. 
  
  \item Let $A \in \setR^{n\times n}$ be a non-singular matrix and let
  $a_1,\ldots,a_n\in \setR^n$ be the columns of $A$.  Show that
  $\cone(\{a_1,\ldots,a_n\})$ is the polyhedron $P = \{ y \in \setR^n \colon
  A^{-1} y\geq0\}$. \label{conv:item:3} Show that $\cone(\{a_1,\ldots,a_k\})$ for
  $k\leq n$ is the set $P_k = \{y \in \setR^n \colon
  a_i^{-1} x\geq0, i=1,\ldots,k, \, a_i^{-1}x = 0, i=k+1,\ldots,n\}$, where
  $a_i^{-1}$ denotes the $i$-th row of $A^{-1}$. 
  
\item Prove that for a finite set $X\subseteq\setR^n$ the conic hull $\cone(X)$ is closed
  and convex. \label{conv:item:2} 
  
  \emph{Hint: Use Carath\'eodory's theorem and exercise~\ref{conv:item:3}.}


\end{enumerate}



  

\end{document}

%%% Local Variables:
%%% mode: latex
%%% TeX-master: t
%%% End:
