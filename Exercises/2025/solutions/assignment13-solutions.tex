\documentclass[11pt]{article}

\usepackage{../ppackage}
\usepackage{bbm}
\usepackage{import}






\usepackage{tikz}
\usetikzlibrary{arrows.meta,patterns}
\usetikzlibrary{ipe} % ipe compatibility library

\usepackage{../../../Notes/tikzit}
\usepackage{../../../Notes/utf8math}
\usetikzlibrary{positioning, 
                quotes}

\input{../../../Notes/TIKZ/digraph.tikzstyles}



\usepackage{url}


\newcommand{\solution}{
\bigskip\noindent
	\textbf{Solution: \\}}
	


\DeclareMathOperator{\size}{size}
\DeclareMathOperator{\conv}{conv}
\newcommand{\SV}{\mathrm{SV}}
\newcommand{\bigO}{O}
\newcommand{\cut}{\mathrm{cut}}
\newcommand{\LLL}{\mathrm{LLL}}
\newcommand{\setR}{\mathbb{R}}
\newcommand{\setZ}{\mathbb{Z}}
\newcommand{\setQ}{\mathbb{Q}}
\newcommand{\setC}{\mathbb{C}}
\newcommand{\setN}{\mathbb{N}}
\newcommand{\wt}[1]{\widetilde{#1}}
\newcommand{\opt}{{\sc 0/1-opt}\xspace}
\newcommand{\aug}{{\sc 0/1-aug}\xspace}
\newcommand{\psep}{{\sc 0/1-psep}\xspace}
\newcommand{\sep}{{\sc 0/1-sep}\xspace}
\newcommand{\fopt}{{\sc 0/1-testopt\xspace} }

\newcommand{\hpp}{\mathrm{HPP}}
\newcommand{\nodes}{\mathcal{V}}
\newcommand{\vol}{\mathrm{vol}}
\newcommand{\diag}{\mathrm{diag}}
\newcommand{\arcs}{\mathcal{A}}
\newcommand{\edges}{\mathcal{E}}
\newcommand{\paths}{\mathscr{P}}
\newcommand{\cycles}{\mathcal{C}}




\newcommand{\K}{{\mathcal K}}
\newcommand{\A}{{A}}
\newcommand{\B}{{B}}
\newcommand{\T}{\mathscr{T}}
\newcommand{\eE}{\mathscr{E}}
\newcommand{\eS}{\mathscr{S}}
\newcommand{\eP}{\mathscr{P}}
\newcommand{\eM}{\mathscr{M}}



\newcommand{\transp}{^{\mathrm{T}}}

\newcommand{\smallmat}[1]{\left( \begin{smallmatrix} #1 \end{smallmatrix}\right)}

\newcommand{\mat}[1]{ \begin{pmatrix} #1 \end{pmatrix}}
\newcommand{\smat}[1]{ \big(\begin{smallmatrix} #1 \end{smallmatrix}\big)}

\newcommand{\pc}{\mathscr{P}}
\newcommand{\ob}{\mathscr{O}}
\newcommand{\odds}{\mathscr{W}}
\newcommand{\up}{\mathscr{U}}
\newcommand{\ef}{\mathscr{F}}
\newcommand{\eh}{\mathscr{H}}
\newcommand{\ev}{\mathscr{V}}
\newcommand{\ec}{\mathscr{C}}
\newcommand{\eu}{\mathscr{U}}

\newcommand{\lex}{\mathrm{lex}}

\renewcommand{\leq}{\leqslant}
\renewcommand{\geq}{\geqslant}









\newcommand{\linhull}{\mathrm{lin.hull}}
\newcommand{\affhull}{\mathrm{affine.hull}}
\newcommand{\charcone}{\mathrm{char.cone}}
\newcommand{\cone}{\mathrm{cone}}
\newcommand{\rank}{\mathrm{rank}}
\newcommand{\wb}[1]{\overline{#1}}



\usepackage{enumerate}

      
\institute{\'Ecole Polytechnique F\'ed\'erale de Lausanne}
\lecture{Discrete Optimization}
\faculty{Prof. Eisenbrand}
\term{Spring 2025}
\publishdate{May 20, 2025}
\duedate{ }
\problemset{Assignment~13}

\begin{document}
\makeheader

\begin{enumerate}[1)]

%\item A \emph{simplex}  is a polytope of the form
  %\begin{displaymath}
    %Σ = \conv\{v_0,\dots,v_n\}, 
    %\end{displaymath}
   % where $v_0,\dots,v_n ∈ ℝ^n$ are affinely independent, i.e. $v_1-v_0, \dots,v_n - v_0$ are linearly independent.

    
\item Let $A ∈ ℤ^{m ×n}$ be a matrix of rank $n$ and let $b ∈ ℤ^m$. Show the following: If $P = \{ x ∈ ℝ^n : Ax ≤ b\}$ is full-dimensional, then there exist $n+1$ vertices of $P$ that are \textit{affinely} independent. 

\begin{solution}
Note that P is a polytope and thus the convex hull of its vertices. We will consecutively choose
vertices of P while ensuring that each newly selected vertex is affinely independent to the previous
ones. Initially, we can choose the first vertex arbitrarily, as every single point is affinely independent.
Now, assume that we have already chosen vertices $v_1,...,v_k$. Out of the remaining vertices we pick
as $v_{k+1}$ an arbitrary vertex of $P$ such that $v_1,...,v_k,v_{k+1}$ is affinely independent. We continue this
process until $k$ is such that all the remaining vertices are affinely dependent with $v_1,...,v_k$. This
means that all the vertices and thus their convex hull P are contained in the $k−1$-dimensional affine
subspace defined by $v_1,...,v_k$. Since P is full-dimensional, this can only happen for $k= n+ 1$. At
that point we are done.
\end{solution}



\item Given vectors $c∈\setR^n$ and $a∈\setR^n$, and a symmetric positive definite matrix $A∈\setR^{n×n}$, provide a formula for the ellipsoid containing the half-ball $H= \{x∈\setR^n : ∥x∥_2≤1, c^Tx≥0\}$.
  
  
 \begin{solution}
For convenience, we use a different representation of the ellipsoid than before. Here, we define it as
$$E(A,a) = \{x∈\setR^n :(x−a)^T A^{-1}(x−a) ≤1\}$$
where $A∈\setR^{n×n}$ is a positive definite (PD) matrix, and $a∈\setR^n$ is the center. We remark that for every PD matrix
$A$ there exists a unique PD matrix, denote it with $A^{1/2}$, such that $A= A^{1/2}A^{1/2}$. Thus,
$E(A,a) = \{x∈\setR^n :(x−a)^T {(A^{1/2})}^{−1}
{(A^{1/2})}^{−1}(x−a) ≤1\}$.

Denote with $e_1 ∈\setR^n$ the unit vector with the first coordinate equal to $1$, and with $I_n$ the $n×n$
identity matrix. When rewriting the ellipsoid, we
obtain that the half-ball $\{x∈\setR^n :∥x∥≤1, e^T1 x≥0\}$ is contained in the ellipsoid
$E(A(e_1),a(e_1))$, where $$A(e1) = \frac{n^2}{n^2 -1}\left(I_n - \frac{2}{n+1}e_1e_1^T\right) \quad \text{ and } a(e_1) = \frac{1}{n+1}e_1.$$

Without loss of generality we could assume that $∥c∥= 1$. Similarly to the proof of the Half-ball
lemma in the lecture notes, one can verify that $e_1$ can be replaced with $c$ in the above formulas.
I.e., the ellipsoid $E(A(c),a(c))$ contains the half-ball H given in the problem statement.


\item Let $n ∈ ℕ$ and consider the space $ℝ^{n^2}$. An element $x ∈ℝ^{n^2}$  be interpreted as a matrix $A ∈ ℝ^{n ×n}$  in the obvious way as
  \begin{displaymath}
    A =
    \begin{pmatrix}
      x_1 & x_2 & \cdots & x_n \\
      x_{n+1} & x_{n+2} & \cdots & x_{2n} \\
      & &  \vdots \\
      x_{n^2-n+1} & x_{n^2-n+2} & \cdots &x_{n^2} \\
    \end{pmatrix}
  \end{displaymath}

  Let  $X ⊆ ℝ^{n^2}$ be the subset of $ℝ^{n^2}$ consisting of symmetric and positive semidefinite matrices. 
  \begin{enumerate}[i)]
  \item Show that $X$ is convex.
  \item Let $A ∈ ℝ^{n ×n}$, $A ∉ X$. Describe a hyperplane $a^Tx = β$ that separates $A$ from $X$. 
  \end{enumerate}



\begin{solution}
\begin{enumerate}[i)]
\item Let $x, y \in X$ and let their matrix representations be $A_x, A_y$. Then for any $\lambda \in [0,1]$, take $z = \lambda x + (1-\lambda) y$. Then $A_z = \lambda A_x + (1-\lambda)A_y$ and for any indeces $i,j$, $$A_z[ij] = \lambda A_x[ij] + (1-\lambda) A_y[ij] =\lambda A_x[ji] + (1-\lambda) A_y[ji] = A_z[ji]$$ 
where the second equality follows from symmetry of $A_x$ and $A_y$. Hence, $A_z$ is symmetric. Moreover, consider any vector $v \in \setR^n$ and 
\end{enumerate}
\end{solution}

 
 \end{solution}



\item Give an example for a linear program with no maximum (in other
words, unbounded linear program) such that the corresponding integer
program is not unbounded.
\begin{solution}
An example is to take the feasible region as a line through the origin that does not
contain any integer point except for the origin (for example the line
spanned by $(1,√2)$). Then let the objective maximize the x-coordinate of a point on
this line. As the line is unbounded, so is the LP but the integer program only contains the point $(0,0)$ which has bounded objective value.
\end{solution}


\item Let $E ⊂\setR^3$ be the ellipsoid $E= \{(x,y,z) :x^2 + \frac{y^2}{4} + \frac{z^2}
{9} ≤1\}$.
Let $H^+$ be the half-space $H^+ = \{(x,y,z) : x+ y+ z ≥0\}$.
Find an ellipsoid $E'$ such that $E'⊃E∩H^+$ and $Vol(E') ≤Vol(E)e^{−1/(2(3+1))}$.

\begin{solution}
The linear transformation $T(x,y,z) = (x,y/2, z/3)$ takes the ellipsoid $E$ to the unit ball $B$. Notice that $T$ takes $H^+$ to the half-space
$G^+ = \{x+2y+3z ≥0\}$. Now we find a linear transformation $R$ that will
rotate the space about the origin such that $B$ will remain fixed but $G^+$
will go to the half space $\{x ≥0\}$. One way to do this is to take the vector $u=1/√14 (1,2,3)$ and complete it to an
orthonormal basis: take also the vector $v=1/√5 (−2,1,0)$ and the vector
$w$ perpendicular to both $u$ and $v$ (you need to find it). Then consider
the linear transformation $R$ that takes $u,v,w$ to $(1,0,0),(0,1,0)$, and
$(0,0,1)$. This is the inverse of the matrix whose columns are $u,v,w$.
The linear transformation $R(T(·))$ takes $E∩H^+$ to the half ball $B∩\{x≥
0\}$. By the half-ball theorem, then the ellipsoid
$F= \{\frac{16}{9}(x−\frac{1}{4})^2 + \frac{8}{9} y^2 + \frac{8}{9} z^2 ≤1\}$ contains the half ball $B∩\{x≥0\}$
and we also have $Vol(F) ≤e^{−1/(2(3+1))} Vol(B)$.
If we now apply $T^{−1}(R^{−1}(·))$ on $F$ we get the desired ellipsoid $E'$. One
can get from here the algebraic description of $E'$. Technichally, we need
to replace $x,y$, and $z$ in the algebraic discription of $F$ by $R(T(x,y,z))$.
\end{solution}
  





\end{enumerate}




  

\end{document}

%%% Local Variables:
%%% mode: latex
%%% TeX-master: t
%%% End:
