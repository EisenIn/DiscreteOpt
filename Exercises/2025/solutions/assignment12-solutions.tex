\documentclass[11pt]{article}

\usepackage{../ppackage}
\usepackage{bbm}
\usepackage{import}






\usepackage{tikz}
\usetikzlibrary{arrows.meta,patterns}
\usetikzlibrary{ipe} % ipe compatibility library

\usepackage{../../../Notes/tikzit}
\usepackage{../../../Notes/utf8math}
\usetikzlibrary{positioning, 
                quotes}

\input{../../../Notes/TIKZ/digraph.tikzstyles}



\usepackage{url}


\newcommand{\solution}{
\bigskip\noindent
	\textbf{Solution: \\}}
	


\DeclareMathOperator{\size}{size}
\DeclareMathOperator{\conv}{conv}
\newcommand{\SV}{\mathrm{SV}}
\newcommand{\bigO}{O}
\newcommand{\cut}{\mathrm{cut}}
\newcommand{\LLL}{\mathrm{LLL}}
\newcommand{\setR}{\mathbb{R}}
\newcommand{\setZ}{\mathbb{Z}}
\newcommand{\setQ}{\mathbb{Q}}
\newcommand{\setC}{\mathbb{C}}
\newcommand{\setN}{\mathbb{N}}
\newcommand{\wt}[1]{\widetilde{#1}}
\newcommand{\opt}{{\sc 0/1-opt}\xspace}
\newcommand{\aug}{{\sc 0/1-aug}\xspace}
\newcommand{\psep}{{\sc 0/1-psep}\xspace}
\newcommand{\sep}{{\sc 0/1-sep}\xspace}
\newcommand{\fopt}{{\sc 0/1-testopt\xspace} }

\newcommand{\hpp}{\mathrm{HPP}}
\newcommand{\nodes}{\mathcal{V}}
\newcommand{\vol}{\mathrm{vol}}
\newcommand{\diag}{\mathrm{diag}}
\newcommand{\arcs}{\mathcal{A}}
\newcommand{\edges}{\mathcal{E}}
\newcommand{\paths}{\mathscr{P}}
\newcommand{\cycles}{\mathcal{C}}




\newcommand{\K}{{\mathcal K}}
\newcommand{\A}{{A}}
\newcommand{\B}{{B}}
\newcommand{\T}{\mathscr{T}}
\newcommand{\eE}{\mathscr{E}}
\newcommand{\eS}{\mathscr{S}}
\newcommand{\eP}{\mathscr{P}}
\newcommand{\eM}{\mathscr{M}}



\newcommand{\transp}{^{\mathrm{T}}}

\newcommand{\smallmat}[1]{\left( \begin{smallmatrix} #1 \end{smallmatrix}\right)}

\newcommand{\mat}[1]{ \begin{pmatrix} #1 \end{pmatrix}}
\newcommand{\smat}[1]{ \big(\begin{smallmatrix} #1 \end{smallmatrix}\big)}

\newcommand{\pc}{\mathscr{P}}
\newcommand{\ob}{\mathscr{O}}
\newcommand{\odds}{\mathscr{W}}
\newcommand{\up}{\mathscr{U}}
\newcommand{\ef}{\mathscr{F}}
\newcommand{\eh}{\mathscr{H}}
\newcommand{\ev}{\mathscr{V}}
\newcommand{\ec}{\mathscr{C}}
\newcommand{\eu}{\mathscr{U}}

\newcommand{\lex}{\mathrm{lex}}

\renewcommand{\leq}{\leqslant}
\renewcommand{\geq}{\geqslant}









\newcommand{\linhull}{\mathrm{lin.hull}}
\newcommand{\affhull}{\mathrm{affine.hull}}
\newcommand{\charcone}{\mathrm{char.cone}}
\newcommand{\cone}{\mathrm{cone}}
\newcommand{\rank}{\mathrm{rank}}
\newcommand{\wb}[1]{\overline{#1}}



\usepackage{enumerate}

      
\institute{\'Ecole Polytechnique F\'ed\'erale de Lausanne}
\lecture{Discrete Optimization}
\faculty{Prof. Eisenbrand}
\term{Spring 2025}
\publishdate{May 13, 2025}
\duedate{ }
\problemset{Assignment~12}

\begin{document}
\makeheader

\begin{enumerate}[1)]

\item Show that the unit simplex $∆ = conv\{0,e_1,\hdots,e_n\}⊂\setR^n$ has volume $\frac{1}{n!}$.


\begin{solution}
We will solve the problem using induction on n. Starting with $n = 2$, we have that $∆ =
∆((0,0),(1,0),(0,1)) = conv \{0,e_1,e_2\}$, where $∆((0,0),(1,0),(0,1))$ means the triangle between
the three vertices $(0,0),(1,0) and (0,1)$. It is clear that the volume of $∆$ is equal to $1/2$.
For the inductive step we use

\begin{align*}
vol_n(\Delta_n) & = \int_0^1 vol_{n-1}(x\cdot \Delta_{n-1}) dx \\
& = \int_0^1 x^{n-1}vol_{n-1}(\Delta_{n-1}) dx \\
& = \frac{1}{(n-1)!}\int_0^1 x^{n-1}dx \\
& = \frac{1}{n!}
\end{align*}
In this equation, we have use the result that, $vol_n(λK) = λ^nvol_n(K)$, for any $K ⊂\setR^n$. We also
have used the induction hypothesis in the third equation. Therefore, we get the desired result.

\end{solution}



\item Let $P= \{x∈\setR^n : Ax≤b\}$ be a full dimensional $0/1$ polytope and $c ∈\setZ^n$. We will show how we
can use the ellipsoid method to solve the optimization problem $\max c^Tx: x∈P$.
Define $z^*:= \max c^Tx: x∈P$ and $c_{max} := \max \{|c_i|: 1 ≤i≤n\}$.
\begin{enumerate}[i)]
\item Show that the ellipsoid method requires $O(n^3 \log(n)c_{max})$ iterations to decide whether $P ∩
(c^⊤x≥β) = ∅$ for some integer $β$ (i.e. find a suitable initial ellipsoid and stopping value L).
\item Show that we can use binary search to find $z^*$ with $\log(nc_{max})$ calls to the ellipsoid method.
\item Show how you can find an optimal solution $x^*$ such that $c^Tx^*= z^*$ in polynomial time.
\end{enumerate}


\begin{solution}
\begin{enumerate}[i)]
\item First note that all $0/1$ polytopes are contained inside the ball centered at $1/2\cdot \vec{1}$ with radius $√n/2$.
We can upper bound the volume of this ball by $√n^n$. Then observe that $P ∩(c^Tx ≥β) = ∅$ iff
$P':= P ∩(c^Tx≥β−1/2) = ∅$, since the vertices of $P$ are integral. So we want to lower bound the
volume of $P'$. Let $x_0$ be an integral vertex in $P'$. Since $P$ is full dimensional it contains a simplex
$∆ = conv \{x_0,x_1,\hdots,x_n\}$ of volume $1/n!$. The idea is to scale this simplex so that it is contained in
$P'$. We define $α=\frac{1}{2nc_{max}}$ and consider the simplex $∆'= [z_0,z_1,\hdots,z_{n−1}]$ where $z_i = x_0 +α(x_i−x_0)$.
To see that $∆'$ is contained in $P'$ we can check that each $z_i$ is in $P$ and satisfies $c^Tz_i ≥β$. Then we
obtain that
$$vol(P') ≥vol(∆') ≥\frac{1}{n!}\left( \frac{1}{2nc_{max}}\right).$$
Setting $L= vol(∆')$ and using the fact that the ellipsoid method terminates in $O(n\log(vol(I_{init})/L))$
gives us the correct bound.
\item We use the fact that the value of $c^Tx$ lies between $−nc_{max}$ and $nc_{max}$ for any vertex $x$ of $P$ to
conclude that $z^*$ must also lie within these bounds. Using binary search on this interval of integer
points takes $\log(nc_{max})$ steps.
\item The algorithm described in (i) and (ii) gives us the optimal value $z^*$ but also a point $y ∈P$
such that $z^*≥c^Ty≥z^*−1$. We now take this point and project it onto the hyperplane $c^Tx= z^*.$
Let $y'$ be the projection. If $y'∈P$ then we are done, otherwise we find a point on the line segment
$yy'$ that intersects a facet of $P$. We have now reduced the dimension of our problem by one and
can proceed by once again projecting this new point onto the hyperplane $c^Tx= z^*.$ Continuing in
this way we will arrive at an optimal solution in polynomial time.


\end{enumerate}

\end{solution}



\item Consider the complete graph $G_n$ with 3 vertices, i.e., $G= (\{1,2,3\}, \binom{3}
{2} )$. Is the polyhedron of the linear programming relaxation of the vertex-cover integer program integral?

\begin{solution}
The vertex-cover IP for the $G_3$ looks as follows:
\begin{align*}
\min \quad w_1x_1 + & w_2x_2 + w_3x_3\\
&x1 + x2  ≥ 1\\
&x1 + x3 ≥ 1\\
&x2 + x3 ≥ 1\\
&x1, x2, x3 ∈ \setN.
\end{align*}
Let $w_1 = w_2 = w_3 = 1$. Observe that we need at least two of the vertices in our vertex cover,
i.e. the optimum of this integer program is $2$. On the other hand, the vector $x =(1/2, 1/2, 1/2)$ is a
feasible solution to the linear programming relaxation of objective value $3/2 < 2$.
Assume that the polyhedron of the LP relaxation is integral. Thus, then the simplex algorithm will compute an optimal integer solution to the relaxation.
As we have seen above, every integer solution has a value of at least 2, while an optimum
fractional solution has a value of at less than $3/2$ , a contradiction.
\end{solution}


\item Consider the polyhedron $P = \{x ∈ \setR^3 : x_1 + 2 x_2 + 4 x_3 ≤ 4, x ≥ 0\}$. Show that this polyhedron is integral. 


\begin{solution}
We can write $P$ as follows:
$P = \{x ∈ \setR^3 : Ax ≤ b\}$, where $$A = \begin{pmatrix} 1& 2&4 \\-1& 0 & 0 \\ 0 & -1& 0 \\0 & 0 & -1\end{pmatrix}$$ and $b= \begin{pmatrix}4 & 0 & 0 & 0\end{pmatrix}$ . Clearly $A$ is of full column rank, thus we know that $P$ has a vertex. It is sufficient to show
that every vertex is integral. The vertices of P are defined as the unique solutions of the $\binom{4}{3}$ many different subsystems obtained
by choosing three out of the four rows. Then, the unique solution of the system
$$\begin{pmatrix} 1 & 2 & 4 \\ -1 & 0 & 0 \\ 0 & -1 & 0\end{pmatrix}x = \begin{pmatrix} 4 \\ 0 \\ 0 \end{pmatrix}$$
is $x = (0, 0, 1)$. The unique solution to the system 
$$\begin{pmatrix} 1 & 2 & 4 \\ -1 & 0 & 0 \\ 0 & 0 & -1\end{pmatrix}x = \begin{pmatrix} 4 \\ 0 \\ 0 \end{pmatrix}$$
is $x = (0,2,0)$. The unique solution to the system 
$$\begin{pmatrix} 1 & 2 & 4 \\ 0 & -1 & 0 \\ 0 & 0 & -1\end{pmatrix}x = \begin{pmatrix} 4 \\ 0 \\ 0 \end{pmatrix}$$
is $x = (4, 0, 0)$. The unique solution to the system 
$$\begin{pmatrix} -1 & 0 & 0 \\ 0 & -1 & 0 \\ 0 & 0 & -1\end{pmatrix}x = \begin{pmatrix} 0 \\ 0 \\ 0 \end{pmatrix}$$
is $x = (0, 0, 0)$. 
Hence, P is integral.

\end{solution}

\end{enumerate}




  

\end{document}

%%% Local Variables:
%%% mode: latex
%%% TeX-master: t
%%% End:
