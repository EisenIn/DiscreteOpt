\documentclass[11pt]{article}

\usepackage{../ppackage}
\usepackage{bbm}
\usepackage{import}






\usepackage{tikz}
\usetikzlibrary{arrows.meta,patterns}
\usetikzlibrary{ipe} % ipe compatibility library

\usepackage{../../../Notes/tikzit}
\usepackage{../../../Notes/utf8math}

\input{../../../Notes/TIKZ/digraph.tikzstyles}



\usepackage{url}


\newcommand{\solution}{
\bigskip\noindent
	\textbf{Solution: \\}}
	


\DeclareMathOperator{\size}{size}
\DeclareMathOperator{\conv}{conv}
\newcommand{\SV}{\mathrm{SV}}
\newcommand{\bigO}{O}
\newcommand{\cut}{\mathrm{cut}}
\newcommand{\LLL}{\mathrm{LLL}}
\newcommand{\setR}{\mathbb{R}}
\newcommand{\setZ}{\mathbb{Z}}
\newcommand{\setQ}{\mathbb{Q}}
\newcommand{\setC}{\mathbb{C}}
\newcommand{\setN}{\mathbb{N}}
\newcommand{\wt}[1]{\widetilde{#1}}
\newcommand{\opt}{{\sc 0/1-opt}\xspace}
\newcommand{\aug}{{\sc 0/1-aug}\xspace}
\newcommand{\psep}{{\sc 0/1-psep}\xspace}
\newcommand{\sep}{{\sc 0/1-sep}\xspace}
\newcommand{\fopt}{{\sc 0/1-testopt\xspace} }

\newcommand{\hpp}{\mathrm{HPP}}
\newcommand{\nodes}{\mathcal{V}}
\newcommand{\vol}{\mathrm{vol}}
\newcommand{\diag}{\mathrm{diag}}
\newcommand{\arcs}{\mathcal{A}}
\newcommand{\edges}{\mathcal{E}}
\newcommand{\paths}{\mathscr{P}}
\newcommand{\cycles}{\mathcal{C}}




\newcommand{\K}{{\mathcal K}}
\newcommand{\A}{{A}}
\newcommand{\B}{{B}}
\newcommand{\T}{\mathscr{T}}
\newcommand{\eE}{\mathscr{E}}
\newcommand{\eS}{\mathscr{S}}
\newcommand{\eP}{\mathscr{P}}
\newcommand{\eM}{\mathscr{M}}



\newcommand{\transp}{^{\mathrm{T}}}

\newcommand{\smallmat}[1]{\left( \begin{smallmatrix} #1 \end{smallmatrix}\right)}

\newcommand{\mat}[1]{ \begin{pmatrix} #1 \end{pmatrix}}
\newcommand{\smat}[1]{ \big(\begin{smallmatrix} #1 \end{smallmatrix}\big)}

\newcommand{\pc}{\mathscr{P}}
\newcommand{\ob}{\mathscr{O}}
\newcommand{\odds}{\mathscr{W}}
\newcommand{\up}{\mathscr{U}}
\newcommand{\ef}{\mathscr{F}}
\newcommand{\eh}{\mathscr{H}}
\newcommand{\ev}{\mathscr{V}}
\newcommand{\ec}{\mathscr{C}}
\newcommand{\eu}{\mathscr{U}}

\newcommand{\lex}{\mathrm{lex}}

\renewcommand{\leq}{\leqslant}
\renewcommand{\geq}{\geqslant}









\newcommand{\linhull}{\mathrm{lin.hull}}
\newcommand{\affhull}{\mathrm{affine.hull}}
\newcommand{\charcone}{\mathrm{char.cone}}
\newcommand{\cone}{\mathrm{cone}}
\newcommand{\rank}{\mathrm{rank}}
\newcommand{\wb}[1]{\overline{#1}}



\usepackage{enumerate}

      
\institute{\'Ecole Polytechnique F\'ed\'erale de Lausanne}
\lecture{Discrete Optimization}
\faculty{Prof. Eisenbrand}
\term{Spring 2025}
\publishdate{April 8, 2025}
\duedate{ }
\problemset{Assignment~8}

\begin{document}
\makeheader

\begin{enumerate}[1)]



\item Let $a, e, k ∈ \setN$ be three given natural numbers.
\begin{enumerate}
\item Argue that $a^{2^k}$ can be computed using $Θ(k)$ multiplications.
\item How many bits ($Θ$-notation) does $a^{2^k}$ have?
\item Let $(e_0, \hdots, e_l)$ be the bit representation of $e$, that is, $e = \displaystyle\sum_{i= 0}^l e_i2^i$ with $e_i ∈ \{0, 1\}$ for $i= 0, \hdots, l$.
Complete the following algorithm by replacing each occurrence of three question marks (???) so
that it computes $a^e$ using $O(l)$ many arithmetic operations.
\begin{align*}
& E = 1\\
& S = a\\
&\text{For }(i=0 \text{ to }l) \\
& \quad \text{if  }(e_i == 1)\\
& \quad \quad E = E\cdot ??? \\
&\quad S = ??? \\
&\text{return }???
\end{align*}
\item Show that for given $a, e, N ∈ \setN$ one can compute $a^e (\mod N)$ in time polynomial in the binary
encoding length of $a, e$ and $N$.
\item Let $a, b, c, N ∈ \setN$ be given and suppose that $N$ is a prime number. Show that $a^{b^c} (\mod N)$ can
be computed in polynomial time in the binary encoding length of $a, b, c$ and $N$. You may use
Fermat’s little theorem: $a^N ≡ a (\mod N)$.
\end{enumerate}


\begin{solution}
\begin{enumerate}
\item Start with $a(0) = a$, and compute $a(i) = a(i− 1)· a(i− 1)$ for $i ∈ [k]$. By induction
$$a(k) = a^{2^{k−1}}· a^{2^{k−1}} = a^{2^k}.$$
Thus, $a^{2^k}$ has been obtained by performing $k$ multiplications.
\item  This has $\log(a^{2^k}) = Θ(2^k \log a)$ many bits.
\item One has $E = E\cdot S$, $S = S\cdot S$ and return $E$.
\item We use the same algorithm as in part c), but do operations modulo $N$, in particular we replace $E
= E\cdot S$ with $E = E\cdot S \mod N$ and $S = S\cdot S$ with $S = S\cdot S \mod N$. Observe that the modulo can be obtained by performing division with remainder. The correctness of the algorithm follows from
the fact that for any $a, b ∈ \setN$, $ab ≡ (a (\mod N))(b (\mod N)) (\mod N)$. The algorithm performs
$O(l)$ steps, where $l$ is the size of $e$, and in each step we perform division with remainder on
$E \cdot S$ (or $S \cdot S$) and $N$: this takes time polynomial in the binary encoding length (size) of the
input, in particular of $a, e, N$ (notice that the value of $A (\mod N)$ has size at most the size of $N$,
irrespectively of the size of $A$). Hence the total number of operations performed is polynomial
in the size of $a, e, N$.
\item  Using the hint, we have that $abc ≡ abc (\mod N−1) (\mod N)$. Hence we first apply the algorithm
from part d) to obtain $bc (\mod N− 1)$ in polynomial time in the size of $b, c, N$. Notice that the
size of this number, which we denote by $e$, is bounded by the size of N only. Now, we apply
again the algorithm from part d) to obtain $ae (\mod N)$, in time polynomial in the size of $a, N$.

\end{enumerate}
\end{solution}


\item There are n types of animals, and you want to assign them to two stables. Unfortunately,
some animals would eat other animals when left unattended. Therefore you need to assign
the animals carefully. There are m relations of the form “u eats v“, where u and v are animals.
Find an $O(n + m)$ algorithm that decides whether there is an assignment of animals to the
two stables such that no animal eats another one of the same stable, and outputs a feasible
assignment.\\
\textit{Hint: Breadth-first-search might be useful.}

\begin{solution}
Consider the following directed graph $D = (V, A)$, where each node in $V$ corresponds to an
animal, and we have an arc $(u, v)$ for each relation ”u eats v“. Observe an assignment of the
animals to the two stables is feasible if and only if the corresponding partition of the nodes
into sets $V_1$ and $V_2$ we have that $(u, v) ∉ A$ for all $u, v ∈ V_1$ and $(u, v) ∉ A$ for all $u, v ∈ V_2$.
In other words, there is a feasible assignment if and only if the underlying undirected
graph $G= (V,E )$, where $E := \{\{u, v\} : (u, v) ∈ A\}$ is bipartite.
Consider the following algorithm:
\begin{align*}
&FULLBFS(G= (V,E )) \\
& \quad D[v] \leftarrow \infty \ \ ∀v ∈ V \\
& \quad \text{for all }v ∈ V \\
& \quad \quad \text{if } D[v]= \infty \\
&\quad \quad \quad D[v]\leftarrow 0 \\
& \quad \quad \quad BFS(G, v, D) \\
& \quad \text{return } D
\end{align*}
We run this algorithm on the graph $G$. Let $D$ be the output. We define
$V_1 := \{v ∈ V : D[v]\text{ is odd}\}$
and
$V_2 := \{v ∈ V : D[v]\text{ is even}\}$.
We claim that the assignment given by $V_1$ and $V_2$ is a feasible solution for our problem if and
only if there exists a feasible solution. With the observation from above, it is sufficient to
show that $V_1$ and $V_2$ are feasible if and only if $G$ is bipartite.
Trivially if $V_1$ and $V_2$ are feasible, then $G$ is bipartite, since $V_1$ and $V_2$ give a bipartite partition of the nodes.
Now assume that $V_1$ and $V_2$ are infeasible, i.e. there is an edge $e ∈ E$ such that $e ⊆ V_1$ (or
$e ⊆ V_2$. In the latter case we swap the roles of V1 and V2). We need to show that G is not
bipartite. As seen in the lecture, it is sufficient to show that G contains an odd cycle.
Let $e = (u, w )$. Hence w is reachable from u and vice versa. Hence, u and w got their D-
label assigned in the same run of BFS in Line 6. Thus there is a node v such that there is a
v,u-path P of length $D[u]$ and a v, w -path P' of length $D[w ]$. Since u and w both belong to
V1, we have $D[u] ≡ D[w ] \mod 2$, and therefore $D[u] + D[v]$ is even. Let $P ∆P'$ be the set of
edges that are contained in either P or P' (but not in both of them). Note that since $D[u] +
D[v]$ is even, the number of arcs in this set even as well. Moreover $e$ is contained in neither
of the paths (otherwise the algorithm would not have assigned u and w to the same Vi ).
We conclude that $(P ∆P') ∪ \{e\}$ is an odd cycle in G.

\end{solution}


\item Let $M_{2^k}$ be a matrix of order $n:= 2^k$, where $k∈\setN_{>0}$ such that it is recursively defined as follows:
$$M_{2^k} = \begin{pmatrix} M_{2^{k−1}} &  M_{2^{k−1}} \\ M_{2^{k−1}} & −M_{2^{k−1}} \end{pmatrix} $$

and $M_1 = [1]$, a $1 \times 1$ matrix. Prove that $|\det(M_n)|= n^{n/2}$, i.e. that the Hadamard bound is tight.

\begin{solution}
We prove the statement by induction on $k∈\setN_0$ that $M^2_{2^k} = 2^kI_{2^k}$. For $k= 0$ one has $M^2_0 = M_1 =[1] = 2^0I_{2^0}$. Assume that the statement holds for $k$ and prove it for $k+ 1$. Then 


$$M^2_{2^{k+1}} = \begin{pmatrix} M_{2^{k}} &  M_{2^{k}} \\ M_{2^{k}} & −M_{2^{k}} \end{pmatrix} \begin{pmatrix} M_{2^{k}} &  M_{2^{k}} \\ M_{2^{k}} & −M_{2^{k}} \end{pmatrix} = \begin{pmatrix} 2M^2_{2^k} & 0  \\ 0 & 2M^2_{2^k} \end{pmatrix} = \begin{pmatrix} 2\cdot 2^k I_{2^k} & 0  \\ 0 & 2\cdot 2^k I_{2^k} \end{pmatrix} = 2^{k+1}I_{2^{k+1}}$$
where we have used the induction hypothesis in the second to last equality.
By using the statement above we have that $\det(M^2_n) = \det(n I_n) = n^n$ so $|\det(M_n)|= n^{n/2}.$
\end{solution}


\item The determinant of a matrix $A∈\setR^{n×n}$ can be computed by the recursive formula
$$\det(A) = \displaystyle\sum_{j=1}^n (-1)^{1+j} a_{1j}\det(A_{1j}), $$
where $A_{1j}$ is the $(n−1)\times(n−1)$ matrix that is obtained from $A$ by deleting its first row and j-th
column. This yields the following recursive algorithm.\\
Input: $A∈\setR^{n×n}$ \\
Output: $\det(A)$\\
\begin{align*}
& \text{if } (n= 1) \\
&\text{return }a_{11}\\
&\text{else} \\
& \quad d:= 0\\
& \quad\text{for }j = 1,...,n\\
& \quad \quad d:= (−1)^{1+j} \det(A_{1j} ) + d \\
&\quad \text{return }d
\end{align*}
Let $A∈\setR^{n×n}$ and suppose that the $n^2$ components of $A$ are pairwise different.
\begin{enumerate}
\item Suppose that $B$ is a matrix that can be obtained from $A$ by deleting the first k rows and k of
the columns of A. How many (recursive) calls of the form $\det(B)$ does the algorithm create?
\item How many different submatrices can be obtained from $A$ by deleting the first k rows and some
set of k columns? Conclude that the algorithm remains exponential, even if it does not expand
repeated subcalls.
\end{enumerate}

\begin{solution}
a) Let $i_1,...,i_k$ be the indices of the columns of $A$ that were removed to obtain $B$. We have to
count the number of nodes of the form $\det(B)$ in the recursion tree of the algorithm. Each node
of the tree can be identified by its level (nodes of the form $\det(A_{ij})$ are at level i) and a sequence
of column indices representing the columns of $A$ that are not columns of the submatrix called
by the node. Hence the nodes of the form $\det(B)$ are at level $k$ of the tree, and their sequences
are permutations of $i_1,...,i_k$ (note that there is only one submatrix of $A$ equal to $B$ since all
the entries are pairwise different). Hence there are $k!$ such nodes. \\
b) The number of such submatrices is clearly $\binom{n}{k}$, which is exponential for instance for $k= n/2$, we have that:
$$\binom{n}{n/2} = \frac{n \cdot \hdots (n/2 + 1)}{n/2 \cdot \hdots \cdot 1} \geq 2^{n/2}$$
such that even if the algorithm calls $\det(B)$ only once for any submatrix $B$, the algorithm remains exponential.
\end{solution}

\end{enumerate}



  

\end{document}

%%% Local Variables:
%%% mode: latex
%%% TeX-master: t
%%% End:
