\documentclass[11pt]{article}

\usepackage{../ppackage}
\usepackage{bbm}
\usepackage{import}






\usepackage{tikz}
\usetikzlibrary{arrows.meta,patterns}
\usetikzlibrary{ipe} % ipe compatibility library

\usepackage{../../../Notes/tikzit}
\usepackage{../../../Notes/utf8math}

\input{../../../Notes/TIKZ/digraph.tikzstyles}



\usepackage{url}


\newcommand{\solution}{
\bigskip\noindent
	\textbf{Solution: \\}}
	


\DeclareMathOperator{\size}{size}
\DeclareMathOperator{\conv}{conv}
\newcommand{\SV}{\mathrm{SV}}
\newcommand{\bigO}{O}
\newcommand{\cut}{\mathrm{cut}}
\newcommand{\LLL}{\mathrm{LLL}}
\newcommand{\setR}{\mathbb{R}}
\newcommand{\setZ}{\mathbb{Z}}
\newcommand{\setQ}{\mathbb{Q}}
\newcommand{\setC}{\mathbb{C}}
\newcommand{\setN}{\mathbb{N}}
\newcommand{\wt}[1]{\widetilde{#1}}
\newcommand{\opt}{{\sc 0/1-opt}\xspace}
\newcommand{\aug}{{\sc 0/1-aug}\xspace}
\newcommand{\psep}{{\sc 0/1-psep}\xspace}
\newcommand{\sep}{{\sc 0/1-sep}\xspace}
\newcommand{\fopt}{{\sc 0/1-testopt\xspace} }

\newcommand{\hpp}{\mathrm{HPP}}
\newcommand{\nodes}{\mathcal{V}}
\newcommand{\vol}{\mathrm{vol}}
\newcommand{\diag}{\mathrm{diag}}
\newcommand{\arcs}{\mathcal{A}}
\newcommand{\edges}{\mathcal{E}}
\newcommand{\paths}{\mathscr{P}}
\newcommand{\cycles}{\mathcal{C}}




\newcommand{\K}{{\mathcal K}}
\newcommand{\A}{{A}}
\newcommand{\B}{{B}}
\newcommand{\T}{\mathscr{T}}
\newcommand{\eE}{\mathscr{E}}
\newcommand{\eS}{\mathscr{S}}
\newcommand{\eP}{\mathscr{P}}
\newcommand{\eM}{\mathscr{M}}



\newcommand{\transp}{^{\mathrm{T}}}

\newcommand{\smallmat}[1]{\left( \begin{smallmatrix} #1 \end{smallmatrix}\right)}

\newcommand{\mat}[1]{ \begin{pmatrix} #1 \end{pmatrix}}
\newcommand{\smat}[1]{ \big(\begin{smallmatrix} #1 \end{smallmatrix}\big)}

\newcommand{\pc}{\mathscr{P}}
\newcommand{\ob}{\mathscr{O}}
\newcommand{\odds}{\mathscr{W}}
\newcommand{\up}{\mathscr{U}}
\newcommand{\ef}{\mathscr{F}}
\newcommand{\eh}{\mathscr{H}}
\newcommand{\ev}{\mathscr{V}}
\newcommand{\ec}{\mathscr{C}}
\newcommand{\eu}{\mathscr{U}}

\newcommand{\lex}{\mathrm{lex}}

\renewcommand{\leq}{\leqslant}
\renewcommand{\geq}{\geqslant}









\newcommand{\linhull}{\mathrm{lin.hull}}
\newcommand{\affhull}{\mathrm{affine.hull}}
\newcommand{\charcone}{\mathrm{char.cone}}
\newcommand{\cone}{\mathrm{cone}}
\newcommand{\rank}{\mathrm{rank}}
\newcommand{\wb}[1]{\overline{#1}}



\usepackage{enumerate}

      
\institute{\'Ecole Polytechnique F\'ed\'erale de Lausanne}
\lecture{Discrete Optimization}
\faculty{Prof. Eisenbrand}
\term{Spring 2025}
\publishdate{April 15, 2025}
\duedate{ }
\problemset{Assignment~9}

\begin{document}
\makeheader

\begin{enumerate}[1)]


\item Sudoku is the following puzzle: Given a matrix
  \begin{displaymath}
    A ∈ \{1,\dots,9,X\}^{9 ×9} 
  \end{displaymath}
  the task is to replace each $X$ in $A$ by a number in $\{1,\dots,9\}$ such that the following holds.
  \begin{enumerate}[a)] 
  \item Each line of $A$ contains all numbers in  $\{1,\dots,9\}$
  \item Each column of $A$ contains all numbers in  $\{1,\dots,9\}$
  \item If $A$ is written as
    \begin{displaymath}
      A =
      \begin{pmatrix}
        B_{11} & B_{12} & B_{13} \\
        B_{21} & B_{22} & B_{23} \\
        B_{31} & B_{32} & B_{33} 
      \end{pmatrix}
    \end{displaymath}
    where the $B_{ij}$ are $3 ×3$ matrices, the each of them contains all numbers in  $\{1,\dots,9\}$.

    \bigskip \noindent

    In the following, you are asked to design an integer program whose feasible solutions are solutions of a given Sudoku. The integer program has $9^3$ variables
    \begin{displaymath}
      x_{i,j,k} =
      \begin{cases}
        1 & \text{ if } s_{i,j} = k \\
        0 & \text{ otherwise.} 
      \end{cases}
    \end{displaymath}
    where $S ∈ \{0,\dots,9\}^{9 ×9}$ is the solution of the Sudoku represented by the variables. 


    The following set of constraints guarantees that each cell of the solution contains a number
    \begin{displaymath}
      1≤i,j≤9: \quad  ∑_{k=1}^9 x_{i,j,,k} = 1. 
    \end{displaymath}

    Write now constraints that guarantee the following.
    \begin{itemize}
    \item Each row must contain each number.
    \item Each column must contain each number.
    \item Each $B_{ij}$ must contain each number. 
    \end{itemize}

    Describe the final integer programming problem that links some of the variables to the input Sudoku. 
    
  \end{enumerate}

\begin{solution}
The condition that each row must contain each number, corresponds to the set of constraints
\[
\sum_{j=1}^9 x_{i,j,k} = 1,\quad \forall\, 1 \leq i,k \leq 9.
\]
The condition that each column must contain each number, corresponds to the set of constraints
\[
\sum_{i=1}^9 x_{i,j,k} = 1,\quad \forall\, 1 \leq j,k \leq 9.
\]
The condition that each $B_{ij}$ must contain each number, corresponds to the set of constraints
\[
\sum_{\alpha=0}^2 \sum_{\beta=0}^2 x_{i+\alpha,j+\beta,k} = 1,\quad \forall\, i,j=1,4,7,\; 1\leq k\leq 9.
\]
To get the final integer program, we only need the set of constraints which guarantees that the solution is compatible with the input, i.e., 
\[
x_{i,j,k} = 1,\quad \text{if in the input of Sudoku, the cell}\; (i,j) \;\text{is assigned the number} \; k
\]
and the integrality constraints
\[
x_{i,j,k} \in \{0,1\},\quad \forall\, 1 \leq i,j,k \leq 9.
\]
\end{solution}

\item Let $a, b ∈ℤ$ not both equal to zero. Describe an integer program with variables $x,y ∈ℤ$ whose optimal value is the greatest common divisor of $a$ and $b$ and with optimal solution $x,y ∈ℤ$ being the corresponding  Bézout-coefficients
  \begin{displaymath}
    \gcd(a,b) = x⋅a + y⋅b. 
  \end{displaymath}

\begin{solution}
Consider the following integer program:
\begin{align*}
  \min \quad & ax + by & \\
  s.t. \quad
  & x, y \in \mathbb{Z} \\
  & ax + by \geq 1
\end{align*}
We claim that the optimal solution to the integer program above is the greatest common divisor of $a$ and $b$. 
Notice that the integer program is feasible since $a,b$ are not both equal to zero. Let $d$ be the optimal solution of the integer program, with optimal solution $x,y$. 
It is easy to see that $\gcd(a,b) \mid d$. To show $d \mid \gcd(a,b)$, it suffices to show that $d\mid a$ and $d\mid b$. Suppose that $d\nmid a$. Then there exists $q,r \in \mathbb{Z}, 1\leq r < d$ such that $a=qd + r$. Therefore we have $r = a-qd = a-q(ax+by) = (1-qx)a+(-qy)b$, which contradicts the optimality of $d$. Similarly one can prove that $d\mid b$.
\end{solution}
 

 \item Consider the polyhedron $P = \{ x \in \setR^3 \colon x_1 + 2\,x_2 + 4\,
  x_3 \leq 4, \, x\geq0\}$. Show that this polyhedron is
  integral. 

\begin{solution}
By the result of Exercise 5, it suffices to show that each vertex of $P$ is integral. 
The system of constraints are
\begin{align*}
  x_1 + 2x_2 + 4x_3 &\leq 4, \\
  x_1 &\geq 0, \\
  x_2 &\geq 0, \\
  x_3 &\geq 0. 
\end{align*}
Hence the 4 vertices of $P$ are $(0,0,0), (4,0,0), (0,2,0), (0,0,1)$.
\end{solution}


\item \emph{(Clustering)} 
The following is a recurring problem in data science. Suppose we are given $m$ points $v_1,\dots,v_m ∈ ℝ^n$ and a number $k$. We want to identify a subset $S ⊆ \{v_1,\dots,v_m\}$ of size $k$ such that
\begin{displaymath}
  ∑_{i=1}^m d(v_i, S)  \text{ is minimal.} 
\end{displaymath}
Here $d (v,S)$ is the euclidean distance of $v$ to $S$. 
Write an integer program that models this problem.

\begin{solution}
We assume that all the pairwise Euclidean distance $d_{i,j}$ between any points $v_i, v_j, 1\leq i,j\leq m$ have been pre-computed.
Let $y_j$ be the decision variable indicating whether the point $v_j$ is chosen to be in $S$. 
Let $x_{i,j}$ to be the variable indicating whether the point $v_j$ is the closest point to $v_i$ in $S$. 
Consider the following integer program, which models the clustering problem:
\begin{align*}
  \min \quad & \sum_{i=1}^m \sum_{j=1}^m d_{i,j} x_{i,j} & \\
  s.t. \quad
  & \sum_{j=1}^m x_{i,j} = 1, \quad \forall\, 1\leq i\leq m \\
  & x_{i,j} \leq y_j, \quad \forall\, 1\leq i,j\leq m \\
  & \sum_{j=1}^m y_j = k, \\
  & x_{i,j}, y_j \in \{0,1\}, \quad \forall\, 1\leq i,j\leq m 
\end{align*}
\end{solution}


\item Show the following: A polyhedron $P \subseteq\setR^n$ with vertices is
integral, if and only if each vertex is integral. \label{i:item:3}

\begin{solution}
The ``only if'' part is easy since each vertex is itself a face of $P$. To show the ``if'' part, assume that each vertex of $P$ is integral. Let $P= \{x\in \mathbb{R}^n: Ax\leq b\}$.
Consider any nonempty face $F$ of $P$, which is of the form $F = P\cap \{x\in\mathbb{R}^n: c^T x = \beta\}$ where $c^T x \leq \beta$ is valid for $P$.

We first claim that there exists $u \in \mathbb{R}^m, u\geq 0$ such that $u^T A = c^T$ and $u^T b = \beta$. Indeed, consider the following pair of primal and dual linear programs:
\begin{displaymath}
  \begin{array}[t]{c}
    \max c^Tx \\
         Ax ≤ b
  \end{array} \quad \text{ and }\quad
  \begin{array}[t]{c}
    \min b^Ty \\
    y^T A = c^T\\
    y ≥ 0      
  \end{array} 
\end{displaymath}
Since $c^T x \leq \beta$ is valid for $P$, we know that $P$ is bounded. Also $P$ is feasible since it has vertices. The optimal value of the primal problem is $\beta$, since the face $F$ is nonempty. 
Then by the strong duality theorem, the dual problem has an optimal solution, say $u \in \mathbb{R}^m, u\geq 0$ with $u^T A = c^T$, such that the value is also $\beta$, i.e., $u^T b=\beta$.
Then we have $F = P \cap \{x\in \mathbb{R}^n: A^\prime x = b^\prime\}$ where $A^\prime x\leq b^\prime$ is a sub-system of $Ax\leq b$ by taking the constraint $a_i^Tx \leq b_i$ of $Ax\leq b$ if and only if $u_i > 0$.
\end{solution}
  
\end{enumerate}



  

\end{document}

%%% Local Variables:
%%% mode: latex
%%% TeX-master: t
%%% End:
