\documentclass[11pt]{article}

\usepackage{../ppackage}
\usepackage{bbm}
\usepackage{import}




\usepackage{tikz}
\usetikzlibrary{arrows.meta,patterns}
\usetikzlibrary{ipe} % ipe compatibility library

\usepackage{../../../Notes/tikzit}
\usepackage{../../../Notes/utf8math}

\input{../../../Notes/TIKZ/digraph.tikzstyles}



\usepackage{url}


\newcommand{\solution}{
\bigskip\noindent
	\textbf{Solution: \\}}


\usepackage{enumerate}

      
\institute{\'Ecole Polytechnique F\'ed\'erale de Lausanne}
\lecture{Discrete Optimization}
\faculty{Prof. Eisenbrand}
\term{Spring 2025}
\publishdate{February 18, 2025}
\duedate{ }
\problemset{Assignment~1}

\begin{document}
\makeheader

\problem
Provide a certificate (as in Theorem $1.1$ in the lecture notes) of the unsolvability of the linear equation 
  \begin{displaymath}
    \begin{pmatrix}
      2 & 1 & 0 \\
      5 & 4 & 1 \\
      7 & 5 & 1
    \end{pmatrix} \,
    \begin{pmatrix}
      x_1 \\ x_2 \\ x_3 
    \end{pmatrix} =
    \begin{pmatrix}
      1\\2\\4
    \end{pmatrix}
  \end{displaymath}
  
  \paragraph{Solution:}
  
  Consider $y = \begin{pmatrix} 1 \\ 1 \\ -1 \end{pmatrix}$ so that $y^T A = \begin{pmatrix} 7-7 \\ 5-5 \\ 1-1 \end{pmatrix} = \begin{pmatrix}0 \\ 0 \\ 0 \end{pmatrix}$ and $y^T b = 3-4 = -1$. 
  
  Then $y^T A = \vec{0}$ and $y^T b < 0$ such that the system must be infeasible. See that if there did exist a solution $\vec{x}$, then 
  \begin{align*}
  y^TA x & = y^T b \\
  \implies \vec{0}^T x & = y^Tb \\
  \implies 0 & = -1.
  \end{align*}
  
  As this is impossible, the system must be infeasible. 
  
  
 

  \problem

  Let $A ∈ ℝ^{3 ×2}$ be the matrix
  \begin{displaymath}
    A =
    \begin{pmatrix}
      -1 & 1 \\
      1 & 1 \\
      0 & -1 
    \end{pmatrix}
  \end{displaymath}
  and $b ∈ ℝ^3$ be the vector
  \begin{displaymath}
    b =
    \begin{pmatrix}
      1 \\ 1 \\ -2
    \end{pmatrix}
  \end{displaymath}
defining the system of inequalities $Ax ≤b$ that does not have a feasible solution. Find a Farkas' certificate, i.e., a  $λ ∈ ℝ^3_{≥0}$ with $λ^TA = 0$ and $λ^T b = -1$. 

\paragraph{Solution:}

Let $\lambda = \begin{pmatrix} 1/2 \\ 1/2 \\ 1 \end{pmatrix}$. Then $\lambda^TA = \begin{pmatrix} 0 \\ 0 \end{pmatrix}$ while $\lambda^Tb = -1$ such that this is a certificate of infeasibility. 


  

 \problem Show the ``if'' direction of the Farkas' lemma: given $A\in \R^{m\times n}, b\in \R^m$, if there exist a $\lambda \in \R^m_{\geqslant 0}$ such that $\lambda^\top A=0$ and $\lambda^\top b=-1$, then the system $Ax\leq b$ of linear inequalities does not have a solution. 
 
 \paragraph{Solution:}

Suppose there exists a vector $\vec{x}$ such that $Ax \leq b$. Then we have that 
\begin{align*}
\lambda^TAx & = \lambda^T(Ax) \\
&\leq \lambda^T b \\
& = -1 
\end{align*}
where the inequality follows from $\lambda \geq \vec{0}$.

But 
\begin{align*}
\lambda^TAx & = (\lambda^TA)x \\
& = \vec{0}^Tx \\
& = 0
\end{align*}
so that with the two equations together, we get $0 \leq -1$ which is a contradiction. 

 
 
 
\problem Consider the following linear program:
\begin{center}
\begin{tabular}{rllllll}
$\max$ & \ & $x$ & $+$ & $y$ & \ & \\ 
s.t. & \ & $3x$ & $+$ & $2y$ & $\leq$ & $6$ \\ 
\ & \ & $x$ & $+$ & $4y$ & $\leq$ & $4$. \
\end{tabular} 
\end{center} 
The solution $(x, y)=(8/5 , 3/5)$ satisfies the both constraints and has the objective value $11/5$. Provide a certificate that this is an optimal solution.


\paragraph{Solution:}
The LP for our problem is given by 
\begin{align*}
\max (x + y) \\
s.t. \begin{pmatrix}  3 & 2 \\ 1 & 4 \end{pmatrix} \begin{pmatrix} x \\ y \end{pmatrix} \leq \begin{pmatrix} 6 \\ 4 \end{pmatrix}.
\end{align*}

Consider the following linear combination of the two constraints given by $$ \lambda (3x + 2y) + \alpha(x + 4y)$$
where $\lambda, \alpha \geq 0$. Clearly, this preserves the inequalities so that $$\lambda (3x + 2y) + \alpha(x + 4y) \leq 6 \lambda + 4 \alpha$$
for any feasible $x, y$ for the LP. 

Now, we choose $\lambda = 3/10, \alpha = 1/10$ and get that for any feasible $x, y$ for the LP, 
\begin{align*}
&\lambda (3x + 2y) + \alpha(x + 4y) \leq 6 \lambda + 4 \alpha \\
\implies & 9/10 x + 6/10 y + 1/10 x + 4/10 y \leq 18/10 + 4/10 \\
\implies & x + y \leq 22/10 \\
\implies & x + y \leq 11/5.
\end{align*}

This shows that any feasible solution satisfies the objective $x + y \leq 11/5$ such that $(x, y)=(8/5 , 3/5)$ satisfies the both constraints and has the optimal objective value $11/5$. 



\newpage 
\problem

 Let $G = (V,A)$ be a directed graph and $s,t ∈ V$ be two designated vertices. For a vertex $v ∈ V$ we let
  \begin{displaymath}
    δ^+(v) = \{ (u,v) : u ∈ V, \, (u,v) ∈A\} \text{ and } δ^-(v) = \{ (v,u) : u ∈ V, \, (v,u) ∈A\}
  \end{displaymath}
  the \emph{arcs entering} and \emph{leaving} $v$ respectively.  Consider the following inequalities
  \begin{equation} 
    \begin{array}{rclc}      \displaystyle 
    ∑_{a ∈ δ^+(v) } x_a -  ∑_{a ∈ δ^-(v) } x_a & = &  0 & \, \quad v ∈ V \setminus \{s,t\} \\ \displaystyle 
      ∑_{a ∈ δ^+(s) } x_a -  ∑_{a ∈ δ^-(s) } x_a & = &  -1 & \\
      \displaystyle 
      ∑_{a ∈ δ^+(t) } x_a -  ∑_{a ∈ δ^-(t) } x_a & = &  1 & \\
      x_a &≥& 0 & a ∈A.
    \end{array}
    \label{eq:1}
  \end{equation}
  
  \begin{enumerate}[a)] 
  \item Consider the following digraph with $s$ and $t$ and a partial assignment of arc variables. Can this partial assignment be completed to a feasible solution satisfying the inequalities~\eqref{eq:1}? If yes, complete the assignment.
    \begin{center}
          \begin{tikzpicture}
	\begin{pgfonlayer}{nodelayer}
		\node [style=new style 0] (0) at (0, 4) {$s$};
		\node [style=new style 0] (1) at (0, -4) {$t$};
		\node [style=new style 0] (2) at (3, 2) {};
		\node [style=new style 0] (3) at (-3, 2) {};
		\node [style=new style 0] (4) at (0, 0) {};
		\node [style=new style 0] (5) at (3, -2) {};
		\node [style=new style 0] (6) at (-3, -2) {};
		\node [style=none] (7) at (-1.75, 3.5) {$.5$};
		\node [style=none] (8) at (2.25, 3.75) {$.75$};
		\node [style=none] (9) at (0.75, 2.5) {};
		\node [style=none] (10) at (1.25, 1.25) {$.25$};
		\node [style=none] (11) at (2.25, 0.25) {};
		\node [style=none] (12) at (-3.75, 0) {};
		\node [style=none] (13) at (-1.75, -3.25) {};
		\node [style=none] (14) at (0.5, -2) {$0$};
		\node [style=none] (15) at (2.25, -1) {};
		\node [style=none] (16) at (1.75, -3.25) {};
		\node [style=none] (17) at (-1, 1.25) {.25};
	\end{pgfonlayer}
	\begin{pgfonlayer}{edgelayer}
		\draw [style=edge] (0) to (3);
		\draw [style=edge] (3) to (4);
		\draw [style=edge, bend left] (0) to (2);
		\draw [style=edge, bend left=45] (2) to (4);
		\draw [style=edge, bend left] (2) to (0);
		\draw [style=edge, bend left, looseness=1.25] (4) to (2);
		\draw [style=edge] (3) to (6);
		\draw [style=edge] (4) to (1);
		\draw [style=edge] (4) to (5);
		\draw [style=edge] (6) to (1);
		\draw [style=edge] (5) to (1);
	\end{pgfonlayer}
\end{tikzpicture}
   
    \end{center}
    
    
    \paragraph{Solution:}
    
    \begin{center}
   \begin{tikzpicture}
	\begin{pgfonlayer}{nodelayer}
		\node [style=new style 0] (0) at (0, 4) {$s$};
		\node [style=new style 0] (1) at (0, -4) {$t$};
		\node [style=new style 0] (2) at (3, 2) {};
		\node [style=new style 0] (3) at (-3, 2) {};
		\node [style=new style 0] (4) at (0, 0) {};
		\node [style=new style 0] (5) at (3, -2) {};
		\node [style=new style 0] (6) at (-3, -2) {};
		\node [style=none] (7) at (-1.75, 3.5) {$.5$};
		\node [style=none] (8) at (2.25, 3.75) {$.75$};
		\node [style=none] (9) at (0.75, 2.5) {$.25$};
		\node [style=none] (10) at (1.25, 1.25) {$.25$};
		\node [style=none] (11) at (2.25, 0.25) {$.75$};
		\node [style=none] (12) at (-3.75, 0) {$.25$};
		\node [style=none] (13) at (-1.75, -3.25) {$.25$};
		\node [style=none] (14) at (0.5, -2) {$0$};
		\node [style=none] (15) at (2.25, -1) {$.75$};
		\node [style=none] (16) at (1.75, -3.25) {$.75$};
		\node [style=none] (17) at (-1, 1.25) {.25};
	\end{pgfonlayer}
	\begin{pgfonlayer}{edgelayer}
		\draw [style=edge] (0) to (3);
		\draw [style=edge] (3) to (4);
		\draw [style=edge, bend left] (0) to (2);
		\draw [style=edge, bend left=45] (2) to (4);
		\draw [style=edge, bend left] (2) to (0);
		\draw [style=edge, bend left, looseness=1.25] (4) to (2);
		\draw [style=edge] (3) to (6);
		\draw [style=edge] (4) to (1);
		\draw [style=edge] (4) to (5);
		\draw [style=edge] (6) to (1);
		\draw [style=edge] (5) to (1);
	\end{pgfonlayer}
\end{tikzpicture}
\end{center}

    
    
\item Show the following for a digraph $G = (V,A)$ with $s,t ∈ V$: If there is a path connecting $s$ and $t$ in $G$, then the system of inequalities~\eqref{eq:1} has a feasible solution


\paragraph{Solution:}
Given a path $t$ to $s$, set for every edge on the path the value $x_e = +1$ and for every edge not on the path the flow value $x_e = 0$. Given this choice of flow, it is not hard to see that all constraints are satisfied by this solution. 


  \item (*)  Show the following for a digraph $G = (V,A)$ with $s,t ∈ V$: If  the system of inequalities~\eqref{eq:1} has a feasible solution, then  there is a path connecting $s$ and $t$ in $G$. 
  
  
  \paragraph{Solution:}
  Suppose that the system of inequalities has a feasible solution $\vec{x}$ but that there is no path from $s$ to $t$. Call A the set of nodes that are reachable from s using edges (u, v) that have $x(u, v) > 0$. Then A contains s and it does not
contain t as we assume that $t$ is not reachable from $s$. Thus the set $A \subset V$ is a cut where $A$ contains $s$ and $V \setminus A$ contains $t$. The net value out of A is equal to $\sum_{(u,v): u \in A, v \notin A} x_{(u, v)}$. However by definition of $A$, any such edge that exits $A$ must have value $0$ in the vector $x$ as otherwise the node would be reachable from $s$ using positive valued edges. This means that for any $v \in V \setminus A$, $\sum_{a \in \delta^+(v)} x_a = 0$. In order to satisfy the first equality, this means that for every $v \in V \setminus A - t$, it must also be that $\sum_{a \in \delta^-(v)} x_a = 0$. But then $t$ receives $0$ value on incoming edges as it is only reachable by nodes in $V \setminus A$ such that the third equality of the LP is not satisfied. This is a contradiction to $\vec{x}$ being feasible. 
  
   \end{enumerate}
  

\end{document}

%%% Local Variables:
%%% mode: latex
%%% TeX-master: t
%%% End:
