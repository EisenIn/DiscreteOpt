\documentclass[11pt]{article}

\usepackage{../ppackage}
\usepackage{bbm}
\usepackage{import}






\usepackage{tikz}
\usetikzlibrary{arrows.meta,patterns}
\usetikzlibrary{ipe} % ipe compatibility library

\usepackage{../../../Notes/tikzit}
\usepackage{../../../Notes/utf8math}

\input{../../../Notes/TIKZ/digraph.tikzstyles}



\usepackage{url}


\newcommand{\solution}{
\bigskip\noindent
	\textbf{Solution: \\}}
	


\DeclareMathOperator{\size}{size}
\DeclareMathOperator{\conv}{conv}
\newcommand{\SV}{\mathrm{SV}}
\newcommand{\bigO}{O}
\newcommand{\cut}{\mathrm{cut}}
\newcommand{\LLL}{\mathrm{LLL}}
\newcommand{\setR}{\mathbb{R}}
\newcommand{\setZ}{\mathbb{Z}}
\newcommand{\setQ}{\mathbb{Q}}
\newcommand{\setC}{\mathbb{C}}
\newcommand{\setN}{\mathbb{N}}
\newcommand{\wt}[1]{\widetilde{#1}}
\newcommand{\opt}{{\sc 0/1-opt}\xspace}
\newcommand{\aug}{{\sc 0/1-aug}\xspace}
\newcommand{\psep}{{\sc 0/1-psep}\xspace}
\newcommand{\sep}{{\sc 0/1-sep}\xspace}
\newcommand{\fopt}{{\sc 0/1-testopt\xspace} }

\newcommand{\hpp}{\mathrm{HPP}}
\newcommand{\nodes}{\mathcal{V}}
\newcommand{\vol}{\mathrm{vol}}
\newcommand{\diag}{\mathrm{diag}}
\newcommand{\arcs}{\mathcal{A}}
\newcommand{\edges}{\mathcal{E}}
\newcommand{\paths}{\mathscr{P}}
\newcommand{\cycles}{\mathcal{C}}




\newcommand{\K}{{\mathcal K}}
\newcommand{\A}{{A}}
\newcommand{\B}{{B}}
\newcommand{\T}{\mathscr{T}}
\newcommand{\eE}{\mathscr{E}}
\newcommand{\eS}{\mathscr{S}}
\newcommand{\eP}{\mathscr{P}}
\newcommand{\eM}{\mathscr{M}}



\newcommand{\transp}{^{\mathrm{T}}}

\newcommand{\smallmat}[1]{\left( \begin{smallmatrix} #1 \end{smallmatrix}\right)}

\newcommand{\mat}[1]{ \begin{pmatrix} #1 \end{pmatrix}}
\newcommand{\smat}[1]{ \big(\begin{smallmatrix} #1 \end{smallmatrix}\big)}

\newcommand{\pc}{\mathscr{P}}
\newcommand{\ob}{\mathscr{O}}
\newcommand{\odds}{\mathscr{W}}
\newcommand{\up}{\mathscr{U}}
\newcommand{\ef}{\mathscr{F}}
\newcommand{\eh}{\mathscr{H}}
\newcommand{\ev}{\mathscr{V}}
\newcommand{\ec}{\mathscr{C}}
\newcommand{\eu}{\mathscr{U}}

\newcommand{\lex}{\mathrm{lex}}

\renewcommand{\leq}{\leqslant}
\renewcommand{\geq}{\geqslant}









\newcommand{\linhull}{\mathrm{lin.hull}}
\newcommand{\affhull}{\mathrm{affine.hull}}
\newcommand{\charcone}{\mathrm{char.cone}}
\newcommand{\cone}{\mathrm{cone}}
\newcommand{\rank}{\mathrm{rank}}
\newcommand{\wb}[1]{\overline{#1}}



\usepackage{enumerate}

      
\institute{\'Ecole Polytechnique F\'ed\'erale de Lausanne}
\lecture{Discrete Optimization}
\faculty{Prof. Eisenbrand}
\term{Spring 2025}
\publishdate{February 25, 2025}
\duedate{ }
\problemset{Assignment~2}

\begin{document}
\makeheader

\begin{enumerate}[1)]

\item  \label{item:ex-9}
  Consider the unit ball $B_n = \{ x ∈ ℝ^n : \|x\|_2≤1\}$. Show that the set of extreme points of $B$ is the sphere $S^{(n-1)} = \{ x ∈ ℝ^n : \|x\|_2 =1\}$.
  
  \begin{solution}
  Let $x \in \setR^n$ be such that $\|x\|_2 =1$. We show that $x$ is then an extreme point of $B$. Firstly, it is clear that $x$ is in $B$. 
  Assume that $x$ wasn't an extreme point such that there exist some $y \neq z \neq x \in B$ with $\lambda y + (1-\lambda) z = x$ for some $\lambda \in (0,1)$. Then $$1 = \| x\|_2 = \| \lambda y + (1-\lambda) z \| \leq \lambda \|y\| + (1-\lambda)\|z\| \leq \lambda \cdot 1 + (1-\lambda) \cdot 1 = 1$$
where the first inequality comes from the triangle inequality and the second follows from $z, y \in B$. So the inequalities must hold with equality everywhere such that 
\begin{align}
&\|z\| = 1 \\
&\|y\| = 1\\
&\|\lambda y + (1 - \lambda )z \| =  \lambda \|y\| + (1-\lambda)\|z\|. 
\end{align}

However, if $\|\lambda y + (1 - \lambda )z \| =  \lambda \|y\| + (1-\lambda)\|z\|$ then $y, z$ must be linearly dependent. But then $z \neq y$, $z = \alpha y$ for some $\alpha \notin \{0, 1\}$.  Then $x = \lambda y + (1 - \lambda )\alpha y = (\lambda + \alpha- \alpha \lambda) y$. But $ 1 = \|x\| =(\lambda + \alpha - \alpha \lambda)\|y \| = (\lambda + \alpha - \alpha \lambda)$. Thus $x = y$ a contradiction to the assumption $y, z \neq x$. Thus $x$ is an extreme point. 

Next, we show there are no other extreme points of $B$. Assume, there is some $z \in \setR^n$ such that $\|z\| < 1 $ and $z$ is an extreme point of $B$. Let $\|z\| = \alpha$ for some value $\alpha <1$. Then the vector $x = \frac{1}{\alpha}z$ satisfies $x \in B, x \neq z$. Moreover, the vector $ y = \vec{0} $ satisfies $y \neq x, y \neq z, y \in B$. But then, notice that $z = \alpha x + (1-\alpha) y$ where $\alpha \in (0, 1)$ such that $z$ cannot be an extreme point. 




  \end{solution}
  
  
\item \label{item:10}
  A \emph{line} is a set $L = \{ x ⋅ d +t : x ∈ ℝ\} ⊆ ℝ^n$ where $d,t ∈ ℝ^n$ $d ≠0$. Show the following.

  A non-empty polyhedron $P = \{ x ∈ ℝ^n : Ax ≤ b\} ⊆ ℝ^n$ contains a line if  and only if $\rank(A) <n$.
  
  \begin{solution}
  We start by showing $P$ contains a line $\implies$ $rank(A) < n$. 
  
  Let $d \cot x + t$ denote the line contained in $P$. As $P$ contains this line, for any $x \in \setR$, $A(d\cdot x +t) \leq b$. 
  \end{solution}

\item \label{item:11} Two different vertices $v_1 ≠ v_2$ of a polyhedron $P = \{ x ∈ ℝ^n : Ax≤b\}$ are called \emph{adjacent}, if there exists a subsystem $A'x ≤ b'$ of $Ax≤b$ with
\begin{enumerate}[i)] 
\item $A'v_1 = b'$ and $A'v_2 = b'$  and
\item $\rank(A') = (n-1)$. 
\end{enumerate}

Show that there exists a valid inequality $c^Tx ≤ δ$ of $P$ with
\begin{displaymath}
 \left( P ∩ \{ x ∈ ℝ^n : c^Tx = δ \} \right) = \conv\{v_1,v_2\}. 
\end{displaymath}
\item Let $\{C_i\}_{i\in I}$ be a family of convex subsets of $\setR^n$.
  Show that the intersection $\bigcap_{i\in I} C_i$ is convex.
\item Show that the set of feasible solutions  of a linear program  is
  convex. \label{conv:item:1}
  
  \item Suppose $P= \{x\in\setR^n : Ax\leq b,Cx\leq d\}$. Show that the set of vertices of
$Q= \{x\in\setR^n : Ax\leq b,Cx= d\}$ is a subset of the set of vertices of $P$.



\end{enumerate}



  

\end{document}

%%% Local Variables:
%%% mode: latex
%%% TeX-master: t
%%% End:
