\documentclass[11pt]{article}

\usepackage{../ppackage}
\usepackage{bbm}
\usepackage{import}






\usepackage{tikz}
\usetikzlibrary{arrows.meta,patterns}

\usepackage{../../../Notes/tikzit}
\usepackage{../../../Notes/utf8math}

\input{../../../Notes/TIKZ/digraph.tikzstyles}



\usepackage{url}


\newcommand{\solution}{
\bigskip\noindent
	\textbf{Solution: \\}}
	


\DeclareMathOperator{\size}{size}
\DeclareMathOperator{\conv}{conv}
\newcommand{\SV}{\mathrm{SV}}
\newcommand{\bigO}{O}
\newcommand{\cut}{\mathrm{cut}}
\newcommand{\LLL}{\mathrm{LLL}}
\newcommand{\setR}{\mathbb{R}}
\newcommand{\setZ}{\mathbb{Z}}
\newcommand{\setQ}{\mathbb{Q}}
\newcommand{\setC}{\mathbb{C}}
\newcommand{\setN}{\mathbb{N}}
\newcommand{\wt}[1]{\widetilde{#1}}
\newcommand{\opt}{{\sc 0/1-opt}\xspace}
\newcommand{\aug}{{\sc 0/1-aug}\xspace}
\newcommand{\psep}{{\sc 0/1-psep}\xspace}
\newcommand{\sep}{{\sc 0/1-sep}\xspace}
\newcommand{\fopt}{{\sc 0/1-testopt\xspace} }

\newcommand{\hpp}{\mathrm{HPP}}
\newcommand{\nodes}{\mathcal{V}}
\newcommand{\vol}{\mathrm{vol}}
\newcommand{\diag}{\mathrm{diag}}
\newcommand{\arcs}{\mathcal{A}}
\newcommand{\edges}{\mathcal{E}}
\newcommand{\paths}{\mathscr{P}}
\newcommand{\cycles}{\mathcal{C}}




\newcommand{\K}{{\mathcal K}}
\newcommand{\A}{{A}}
\newcommand{\B}{{B}}
\newcommand{\T}{\mathscr{T}}
\newcommand{\eE}{\mathscr{E}}
\newcommand{\eS}{\mathscr{S}}
\newcommand{\eP}{\mathscr{P}}
\newcommand{\eM}{\mathscr{M}}



\newcommand{\transp}{^{\mathrm{T}}}

\newcommand{\smallmat}[1]{\left( \begin{smallmatrix} #1 \end{smallmatrix}\right)}

\newcommand{\mat}[1]{ \begin{pmatrix} #1 \end{pmatrix}}
\newcommand{\smat}[1]{ \big(\begin{smallmatrix} #1 \end{smallmatrix}\big)}

\newcommand{\pc}{\mathscr{P}}
\newcommand{\ob}{\mathscr{O}}
\newcommand{\odds}{\mathscr{W}}
\newcommand{\up}{\mathscr{U}}
\newcommand{\ef}{\mathscr{F}}
\newcommand{\eh}{\mathscr{H}}
\newcommand{\ev}{\mathscr{V}}
\newcommand{\ec}{\mathscr{C}}
\newcommand{\eu}{\mathscr{U}}

\newcommand{\lex}{\mathrm{lex}}

\renewcommand{\leq}{\leqslant}
\renewcommand{\geq}{\geqslant}









\newcommand{\linhull}{\mathrm{lin.hull}}
\newcommand{\affhull}{\mathrm{affine.hull}}
\newcommand{\charcone}{\mathrm{char.cone}}
\newcommand{\cone}{\mathrm{cone}}
\newcommand{\rank}{\mathrm{rank}}
\newcommand{\wb}[1]{\overline{#1}}



\usepackage{enumerate}

      
\institute{\'Ecole Polytechnique F\'ed\'erale de Lausanne}
\lecture{Discrete Optimization}
\faculty{Prof. Friedrich Eisenbrand}
\term{Spring 2026}
\publishdate{March 17, 2026}
\duedate{ }
\problemset{Problem Set~4}

\begin{document}
\makeheader

\begin{enumerate}[1)]
\item Let $P = \{x: Ax \leq  b\} \subseteq \setR^n$ be a polyhedron. Show that $x^\ast$ is an extreme point $\iff \forall x_1 \neq x_2 \in P, x^\ast \neq \frac{1}{2}x_1 + \frac{1}{2}x_2$. 


\item Suppose that the linear program $\max \{c^Tx \colon x \in \R^n, \, Ax \leq b\}$ is non-degenerate and $B$ is an optimal basis. Show that the linear program has a unique optimal solution if and only if $\lambda_B>0$. 


\item For each of the following assertion, provide a proof or a counterexample. 
  \begin{enumerate}[i)]
  \item An index that has just left the basis $B$ in the simplex
    algorithm cannot enter in the very next iteration.
  \item An index that has just entered the basis $B$ in the simplex
    algorithm cannot leave again in the very next iteration. 
  \end{enumerate}
  
  
\item Consider the following linear program:
\begin{align*}
\max  \quad & 6a+ 9b+ 2c \\
\text{subject to } & a+ 3b+ c ≤−4 \\
&b+ c ≤−1 \\
& 3a+ 3b−c ≤1 \\
& a ≤0  \\
& b ≤0 \\
& c ≤0. 
\end{align*}
Solve the linear program with the Simplex method and initial vertex $(−1,−1,0)^T$ . For each iteration indicate all the parameters including the optimal value and the proof of optimality.
  
 \item Fill in the blanks to complete the code for the Simplex.py file which runs a simplex algorithm. 




\end{enumerate}



  

\end{document}

%%% Local Variables:
%%% mode: latex
%%% TeX-master: t
%%% End:
