\documentclass[11pt]{article}

\usepackage{../ppackage}
\usepackage{bbm}
\usepackage{import}






\usepackage{tikz}
\usetikzlibrary{arrows.meta,patterns}

\usepackage{../../../Notes/tikzit}
\usepackage{../../../Notes/utf8math}

\input{../../../Notes/TIKZ/digraph.tikzstyles}



\usepackage{url}


\newcommand{\solution}{
\bigskip\noindent
	\textbf{Solution: \\}}
	


\DeclareMathOperator{\size}{size}
\DeclareMathOperator{\conv}{conv}
\newcommand{\SV}{\mathrm{SV}}
\newcommand{\bigO}{O}
\newcommand{\cut}{\mathrm{cut}}
\newcommand{\LLL}{\mathrm{LLL}}
\newcommand{\setR}{\mathbb{R}}
\newcommand{\setZ}{\mathbb{Z}}
\newcommand{\setQ}{\mathbb{Q}}
\newcommand{\setC}{\mathbb{C}}
\newcommand{\setN}{\mathbb{N}}
\newcommand{\wt}[1]{\widetilde{#1}}
\newcommand{\opt}{{\sc 0/1-opt}\xspace}
\newcommand{\aug}{{\sc 0/1-aug}\xspace}
\newcommand{\psep}{{\sc 0/1-psep}\xspace}
\newcommand{\sep}{{\sc 0/1-sep}\xspace}
\newcommand{\fopt}{{\sc 0/1-testopt\xspace} }

\newcommand{\hpp}{\mathrm{HPP}}
\newcommand{\nodes}{\mathcal{V}}
\newcommand{\vol}{\mathrm{vol}}
\newcommand{\diag}{\mathrm{diag}}
\newcommand{\arcs}{\mathcal{A}}
\newcommand{\edges}{\mathcal{E}}
\newcommand{\paths}{\mathscr{P}}
\newcommand{\cycles}{\mathcal{C}}




\newcommand{\K}{{\mathcal K}}
\newcommand{\A}{{A}}
\newcommand{\B}{{B}}
\newcommand{\T}{\mathscr{T}}
\newcommand{\eE}{\mathscr{E}}
\newcommand{\eS}{\mathscr{S}}
\newcommand{\eP}{\mathscr{P}}
\newcommand{\eM}{\mathscr{M}}



\newcommand{\transp}{^{\mathrm{T}}}

\newcommand{\smallmat}[1]{\left( \begin{smallmatrix} #1 \end{smallmatrix}\right)}

\newcommand{\mat}[1]{ \begin{pmatrix} #1 \end{pmatrix}}
\newcommand{\smat}[1]{ \big(\begin{smallmatrix} #1 \end{smallmatrix}\big)}

\newcommand{\pc}{\mathscr{P}}
\newcommand{\ob}{\mathscr{O}}
\newcommand{\odds}{\mathscr{W}}
\newcommand{\up}{\mathscr{U}}
\newcommand{\ef}{\mathscr{F}}
\newcommand{\eh}{\mathscr{H}}
\newcommand{\ev}{\mathscr{V}}
\newcommand{\ec}{\mathscr{C}}
\newcommand{\eu}{\mathscr{U}}

\newcommand{\lex}{\mathrm{lex}}

\renewcommand{\leq}{\leqslant}
\renewcommand{\geq}{\geqslant}









\newcommand{\linhull}{\mathrm{lin.hull}}
\newcommand{\affhull}{\mathrm{affine.hull}}
\newcommand{\charcone}{\mathrm{char.cone}}
\newcommand{\cone}{\mathrm{cone}}
\newcommand{\rank}{\mathrm{rank}}
\newcommand{\wb}[1]{\overline{#1}}



\usepackage{enumerate}

      
\institute{\'Ecole Polytechnique F\'ed\'erale de Lausanne}
\lecture{Discrete Optimization}
\faculty{Prof. Friedrich Eisenbrand}
\term{Spring 2026}
\publishdate{March 3, 2026}
\duedate{ }
\problemset{Problem Set~2}

\begin{document}
\makeheader

\begin{enumerate}[1)]

\item Consider the problem
\begin{align*}
    \min &\quad 2x_1 + 3|x_2-10| \\
    \text{subject to} &\quad |x_1+2| + |x_2| \leq 5,
\end{align*}
and formulate it as a linear programming problem.

\item A meat packing plant produces 480 hams, 400 pork bellies, and 230 picnic hams every day; each of these products can be sold fresh or smoked. 
The total number of hams, pork bellies, and picnic hams that can be smoked during a normal working day is 420; in addition, up to 250 products can be smoked on overtime at a higher cost.
The \emph{net} profits are as follows:
\begin{center}
\begin{tabular}{cccc}
\hline \\
& Fresh & \shortstack{Smoked on \\ regular time} & \shortstack{Smoked on \\ overtime} \\
\cline{2-4}
Hams & \$8 & \$14 & \$11 \\
Pork bellies & \$4 & \$12 & \$7 \\
Picnic hams & \$4 & \$13 & \$9 \\
\hline
\end{tabular}
\end{center}

For example, the following schedule yields a total net profit of \$9,965:
\begin{center}
\begin{tabular}{cccc}
\hline \\
& Fresh & \shortstack{Smoked on \\ regular time} & \shortstack{Smoked on \\ overtime} \\
\cline{2-4}
Hams & 165 & 280 & 35 \\
Pork bellies & 295 & 70 & 35 \\
Picnic hams & 55 & 70 & 105 \\
\hline
\end{tabular}
\end{center}
The objective is to find the schedule that maximizes the total net profit. Formulate this problem as a linear programming problem in the standard form.

\item  \label{item:ex-9}
Consider the unit ball $B_n = \{ x ∈ ℝ^n : \|x\|_2≤1\}$. Show that the set of extreme points of $B$ is the sphere $S^{(n-1)} = \{ x ∈ ℝ^n : \|x\|_2 =1\}$.

\item \label{item:10}
  A \emph{line} is a set $L = \{ x ⋅ d +t : x ∈ ℝ\} ⊆ ℝ^n$ where $d,t ∈ ℝ^n$ $d ≠0$. Show the following.

  A non-empty polyhedron $P = \{ x ∈ ℝ^n : Ax ≤ b\} ⊆ ℝ^n$ contains a line if  and only if $\rank(A) <n$.

% \item \label{item:11} Two different vertices $v_1 ≠ v_2$ of a polyhedron $P = \{ x ∈ ℝ^n : Ax≤b\}$ are called \emph{adjacent}, if there exists a subsystem $A'x ≤ b'$ of $Ax≤b$ with
% \begin{enumerate}[i)] 
% \item $A'v_1 = b'$ and $A'v_2 = b'$  and
% \item $\rank(A') = (n-1)$. 
% \end{enumerate}

% Show that there exists a valid inequality $c^Tx ≤ δ$ of $P$ with
% \begin{displaymath}
%  \left( P ∩ \{ x ∈ ℝ^n : c^Tx = δ \} \right) = \conv\{v_1,v_2\}. 
% \end{displaymath}

\item Let $\{C_i\}_{i\in I}$ be a family of convex subsets of $\setR^n$.
  Show that the intersection $\bigcap_{i\in I} C_i$ is convex.
\item Show that the set of feasible solutions  of a linear program  is
  convex. \label{conv:item:1}
  
%   \item Let $$P= \{x: Ax≤b\}.$$ 
%   Let $A^=$ denote the set of rows of $A$ such that for all $x \in P$, $A^= x = b^=$ such that the rows indexed by $A^=$ are satisfied with equality in $P$. 
%   Prove that 
% $$\text{affine-hull}(P) = \{x∈\setR^n : A^=
% x= b^=\}= \{x∈\setR^n : A^=
% x≤b^=\}.$$



\end{enumerate}



  

\end{document}

%%% Local Variables:
%%% mode: latex
%%% TeX-master: t
%%% End:
