\documentclass[border=12pt,pstricks]{standalone}
\usepackage{pstricks}

\usepackage{pst-node}
\usepackage{pst-coil}

\newgray{vlg}{.85}
\newgray{vvlg}{.95}

\usepackage{amsmath}

\DeclareMathOperator{\size}{size}
\DeclareMathOperator{\conv}{conv}
\newcommand{\SV}{\mathrm{SV}}
\newcommand{\bigO}{O}
\newcommand{\cut}{\mathrm{cut}}
\newcommand{\LLL}{\mathrm{LLL}}
\newcommand{\setR}{\mathbb{R}}
\newcommand{\setZ}{\mathbb{Z}}
\newcommand{\setQ}{\mathbb{Q}}
\newcommand{\setC}{\mathbb{C}}
\newcommand{\setN}{\mathbb{N}}
\newcommand{\wt}[1]{\widetilde{#1}}
\newcommand{\opt}{{\sc 0/1-opt}\xspace}
\newcommand{\aug}{{\sc 0/1-aug}\xspace}
\newcommand{\psep}{{\sc 0/1-psep}\xspace}
\newcommand{\sep}{{\sc 0/1-sep}\xspace}
\newcommand{\fopt}{{\sc 0/1-testopt\xspace} }

\newcommand{\hpp}{\mathrm{HPP}}
\newcommand{\nodes}{\mathcal{V}}
\newcommand{\vol}{\mathrm{vol}}
\newcommand{\diag}{\mathrm{diag}}
\newcommand{\arcs}{\mathcal{A}}
\newcommand{\edges}{\mathcal{E}}
\newcommand{\paths}{\mathscr{P}}
\newcommand{\cycles}{\mathcal{C}}




\newcommand{\K}{{\mathcal K}}
\newcommand{\A}{{A}}
\newcommand{\B}{{B}}
\newcommand{\T}{\mathscr{T}}
\newcommand{\eE}{\mathscr{E}}
\newcommand{\eS}{\mathscr{S}}
\newcommand{\eP}{\mathscr{P}}
\newcommand{\eM}{\mathscr{M}}



\newcommand{\transp}{^{\mathrm{T}}}

\newcommand{\smallmat}[1]{\left( \begin{smallmatrix} #1 \end{smallmatrix}\right)}

\newcommand{\mat}[1]{ \begin{pmatrix} #1 \end{pmatrix}}
\newcommand{\smat}[1]{ \big(\begin{smallmatrix} #1 \end{smallmatrix}\big)}

\newcommand{\pc}{\mathscr{P}}
\newcommand{\ob}{\mathscr{O}}
\newcommand{\odds}{\mathscr{W}}
\newcommand{\up}{\mathscr{U}}
\newcommand{\ef}{\mathscr{F}}
\newcommand{\eh}{\mathscr{H}}
\newcommand{\ev}{\mathscr{V}}
\newcommand{\ec}{\mathscr{C}}
\newcommand{\eu}{\mathscr{U}}

\newcommand{\lex}{\mathrm{lex}}

\renewcommand{\leq}{\leqslant}
\renewcommand{\geq}{\geqslant}





\newcommand{\R}{\setR}
\newcommand{\Z}{\setZ}
\newcommand{\N}{\setN}



\newcommand{\linhull}{\mathrm{lin.hull}}
\newcommand{\affhull}{\mathrm{affine.hull}}
\newcommand{\charcone}{\mathrm{char.cone}}
\newcommand{\cone}{\mathrm{cone}}
\newcommand{\rank}{\mathrm{rank}}
\newcommand{\wb}[1]{\overline{#1}}




\begin{document}

 \begin{pspicture}(0,3)(5,7)
%    \showgrid
%    \cnodeput(0,10){v1}{$v_1$}
    \small 
%    \psset{unit=.8cm}
    
    \cnodeput(1,5){a}{$a$}
%    \rput(0,10){$a$}

    \cnodeput(2,6){b}{$b$}
    \cnodeput(4,6){c}{$c$}

    \cnodeput(2,4){d}{$d$}
    \cnodeput(4,4){e}{$e$}

%     \cnodeput(2,6){b}{$b$}
%     \cnodeput(4,6){c}{$c$}
    
    
    \ncline{->}{a}{b}
    \Aput{\red{$3$}}
    \ncline{->}{a}{d}
    \Bput{\red{$0$}}
    \ncline{->}{b}{c}
    \Aput{\red{$0$}}
    
    \ncline{->}{d}{c}
    \Bput{\red{$2$}}

    \ncline{->}{b}{d}
    \Bput{\red{$3$}}
    
    \ncline{->}{d}{e}
    \Bput{\red{$0$}}


     \pnode(.2,5){in}
     \rput(0,5){$3$}
     \ncline[linewidth=3pt]{->}{in}{a}


     \pnode(2,3.2){o1}
     \rput(2,3){$1$}
     \ncline[linewidth=3pt]{->}{d}{o1}

     \pnode(4.8,6){o2}
     \rput(5,6){$2$}
     \ncline[linewidth=3pt]{->}{c}{o2}       
%     \pnode(-1,10){iv}
%     \ncline[linewidth=3pt]{->}{iv}{v1}
%     \rput(-1.5,10){$2$}
  \end{pspicture}
      

\end{document}
%%% Local Variables: 
%%% mode: latex
%%% TeX-master: t
%%% End: 
