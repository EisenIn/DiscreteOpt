\chapter{The development of the simplex method}
\label{cha:introduction}



In this chapter, we describe an algorithm which solves linear
programming problems $\min\{c^Tx \colon Ax = b, \, x\geq0\}$ in
equation standard form, where $A \in \setR^{m\times n}$ has full row-rank, i.e.,
the rows of $A$ are linearly independent. Our presentation follows
closely the one in~\cite{BertsimasTsitsiklis97}. 




\section{Equation standard form}
\label{sec:equat-stand-form}

We have seen in Chapter~\ref{cha:introduction-1} that each linear
program can be described in inequality standard form, i.e., as
$\max\{c^Tx \colon Ax\leq b\}$. The simplex algorithm is conveniently
described for linear programs in \emph{equation standard form} $\min\{c^Tx \colon
Ax = b, \, x\geq0\}$.  
%, where one can assume that the rows of $A$ are linearly independent. 

The next example shows how to transform a linear program in inequality
standard form in equation standard form.  Consider the linear program 
\begin{equation}
\label{eq:28}
  \begin{array}{lrcl}
    \text{maximize}   & 3 x_1 - 2x_2 \\
    \text{subject to} & 2x_1 - x_2 & \leq & 4 \\
                      & x_1 + 3x_2 & \leq & 5 \\
                      & -x_1 & \leq & 0.     
  \end{array}
\end{equation}
Our goal is to transform this linear program into one of the form
$\min\{c^Tx \colon Ax = b, \, x\geq0\}$.  The objective function can be
re-written as $\min -3x_1 + 2x_2$.  Furthermore, we observe that the
last inequality of~\eqref{eq:28} can be re-written as $x_1\geq0$. The
variable $x_2$ is not bounded from below. The trick is to split $x_2$
into $x_2 = x_2^+ - x_2^-$, where $x_2^+,x_2^- \geq0$. The linear
program then becomes 
\begin{equation}
  \label{eq:29}
  \begin{array}{llcl}
    \text{minimize}   & -3 x_1 + 2x_2^+ -2 x_2^- \\
    \text{subject to} & 2x_1 - x_2^+ + x_2^- & \leq & 4 \\
                      & x_1 + 3x_2^+ -3 x_2^- & \leq & 5 \\
                      & x_1, x_2^+,x_2^- &\geq &0.
  \end{array}
\end{equation}
Next we introduce two new \emph{slack variables} $z_1,z_2\geq0$, which
model $z_1 = 4- (2x_1 - x_2^+ + x_2^-)$ and $z_2 = 5 - (x_1 + 3x_2^+
-3 x_2^-)$ and obtain a linear program in equation
standard form which is equivalent to~\eqref{eq:28} 
\begin{displaymath}
  \begin{array}{llcl}
    \text{minimize }   & -3 x_1 + 2x_2^+ -2 x_2^- \\
    \text{subject to }\, & 2x_1 - x_2^+ + x_2^- + z_1 & = & 4 \\
                      & x_1 + 3x_2^+ -3 x_2^- + z_2  & = & 5 \\
                      & x_1, x_2^+,x_2^-,z_1,z_2 & \geq &0.
  \end{array}
\end{displaymath}
%
In matrix notation, we can describe the transformation procedure from
above as follows. The input is a linear program in inequality standard
form $\max\{c^Tx \colon x \in \setR^n, \,Ax\leq b \}$ with $A \in \setR^{m\times n}$. From this, one creates the
equivalent linear program 
\begin{displaymath}
  \begin{array}{ll}
    \min & -c^T x^+ + c^T x^-  \\
    & A x^+ - A x^- + I_m z =  b\\
     &x^+, x^-, z  \geq 0  \\
    & x^+,x^- \in \setR^n, z \in \setR^m,     
  \end{array}
\end{displaymath}
where $I_m \in \setR^{m\times m}$ is the $m\times m$ identity matrix, i.e. for
$1\leq i,j\leq m$ one has 
\begin{displaymath}
  I_m(i,j) = 
  \begin{cases}
    1, & \text{if } i=j\\
    0 & \text{otherwise.}
  \end{cases}
\end{displaymath}


\subsubsection*{The full row-rank assumption}

We recall some definitions and facts from linear algebra.   For $A \in \setR^{m\times n}$
the \emph{row-rank} of $A$ is the maximum number of linearly independent
rows. The \emph{column-rank} of $A$ is the maximum number of linearly
independent columns. The row-rank and column rank of $A$ are
equal. This number is the \emph{rank} of $A$ and is denoted by
$\rank(A)$. 

\begin{definition}
  \label{def:8}
  A matrix $A \in \setR^{m\times n}$ has \emph{full row-rank}, if $\rank(A) =
  m$. Similarly, $A$ has \emph{full column-rank}, if $\rank(A) =n$. 
\end{definition}

We now show that  we furthermore can assume that the matrix $A$ in the
linear  program 
\begin{equation}
  \label{eq:30}
  \min\{c^Tx \colon x \in \setR^n, \, Ax =b,\, x\geq0\}
\end{equation}
has full row-rank, i.e., the rows of $A$ are linearly independent. 
To see this, suppose that $A\in \setR^{m\times n}$ does not have full row-rank. Then
there is a row $a_j$ of $A$ which is in the span of the other rows,
i.e., one has
\begin{displaymath}
  a_j = \sum_{\substack{i=1\\ i\neq j}}^m \lambda_i a_i \text{ with suitable numbers }
  \lambda_i \in \setR, i\in \{1,\ldots,m\} \setminus\{j\}. 
\end{displaymath}
If $\sum_{\substack{i=1\\ i\neq j}}^m \lambda_i  b(i) = b(j)$, then one has for all  $x \in \setR^n$
with 
$a_i^Tx = b(i), i=1,\ldots,m, i\neq j$ also  $a_j^Tx =  b(j)$ which means that the
$j$-th  equation in  $Ax = b$ can be removed. 
If $\sum_{\substack{i=1\\ i\neq j}}^m \lambda_i  b(i) \neq b(j)$, then there does not exist an $x \in
\setR^n$ with $Ax = b$ and the  LP~(\ref{eq:30}) is infeasible. 

\begin{example}
  \label{sec:full-row-rank}
  Consider the linear program 
  \begin{equation}
    \label{eq:46}
    \begin{array}{rl}
      \min  x_1 + 2 \,x_2 + x_3 + 4 x_4 \\
      x_1 + x_2 + 2 x_3 + 3 x_4 & = 5 \\
      x_1 + 2 x_2 + x_3 + 4 x_4 & =7 \\
      3x_1 +4x_2 + 5 x_3 + 10 x_4& = 17\\
      x_1,x_2,x_3,x_4 \geq0. 
    \end{array}
  \end{equation}
  The third equation is can be obtained by multiplying the first
  equation with $2$ and adding it to the second equation. This linear
  program is not of full row-rank and it is equivalent to the linear
  program 
  \begin{equation}
    \label{eq:47}
    \begin{array}{rl}
      \min  x_1 + 2 \,x_2 + x_3 + 4 x_4 \\
      x_1 + x_2 + 2 x_3 + 3 x_4 & = 5 \\
      x_1 + 2 x_2 + x_3 + 4 x_4 & =7 \\
      x_1,x_2,x_3,x_4 \geq0. 
    \end{array}
  \end{equation}
  which is of full row-rank. The linear program 
 \begin{displaymath}
    \begin{array}{rl}
      \min  x_1 + 2 \,x_2 + x_3 + 4 x_4 \\
      x_1 + x_2 + 2 x_3 + 3 x_4 & = 5 \\
      x_1 + 2 x_2 + x_3 + 4 x_4 & =7 \\
      3x_1 +4x_2 + 5 x_3 + 10 x_4& = 15\\
      x_1,x_2,x_3,x_4 \geq0. 
    \end{array}
  \end{displaymath}
  is not of full row-rank but is infeasible, since, as before, the
  third row of the constraint matrix is $2$ times the first row plus
  the second row, however $2\cdot 5 + 7 =17 \neq 15$. 
\end{example}



\begin{assumption} 
\label{assumption1}
We assume in the following that our linear program which
we want to solve is in equation standard form~\eqref{eq:30}, or in
\emph{standard   form} for short, with $A\in \setR^{m\times n}$ having full
row-rank. 
\end{assumption}



\section{Bases and basic solutions}
\label{sec:bases-basic-solut}


We use the following notation.  For an index set
$I=\{i_1,\ldots,i_k\}\subseteq\{1,\ldots,n\}$, with $i_1<i_2<\cdots<i_k$, $A_I$ denotes
the $m\times k$ matrix whose $j$-th column is $i_j$-th column $a^{i_j}$ of
$A$ for $j=1,\ldots,k$.
Similarly $c_I\in \setR^k$ denotes the vector whose $j$-th component is
$c(i_j)$, $j=1,\ldots,k$.   
\begin{example}
  \label{ex:-2}
  Let $A =\smat{2 & 3 & 1 \\ 4 & 5 & 6}$ and $I = \{2,3\}$, then $A_I
  = \smat{3&1\\5&6}$.   
\end{example}
The proof of the next lemma is very similar to
the proof of Carath\'eodory's theorem, Theorem~\ref{conv:thr:7}.
It will help us to determine an optimal solution to~\eqref{eq:30}.



  \begin{lemma}
    \label{lem:8}    
    Let $x^*$ be a feasible solution of the linear
    program~\eqref{eq:30} and let $J = \{i \colon x^*(i)>0\}$ be the index
    set corresponding to those components of $x^*$ which are strictly
    positive. There exists a procedure that either asserts that the
    linear program is unbounded or computes a solution $\wt{x}$
    of~\eqref{eq:30} such that
    \begin{enumerate}[i)]
    \item     the index set $\wt{J}= \{ i \colon \wt{x}(i)>0\}$ is
       contained in $J$, $\wt{J}\subseteq J$,
    \item $A_{\wt{J}}$ has full column-rank,
    \item $c^Tx^* \geq c^T\wt{x}$. 
    \end{enumerate}

   % If the linear program~\eqref{eq:30}  is feasible and  bounded,
%    then there  exists an optimal solution  $x^* \in \setR^n$
%    of~\eqref{eq:30} such that    $A_J$    has full column rank, where
%    $J = \{ j \mid x^*(j)>0\}$.  
  \end{lemma}


  \begin{proof}    
    Let $x^*$ be a feasible solution and suppose that the columns of
    $A_J$ are linearly dependent.  The idea is now to compute a new
    solution $x'$ with $J'=\{ j \colon  x'(j)>0\} \subset J$ and $c^T
    x' \leq c^T x^*$ or to assert that the linear program is
    unbounded.  After at most $n$  repetitions of this procedure,
    one has found a solution $\wt{x}$  which satisfies the condition of the
    theorem or one has asserted that the linear program is unbounded. 
             

    Let $\wb{J}$ be the index set $\wb{J}= \{1,\ldots,n\} \setminus J$.  Since
    $A_J$ does not have full column rank there exists a $d \in \setR^n$
    with $d\neq0$, $A\, d =0$ and $d(j)=0$ for each $j \in \wb{J}$.  Consider
    the points 
    \begin{equation}
      \label{eq:32}
        x^* \pm \lambda\, d  \text{ for } \lambda \in \setR.
    \end{equation}
    Since $A (x^* \pm \lambda d)
    =b$ for each $\lambda \in \setR$, the only way to make  a point~\eqref{eq:32}
    infeasible  is to violate the lower bounds $x^* \pm \lambda d \geq0$. Since
    $x^*_J>0$ and $d_{\wb{J}} =0$ there exists 
    a sufficiently small $\varepsilon^*>0$ such that for each $\lambda \in \setR$ with
    $0 \leq \lambda \leq \varepsilon^*$ the point $x^* \pm \lambda d$ is feasible. 
    
    Suppose that $c^Td \neq0$. By multiplying $d$ with $-1$
    we can assume that $c^Td <0$. If $d\geq0$, then $x^*+\lambda\,d\geq0$ for
    each $\lambda\geq0$ and since $c^T(x^*+\lambda\,d) = c^Tx^* + \lambda c^Td$, we can
    make the objective value of $x^*+\lambda\,d$ as small as we wish. Thus we
    assert that the linear program is unbounded. Otherwise, let $K =
    \{i \colon d(i) <0\}$ be the index set corresponding to the negative
    components of $d$. How large can we choose $\lambda>0$ such $x^* + \lambda
    d$ is still feasible?  For $\lambda>0$, the point  $x^* + \lambda \,d$  is
    feasible if and only if 
    \begin{equation}
      \label{eq:33}
      \lambda \leq \min\{- x^*(k) / d(k) \colon  k \in K\}. 
    \end{equation}
    Let $\lambda_{\max} = \min\{  - x^*(i) / d(i) \colon i \in K\}$ and let $k^*\in
    K$
    be an index, where the minimum is attained. The point
    $x' = x^*+ \lambda_{\max} d$ is feasible, $c^Tx' < c^Tx^*$ and
    since $x'(k^*)=0$ one has with $J' = \{ i \colon x'(i) >0\}$ also $J'\subset J$. 
    
    Suppose now that $c^Td = 0$. By multiplying $d$ with $-1$ if
    necessary, we can assume that $d$ has strictly negative
    components. We proceed as above with the index set $K$ and
    $\lambda_{\max}$ to     construct a feasible  $x' = x^*+ \lambda_{\max} d$ with
    $c^Tx' \leq c^Tx^*$ and  $J' = \{ i \colon x'(i) >0\}\subset J$.     
  \qed
\end{proof}

\begin{example}
  \label{exa:ll}
  Consider the linear program~\eqref{eq:47} and the feasible solution
  $x^* = (1,1,0,1)$. The set $J$ from the proof above is $J =
  \{1,2,4\}$.  The vector $d^T = (-2, -1, 0, 1)$ satisfies $Ad=0$ and
  $d_{\wb{J}} =0$. A largest $\lambda>0$ such that $(1,1,0,1) + \lambda (-2, -1, 0, 1)
  \geq0$ is $\lambda_{\max} = 1/2$ and the new solution is $(1, 1, 0, 1)
  + (1/2)\cdot  (-2, -1, 0, 1) = (0,0.5,0,1.5)$. This is a basic feasible
  solution and the procedure described in the proof of
  Lemma~\ref{lem:8} ends. 
\end{example}



\begin{definition}[Basis, basic solution, basic feasible solution]
  \label{def:13}
   A set $B\subseteq\{1,\ldots,n\}$ with $|B| = m$ and $A_B$ non-singular is
   called \emph{basis}. A vector $x^* \in \setR^n$ is a \emph{basic
     solution}, if there exists a basis $B$ with $x^*_B = A_B^{-1}b$
   and $x^*_{\wb{B}}=0$, where $\wb{B} = \{1,\ldots,n\} \setminus B$. We say that
   $x^*$ is associated to the basis $B$. If
   additionally $x^*\geq0$ holds, then $x^*$ is a \emph{basic feasible
     solution}. 
\end{definition}


  \begin{lemma}
    \label{lem:3} Let  $x^* \in \setR^n$ be a feasible solution and let $J
    = \{ i \colon x^*(i) \neq 
    0\}$. If the columns of $A_J$ are linearly independent, then $x^*$
    is a basic feasible solution.
  \end{lemma}
  \begin{proof}
    Augment $J$ to index set $B\supseteq J$ such that $A_B$ is non-singular.
    One has $A_Bx^*_B=b$ and $x^*(j)=0$ for all $j \notin B$, thus $x^*$
    is a basic solution.  \qed
  \end{proof}
  

In exercise~\ref{item:5} you are to prove that there could be several
bases which are associated to a basic feasible solution $x^*$. 


Exercise~\ref{ex:16} shows that there exist optimization problems
$\min\{c^Tx  \colon x \in K\}$ where $K\subseteq\setR^n$ is closed and bounded which
are bounded but do not have optimum solutions. A corollary of
Lemma~\ref{lem:8} is that this cannot happen for linear programs. 

\begin{corollary}
  \label{co:2}
  A bounded and feasible linear program has an optimal basic feasible
  solution.  
\end{corollary}

\begin{proof}
  Lemma~\ref{lem:8} together with Lemma~\ref{lem:3} shows that for
  each feasible $x^*$, there exists a basic feasible solution with an
  objective function value which is at most the objective function
  value of $x^*$. 
  Since the number of basic feasible solutions is finite, 
  a bounded linear program has an optimal basic feasible solution.

\end{proof}



\section{A naive algorithm for linear programming} 
 \label{sec:naive-algor-line}
 
 Suppose that we have a bounded linear program $\min\{c^Tx \mid x \in
 \setR^n, \,Ax = b,\, x\geq0\}$ in standard form. We can already describe a
 finite algorithm which computes an optimal solution, if the linear
 program is feasible. Lemma~\ref{lem:8} tells us that, if there exists
 an optimal solution, then there exists also an optimal basic feasible
 solution. 

 
The algorithm maintains a value which reflects the \emph{best
  objective value} observed so far. In the beginning this is $\infty$. 

We now enumerate all index sets $B \in \binom{n}{m}$ and test, whether
$A_B$ is non-singular with Gaussian elimination. If this is the case,
we compute $x^*_B = A_B^{-1}b$ and check whether $x^*_B\geq0$. If this
is also the case, we check whether $c_B^Tx^*_B$ is smaller than the
previously best observed objective value and update this best
objective function value if this is again the case.

This algorithm is however very inefficient. The enumeration of all
index sets $B \in \binom{2m}{m}$ requires already at least $2^m$
steps. 
If $m$ is large this is enormous, see exercise~\ref{item:6}. 
The simplex algorithm is based on a smarter principle.
It \emph{improves} a basic feasible solution, by replacing it by
another basic feasible solution which is in the neighborhood of the
previous one. We will describe this next.
 



\section{Simplex method: First steps}
\label{sec:simplex-method}

 Consider the linear
programming problem 
\begin{equation}
  \label{eq:40}
  \begin{array}{r}
  \begin{matrix}               
    \min \quad \quad  & 2 x_1 & + & 3 x_2 & +&  4 x_3 & +&  4 x_4  & & \\
                      & x_1   & + & 2 x_2 & + & 3 x_3 & + & 4 x_4 & =&  3 \\
                      & x_1   & + & x_2  & + & x_3 & + & x_4 & = & 2                      \\                          
  \end{matrix} \\
  x_1,x_2,x_3,x_4 \geq0. 
\end{array}
\end{equation}
The first two columns of the constraint matrix $A$ are linearly
independent. Let us now compute $x^* \in \setR^n$ such that $x^*_B =
A_B^{-1} \cdot \smat{3\\2}$. For this, we subtract the second row of the 
system 
\begin{displaymath}
  \begin{matrix}    
    x_1   & + & 2 x_2 & + & 3 x_3 & + & 4 x_4 & =&  3 \\
    x_1   & + & x_2  & + & x_3 & + & x_4 & = & 2
    \\                          
  \end{matrix}
\end{displaymath}
from the first, to obtain 
\begin{displaymath}
  \begin{matrix}    
         &   &  x_2 & + & 2 x_3 & + & 3 x_4 & =&  1 \\
    x_1   & + & x_2  & + & x_3 & + & x_4 & = & 2
    \\                          
  \end{matrix}
\end{displaymath}
Then we subtract the first row from the second, to obtain 
\begin{displaymath}
  \begin{matrix}    
         &  &  x_2 & + & 2 x_3 & + & 3 x_4 & =&  1 \\
    x_1   & + &     & + & -1 x_3 & + & -2x_4 & = & 1
    \\                          
  \end{matrix}
\end{displaymath}
Finally we swap the first and second row, and obtain the system 
\begin{displaymath}
  \begin{matrix}       
    x_1   & + &     & + & -1 x_3 & + & -2x_4 & = & 1    
    \\                          
    &   &  x_2 & + & 2 x_3 & + & 3 x_4 & =&  1 
  \end{matrix}
\end{displaymath}

\begin{definition}[Elementary row-operations]
  \label{def:14}
  Let $A\in \setR^{m\times n}$ be a matrix. An \emph{elementary row operation}
  on $A$ is the addition of a multiple of one row, to another row. 
\end{definition}

The following lemma is known from linear algebra. 
\begin{lemma}
  \label{lem:12}
  Let $A,A' \in \setR^{m\times n}$ and $b , b' \in \setR^m$. If the matrix
  $\left[A'|b'\right]$ results from  $\left[A|b\right]$ by a finite
  series of elementary row operations, then  $\{x \in  \setR^n \colon Ax=b\} =
  \{x \in  \setR^n \colon A'x=b'\}$. 
\end{lemma}

\noindent 
In other words, we have re-written our linear program from above as 
\begin{displaymath}
 \begin{array}{r}
  \begin{matrix}               
    \min \quad  & 2 x_1 & + & 3 x_2 & +&  4 x_3 & +&  4 x_4  & & \\
 &      x_1   & + &     & + & -1 x_3 & + & -2x_4 & = & 1    
    \\                          
&    &   &  x_2 & + & 2 x_3 & + & 3 x_4 & =&  1 
  \end{matrix} \\
  x_1,x_2,x_3,x_4 \geq0. 
\end{array}
\end{displaymath}  
From this representation, we can easily read off the basic feasible
solution $x_1=1, x_2=1, x_3 = 0, x_4 = 0$ associated to $B =
\{1,2\}$. We now consider the system
\begin{displaymath}
   \begin{matrix}               
    2 x_1 & + & 3 x_2 & +&  4 x_3 & +&  4 x_4  & =  & 0 \\
    \hline
      x_1   & + &     & + & -1 x_3 & + & -2x_4 & = & 1    
    \\                          
    &   &  x_2 & + & 2 x_3 & + & 3 x_4 & =&  1 
  \end{matrix}. 
\end{displaymath}
If we subtract 2-times the second row and 3-times the third row from
the first row, we obtain the system 
\begin{displaymath}
  \begin{matrix}    
 & + &       & + &     0 x_3 & + &    -1x_4 & = &
-5\\ \hline 
1  x_1 & + &    0 x_2 & + &   -1x_3 & + &    -2 x_4 & = &    1\\
     0x_1 & + &     1x_2 & + &     2x_3& + &     3x_4 & = &     1
   \end{matrix}                             
\end{displaymath}
We have eliminated the basic variables in the first row. The entry
behind the equality sign of the first row is the negative of the
objective value of the basic feasible solution $(1,1,0,0)$. We write
this in a more compact form as follows 
\begin{equation}
  \label{eq:36}
  \begin{tabularx}{.3 \textwidth}{XXXX|X}
    0   &      0 &     0  &    -1   &   -5\\ \hline 
    1   &    0  &    -1  &    -2 &      1\\
    0 &     1 &     2&     3  &     1
  \end{tabularx}
\end{equation}
%
The general interpretation of this approach is as follows. We
start with 
\begin{displaymath}
   \begin{array}{c|c}
    c^T  & 0 \\ \hline 
   A        & b
  \end{array}
\end{displaymath}
and, given a basis $B$ of $A$ compute 
\begin{equation}
  \label{eq:37}
  \begin{array}{c|c}
    c^T - c_B^TA_B^{-1}A & -c_B^Tx^*_B \\ \hline 
    A_B^{-1}A        & x^*_B.
  \end{array}
\end{equation}
\begin{definition}[Tableau, reduced costs]
  The matrix \eqref{eq:37} is called the \emph{tableau} of the basis
  $B$. The tableau is \emph{feasible} if $x^*_B\geq0$. The first row of
  the tableau is the $0$-th row. The remaining $m$ rows constitute the
  \emph{system-matrix} of the tableau. 

  The vector $\overline{c}^T_B = c^T- c_B^TA_B^{-1}A$ is called the \emph{reduced cost vector} of $B$. For
  $i \in \{1,\ldots,n\}$ $\overline{c}_B(i)$ is the reduced cost of the
  variable $x_i$. 
\end{definition}
Now back to our concrete example. Is the solution $(1,1,0,0)$ optimal?
The fact that the reduced cost of $x_4$ is negative suggests to make a
\emph{basis change}. We want to include $x_4$ into the basis, since
then the elimination of the variable $x_4$ in the $0$-th row will make
the element in the top-right corner larger and, remembering that this
entry is the negative of the objective function value, we would have
obtained  a
better solution. % We increase the  value of $x_4$ from $0$
%to $\lambda>0$. How do we have to adapt $x_1$ and $x_2$?  We have to
%maintain the following conditions  
%\begin{displaymath}
%  \begin{array}{c}
%    x_1 -2 \lambda =1 \\
%    x_2 + 3\lambda =1
%  \end{array}
%\end{displaymath}
%which implies $x_1 = 1 + 2\lambda$ and $x_2 = 1- 3 \lambda$. We want to increase
%$\lambda$ by the maximum amount. The only restriction that we have to
%respect is that $x_1$ and $x_2$ must remain nonnegative. We could only
%violate the constraint $x_2\geq0$ and this, only if $\lambda>1/3$.
If $x_4$ should enter the basis, then which variable should leave,
$x_1$ or $x_2$? If $x_1$ should leave the basis, then we need to apply
elementary row-operations to turn the gray column below into the first
unit vector and if $x_2$ should leave the basis, then this column should
become the second unit vector. 

\begin{equation}
\label{eq:45}
  \begin{tabularx}{0.5 \textwidth}{AAAB|X}
\rowcolor{white}    0   &      0 &     0  &    -1   &   -5\\ \hline 
    1   &    0  &    -1  &    -2 &      1\\
    0 &     1 &     2&     3  &     1
  \end{tabularx}
\end{equation}
If $x_1$ leaves the basis, then the tableau becomes infeasible. This
is because the first entry of the gray column is negative.  We decide
therefore to make the following basis change $B = \{1,2\} \to B' =\{ 1, 4\}$. To
compute the new tableau for $B'$ we multiply the last row by $1/3$ 
and  add $2$-times the last row to the second
and add the last row to the first row of the tableau, to obtain 
\begin{equation}
  \label{eq:39}
  \begin{array}{cccc|c}
       0&   1/3&   2/3&     0& -14/3 \\ \hline
       1&    2/3&   1/3&     0&   5/3\\
     0&   1/3&   2/3&     1&   1/3
   \end{array}
 \end{equation}
%
The next lemma shows that the above is indeed the tableau of $B = \{1,4\}$.
\begin{lemma}
  \label{lem:13}
  Let $B\subset\{1,\ldots,n\}$ be an index set %and let  linear program $\min\{c^Tx \colon x \in \setR^n,
%  \, Ax=b, \, x\geq0\}$ 
and    consider a matrix 
  \begin{equation}
    \label{eq:38}
    \begin{array}{c|c}
      d^T & \beta \\ \hline 
      Q & q 
    \end{array}
  \end{equation}
  where 
  \begin{enumerate}[i)]
  \item $[Q|q]$ can be obtained from $[A|b]$ via a series
      of elementary row operations 
  \item $[d^T|\beta] = [c^T | 0] + v^T$, where $v^T$ is in the row-span
    of $[A|b]$ 
  \item $d_B=0$ 
  \item $Q_B = I_m$,
  \end{enumerate}
  then $B$ is a basis and \eqref{eq:38} is the tableau of $B$.
\end{lemma}



%The general description of we have done so far is the following. We
%consider the system 
%\begin{displaymath}
%  \begin{array}{c}
%  c^Tx = 0 \\ \hline 
%  Ax = b
%\end{array}
%\end{displaymath}
%and a basis $B$. Then we multiply $A$ by $A_B^{-1}$. The result is
%that $(A_B^{-1}A)_B = I_m$. Let $B = \{j_1,\ldots,j_m\}$ be the basis with
%$j_1<\cdots<j_m$. The $i$-th row of $A_B^{-1}A$ has a $1$ in the $j_i$-th
%column.  Next we subtract for each $j_i \in B$,  $c(j_i)$ times the  $i$-th
%row of $(A_B^{-1}A)_B$ from the row above the line to obtain 
%\begin{equation}
%  \label{eq:34}
%  \begin{array}{c}
%    (c^T - c_B^T A_B^{-1}A) x = -c_B^T x^*_B \\ \hline 
%  A_B^{-1} A x = x^*_B   
%  \end{array}
%\end{equation}


\section{The dual linear program}
\label{sec:dual-linear-program}

Our guess is that the tableau~\eqref{eq:39} is optimal. To confirm
this suspicion, we introduce the concept of the dual
linear program. 

\begin{definition}[Dual of a linear program in standard form]
  \label{def:15}
  Let $\min\{c^Tx \colon x \in \setR^n, \, Ax=b, \, x\geq0\}$ be a linear
  program in standard form. The linear program $\max\{ b^Ty \colon y \in
  \setR^m, \,A^Ty \leq  c\}$ is  the \emph{dual linear program} of
  $\min\{c^Tx \colon x \in \setR^n, \, Ax=b, \, x\geq0\}$.  The linear program
   $\min\{c^Tx \colon x \in \setR^n, \, Ax=b, \, x\geq0\}$ is  the
  \emph{primal linear program}.  
\end{definition}

\begin{example}
  \label{sec:dual-linear-program-1}
  The dual linear program of \eqref{eq:40} is 
  \begin{equation}
    \label{eq:41}
    \begin{matrix}
      \max \quad &  3 y_1 & + & 2 y_2 & & \\
      4 y_1 & + & y_2 & \leq & 4 \\
      3 y_1 & + & y_2 & \leq & 4\\
      2 y_1 & + & y_2 & \leq & 3\\
      y_1 & +& y_2 & \leq & 2
    \end{matrix}
  \end{equation}
\end{example}

In our example of the simplex method above, we have arrived at a basis
$B = \{1,4\}$ whose reduced cost vector did not have a negative
entry. The next theorem reveals  that such a basis, if it is
feasible, also yields an optimal solution. 


\begin{theorem}[Weak duality]
  \label{thr:15}
  Let $x^*$ and $y^*$ be feasible solutions of the primal and dual
  respectively, then $c^Tx^* \geq b^Ty^*$. 
\end{theorem}

\begin{proof}
  We have 
  \begin{equation}
    \label{eq:42}
    b^T y^* = ({x^*}^TA^T) y^* = {x^*}^T (A^T y^*) \leq {x^*}^T c = c^Tx^*  % = c^T y^* 
  \end{equation}
 where the inequality in \eqref{eq:42} follows from the fact that
 $x^*\geq0$ and $A^T y^* \leq c$. \qed  
\end{proof}


\begin{lemma}
  \label{lem:14}
  Let $x^*$ be a basic solution associated to the basis $B$. 
  If the reduced cost vector of $B$ is non-negative and if
  $x^*_B\geq0$, then $x^*$ is an optimal basic feasible solution. 
\end{lemma}

\begin{proof}
  We show that $y^* = {A_B^{-1}}^T  c_B$ is a feasible solution of the
  dual with objective value $b^Ty^* = c^Tx^*$. The assertion then
  follows from weak duality. We have 
  \begin{eqnarray*}
    A^T y^* & = & A^T {A_B^{-1}}^T  c_B \\
         & \leq & c,
  \end{eqnarray*}
  since $c^T - c_B^T A_B^{-1}A \geq0$. This means that $y^*$ is a
  feasible dual solution. Its objective value 
  is
  \begin{eqnarray*}
    b^Ty^* & = & {y^*}^T A x^* \\
        & = & c_B^T A_B^{-1} A x^* \\
        & = & c_B^T x^*_B\\
        & = & c^Tx^*
  \end{eqnarray*}
  and the assertion follows. \qed 
\end{proof}

This shows that we have found in~\eqref{eq:39}  an optimal basic
feasible solution $(5/3,0,0,1/3)$.  
\begin{definition}[Optimal basis]
  \label{def:2}
  A basis $B$ with $\overline{c}^T = c^T - c_B^T A_B^{-1}A \geq0$
  and with $A_B^{-1}b \geq0$ is   called an \emph{optimal basis}. 
\end{definition}
%
We have seen above that an optimal basis $B$ yields an optimal basic
feasible solution $x^*$ with $x^*_B = A_B^{-1}b$ and
$x^*_{\overline{B}} = 0$. 


\section{A larger example}
\label{sec:larger-example}

Before we describe the simplex method in full generality, we inspect
another example. We consider the linear program 


\begin{equation}
\label{simpleq:44}
  \begin{array}{c}
  \begin{matrix}
    \min \quad &  -5x_1 & - &  3 x_2 & - & 2 x_3 & &    \\
        & 2x_1 &+& 3x_2 &+& x_3 &\leq&5 \\
        & 4x_1 &+& x_2 &+& 2x_3 &\leq& 9 \\
        & 3 x_1 &+& 4 x_2 &+& 2 x_3 &\leq& 10 \\

       \end{matrix}\\
       x_1,x_2,x_3\geq0
\end{array}
\end{equation}

We add variables $x_4,x_5$ and $x_6$ and transform this linear program
in standard form
\begin{equation}
\label{eq:44}
  \begin{array}{r}
  \begin{array}{cccccccccccccc}
    \min \quad &   -5x_1 & - &  3 x_2 & - & 2 x_3  &   &    & & & & & &      \\
               & 2x_1    & + &   3x_2 & + & x_3    & + & x_4 & & & & &
               = & 5  \\
        & 4x_1 &+& x_2 &+& 2x_3 & & & + & x_5  & &   &=& 9 \\
      & 3 x_1 &+& 4 x_2 &+& 2 x_3 & & & & & & + & x_6= & 10 
       \end{array}\\
       x_1,x_2,x_3,x_4,x_5,x_6\geq0
\end{array}
\end{equation}
%
We start with the feasible basis $B = \{4,5,6\}$ and the corresponding
tableau, where the red-columns correspond to basis elements 
\begin{displaymath}
  \begin{tabularx}{0.5\textwidth}{AAARRR|X}   
\rowcolor{white}    -5&-4&-3&0&0&0&0\\\hline 
   2&3&1&1&0&0&5\\
   4&1&2&0&1&0&9\\
   3&4&2&0&0&1&10 
 \end{tabularx}
\end{displaymath}
The reduced cost of $x_1$ is negative, we decide to let $x_1$ enter
the basis. We now want to transform the gray column into a unit
vector such that the tableau is still feasible. The red columns
are the basis-columns. 
\begin{equation}
\label{eq:1}
  \begin{tabularx}{0.5\textwidth}{BAARRR|X}   
\rowcolor{white}    -5&-4&-3&0&0&0&0\\\hline 
   2&3&1&1&0&0&5\\
   4&1&2&0&1&0&9\\
   3&4&2&0&0&1&10 
 \end{tabularx}
\end{equation}
We now need to decide, which element  should leave $B = \{4,5,6\}$. If
it should be $4$, then the gray column should be transformed into
$(1,0,0)^T$. Let us see what happens if we apply elementary
row-operations on the system matrix of the tableau such that the first
column becomes $(1,0,0)^T$. 
\begin{displaymath}
  \begin{tabularx}{0.5\textwidth}{RAAARR|X}   
\rowcolor{white}    0&-4&-3&0&0&0&25/2\\\hline 
   1&3/2&1/2&1/2&0&0&5/2\\
   0&-5&0&-2&1&0&-1\\
   0&2&1/2&-3/2&0&1&1/2 
 \end{tabularx}
\end{displaymath}
The result is the tableau of $\{1,5,6\}$, the objective function value
has improved but the basic solution of $\{1,5,6\}$ is not feasible,
since $x^*(5) = -1$.


What went wrong? Let us abstract from the specific situation and
suppose that we have the following situation. We have a feasible basis
$B = \{B_1,\ldots,B_m\}$ and a tableau  of the basis $B$. The reduced cost
of $j$ are negative and we want to add $j$ to $B$. The picture below
represents this situation, where the $j$-th column is the gray
column. This column is also called the \emph{pivot column}. 

\begin{center}
  \begin{tabularx}{0.5\textwidth}{ABAAA|X}   
       \rowcolor{white} & $\overline{c}(j)$ & &  & & $-c^Tx^*$ \\\hline
       &$ u(1)$  & & & & $x^*(B_1)$  \\
       &$\vdots$  & & & & $\vdots$  \\
       &$ u(i)$  & & & & $x^*(B_i)$  \\
       &$\vdots$  & & & & $\vdots$  \\
       &$ u(m)$  & & & & $x^*(B_m)$  \\
 \end{tabularx}
\end{center}
%
If we now decide to let $B_i$ leave the basis, then we transform the
gray column into the $i$-th unit vector with elementary row-operations
and obtain 

\begin{center}
  \begin{tabular}{cGc|c}   
     \rowcolor{white}   \hspace{2cm} &  $\overline{c}(j)$
     &\hspace{2cm}  & 

$-c^Tx^*$ \\ \hline
&       $ 0 $  &&   $x^*(B_1) -  u(1) \cdot \frac{x^*(B_i)}{u(i)} $  \\
&       $ \vdots $ && $\vdots$  \\
&       $ 0 $  && $x^*(B_{i-1})- u(i-1) \cdot \frac{x^*(B_i)}{ u(i)}    $  \\
&       $ 1$  && $x^*(B_i) / u(i)$  \\
&       $ 0 $  &&  $x^*(B_{i+1})-   u(i+1)\cdot  \frac{x^*(B_i)}{ u(i)}  $  \\
&       $ $  && $\vdots$  \\
&       $ 0 $  &&   $x^*(B_m) - u(m) \cdot \frac{x^*(B_i)}{u(i)}$  
 \end{tabular}
\end{center}

Then we eliminate the $\overline{c}(j)$ by adding $-\overline{c}(j)$
times the $i$-th row of the system matrix to the zeroth row to obtain 



\begin{center}
  \begin{tabular}{cGc|c}   
      \rowcolor{white}   \hspace{2cm} &  $0$
        &\hspace{2cm}  &   $-c^Tx^* - \overline{c}(j)\cdot
       x^*(B_i) / u(i)$ \\ \hline 
   &       $ 0 $  &&   $x^*(B_1) -  u(1) \cdot \frac{x^*(B_i)}{u(i)} $  \\
   &     $ \vdots $ && $\vdots$  \\
   &    $ 0 $  &&  $x^*(B_{i-1})- u(i-1) \cdot \frac{x^*(B_i)}{ u(i)}    $  \\
   &    $ 1$  && $x^*(B_i) / u(i)$  \\
   &    $ 0 $  && $x^*(B_{i+1})-   u(i+1)\cdot  \frac{x^*(B_i)}{ u(i)}  $  \\
   &    $ \vdots$  && $\vdots$  \\
   &    $ 0 $  &&   $x^*(B_m) - u(m) \cdot \frac{x^*(B_i)}{u(i)}$  
 \end{tabular}
\end{center}

Let us deduce a set of conditions, under which these sequence of
operations are allowed and such that the obtained tableau is feasible. 


\begin{enumerate}[a)]
\item $u(i) >0$,\\ since if $u(i) \leq0$, then we either cannot divide by $u(i)$
  or the basis is not feasible since $x^*(B_i) / u(i) <0$.  \label{item:1}
\item $x^*(B_i) / u(i) = \min\{ x^*(B_j) / u(j) \colon B_j \in B, \, u(j)
  >0\}$, \\
  since the condition $x^*(B_j) - u(j) \cdot \frac{x^*(B_i)}{u(i)} \geq0$ is
  satisfied for each $B_j \in B$ if and only if the condition holds. \label{item:2}
\end{enumerate}


Looking back at our example~\eqref{eq:1} we see that
condition~\ref{item:2}) was violated. With $B_1=4,B_2=5,B_3=6$ we see
that $x^*(B_1)/u(1) = 5/2$ and this is larger than $x^*(B_2)/u(2) =
9/4$, which is the minimum of the ratios. We therefore let $B_2=5$
leave the basis and obtain 

\begin{displaymath}
  \begin{tabularx}{0.6\textwidth}{RAARAR|X}   
    \rowcolor{white}   
     0& -11/4&  -1/2&     0&   5/4&     0&  45/4 \\ \hline 
     0&   5/2&     0&     1&  -1/2&     0&   1/2 \\
     1&   1/4&   1/2&     0&   1/4&     0&   9/4 \\
     0&  13/4&   1/2&     0&  -3/4&     1&  13/4
  \end{tabularx}
\end{displaymath}
%
To arrive then at the feasible tableau of the basis $B' = \{1,4,6\}$
we still need to swap the first and second row of the system matrix: 
\begin{displaymath}
  \begin{tabularx}{0.6\textwidth}{RAARAR|X}   
    \rowcolor{white}   
     0& -11/4&  -1/2&     0&   5/4&     0&  45/4 \\ \hline 
     1&   1/4&   1/2&     0&   1/4&     0&   9/4 \\
     0&   5/2&     0&     1&  -1/2&     0&   1/2 \\
     0&  13/4&   1/2&     0&  -3/4&     1&  13/4
  \end{tabularx}
\end{displaymath}

\subsection{One iteration of the simplex method} 
\label{sec:one-iter-simpl}

 We start with a feasible basis  $B \subseteq\{1,\ldots,n\}$ and an associated tableau 
\begin{equation}
  \label{eq:2}
  \begin{array}{c|c}
    \overline{c}^T  & -c_B^Tx^*_B \\ \hline 
    A_B^{-1}A        & x^*_B.
  \end{array}
\end{equation}



\noindent 
{\bfseries The simplex algorithm} starts with a feasible tableau and
keeps iterating the procedure below.



\begin{enumerate}
\item If $\overline{c}\geq0$, then {\bf output optimal basis $B$}
  \label{item:3}
\item Otherwise, let $j$ be an index with $\overline{c}(j)<0$ and let
  $u = A_B^{-1}A^j$ be the $j$-th column of the system matrix of the
  tableau. If $u\leq0$, then {\bf output $-\infty$}, the linear program is
  unbounded. \label{item:7}
\item Otherwise compute the ratios $x^*(B_i)/ u(i)$ for all $i =
  1,\ldots,m$ with $u(i)>0$. Let $i^*$ be an index where this minimum is
  attained. The index $B_{i^*}$ leaves the basis and $j$ enters the
  basis, i.e., the new basis is $B' = B \setminus\{B_{i^*}\} \cup \{j\}$. 
\item Update the tableau to 
  \begin{displaymath}    
  \begin{array}{c|c}
    c^T - c_{B'}^TA_{B'}^{-1}A  & -c_{B'}^Tx^*_{B'} \\ \hline 
    A_{B'}^{-1}A        & x^*_{B'}.
  \end{array}
\end{displaymath}
\end{enumerate}
%
We need to justify that the we correctly assert in step~\ref{item:7} that the linear
program is unbounded if $u\leq0$. 
\begin{theorem}
  \label{thr:1}
  If $\overline{c}(j)<0$  and $u\leq0$, then the linear program is
  unbounded. 
\end{theorem}

\begin{proof}
  Consider the vector $d$ with $d_B = -u$ and $d(j) = 1$, which is
  zero at all other components. Since $A d=0$ and since $d\geq0$, we
  have that for each feasible $x^*$ and for each $\lambda\geq0$, also the
  point $x^* + \lambda d$ is feasible. What is the objective function value
  of $d$? 
  \begin{eqnarray*}
    c^Td & = & c(j) - c_B^T A_B^{-1} A \\
        & = & \overline{c}(j) \\
        & < & 0. 
  \end{eqnarray*}
  This means that  the objective function value of $x^* +
  \lambda d$ which is $c^Tx^* + \lambda c^Td$ can be made as small as we wish by
  increasing $\lambda$, without becoming infeasible.  The linear program is
  unbounded. 
  \qed 
\end{proof}



Let us continue our example. We decide to bring $2$ into the
basis, since $\overline{c}(2) = -11/4 <0$. 

\begin{displaymath}
  \begin{tabularx}{0.8\textwidth}{RBARAR|X}   
    \rowcolor{white}   
     0& -11/4&  -1/2&     0&   5/4&     0&  45/4 \\ \hline 
     1&   1/4&   1/2&     0&   1/4&     0&   9/4 \\
     0&   5/2&     0&     1&  -1/2&     0&   1/2 \\
     0&  13/4&   1/2&     0&  -3/4&     1&  13/4
  \end{tabularx}
\end{displaymath}

We compute the ratios: $9, 1/5, 1$ and decide that the second
 element of $\{1,4, 6\}$ should leave the basis, i.e., we want to have
 the second unit vector in the pivot column. 


\begin{displaymath}
  \begin{tabularx}{0.8\textwidth}{RRBAAR|X}   
    \rowcolor{white}       
    0&      0&   -1/2&  11/10&   7/10&      0&   59/5\\ \hline 
    1&      0&    1/2&  -1/10&   3/10&      0&   11/5\\
    0&      1&      0&    2/5&   -1/5&      0&    1/5\\
    0&      0&    1/2& -13/10&  -1/10&      1&   13/5\\
  \end{tabularx}
\end{displaymath}


After one more iteration we obtain an obtimal feasible tableau

\begin{displaymath}
  \begin{tabularx}{0.8\textwidth}{ARRAAR|X}   
    \rowcolor{white}       
        1&    0&    0&    1&    1&    0&   14\\ \hline 
        0&    1&    0&  2/5& -1/5&    0&  1/5\\
        2&    0&    1& -1/5&  3/5&    0& 22/5\\
       -1&    0&    0& -6/5& -2/5&    1&  2/5\\
  \end{tabularx}
\end{displaymath}





\subsection{What  next?}
\label{sec:further-questions}
 The two main questions that we
are next going to deal with   are the following.  
\begin{enumerate}[I)]
\item How do we find an initially feasible basis? \label{item:8}
\item If we start from a feasible basis, does the simplex method
  terminate? \label{item:9} 
\end{enumerate}




\section{Exercises}

\begin{enumerate}
\item Transform the following linear program into equation standard
  form \label{item:4}
  \begin{displaymath}
    \begin{array}{c}
      \begin{matrix}
        \max \quad &  3x_1& +& 4x_2& +& 2 x_3& & \\
             &  &  & x_2  & + & 2x_3 &\geq & 4 \\ 
             &  2x_1 &+  & x_2 &+  &3x_3 &\leq & 10\\
             &      &   & x_2   &   &    & \leq & 0 
      \end{matrix}
    \end{array}
  \end{displaymath}
\item  \label{item:5}    Provide an example of  a basic solution that can be
  associated to two   different bases. %, thereby showing that the
%  converse of Lemma~\ref{lem:3} is false. 
 
\item This exercise should give you a sense of exponential growth as
  one experiences it with the running time of the naive linear
  programming algorithm. Have a look at this program. What is its
  running time?  Experiment with different inputs and determine the
  largest exponent for which the program needs less running time than
  a day on your computer. Make an estimation on how many rows and
  variables a linear program could have such that the naive algorithm
  for linear programming finishes within a day on your computer.
  \label{item:6} Estimate how many rows and variables a linear needs
  to have such that the naive algorithm does not terminate on your
  computer before our sun will stop to shine. 
\begin{verbatim}
#include <iostream>
#include <cmath>

using namespace std;  

int main()
{
  
  double n; 
  
  cout << "\n Enter the exponent" ;
  cin >> n ;
  long i,j;
  j=0;
  for (i=1; i<= pow(2.0,n); i++)
    j = j+i;
  
  cout << j;

}


\end{verbatim}


\item Prove Lemma~\ref{lem:13}. 
\end{enumerate}




\bibliographystyle{abbrv}

\bibliography{mybib,papers,books,my_publications}

%%% Local Variables: 
%%% mode: latex
%%% TeX-master: "lecture"
%%% End: 
